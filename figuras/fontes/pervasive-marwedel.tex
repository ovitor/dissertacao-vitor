\documentclass{standalone}
\usepackage[utf8x]{inputenc}
\usepackage{tikz}
% \usepackage{pgfplots,pgfplotstable}
\usepackage{color}

\usetikzlibrary{arrows,positioning} 
\definecolor{myorange}{RGB}{230,97,1}
\definecolor{mylightorange}{RGB}{253,184,99}
\definecolor{mypurple}{RGB}{94,60,153}
\definecolor{mylightpurple}{RGB}{178,171,210}
\definecolor{mycolor1}{RGB}{35,139,69}
\definecolor{mycolor2}{RGB}{102,194,164}
\definecolor{mycolor3}{RGB}{178,226,226}
\definecolor{mycolor4}{RGB}{237,248,251}

\begin{document}

\tikzset{
    %Define standard arrow tip
    >=stealth',
    %Define style for boxes
    punkt/.style={
           rectangle,
           rounded corners,
           draw=black, very thick,
           text width=6.5em,
           minimum height=2em,
           text centered},
    % Define arrow style
    pil/.style={
           ->,
           thick,
           shorten <=2pt,
           shorten >=2pt,}
}  

\begin{tikzpicture}[font=\sffamily, node distance=0.2cm, auto,]
   \begin{scope}
        \draw[fill=mycolor4] (0,0) ellipse (3cm and 0.8cm);
        \draw[fill=mycolor4] (-7,0) ellipse (3cm and 0.8cm);
        \draw[fill=mycolor2] (-3.5,-2) ellipse (3cm and 1.2cm);

        \node (A) at (0,0) [align=center]{Tecnologias de Comunicação};
        \node (B) at (-7,0) [align=center]{Sistemas Embarcados};
        \node (C) at (-3.5,-1.8)[align=center]{Computação ubíqua/pervasiva};
        \node at (-3.5,-2.2)[align=center]{\tiny (informação em qualquer lugar e};
        \node at (-3.5,-2.5)[align=center]{\tiny a qualquer momento)};

        \node[above=of A] (A1) {};
        \node[above=of B] (B1) {}; 
        \node[above=of C] (C1) {}; 

        \node[below=of A] (A2) {}; 
        \node[below=of B] (B2) {}; 

        \node[left=of C] (C2) {}; 
        \node[right=of C] (C3) {}; 

        \draw [pil, <->, bend right=45] (A1) edge (B1);
        \draw [pil, bend left=30] (A2) edge (C3);
        \draw [pil, bend right=30] (B2) edge (C2);
    \end{scope}
\end{tikzpicture} 

\end{document}
