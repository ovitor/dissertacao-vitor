\begin{resumo}[Abstract]
 \begin{otherlanguage*}{english}

According to IBGE, the life expectancy of Brazilian people increased from 45,5
to 75,5 years between 1940 and 2015. Because of inherent fragility of age, the
elderly demands more hospital care, with more recovery time which increases the
money that the Estate spend in each patient. The Home Care - health care
modality that treats the patient in his own home - presents as more adequate
option for the patient, besides the reduction of hospital expenses the Home
Care allows an intense contact with relatives. Studies have identified problems
in the scenario of home care such as the difficulty that the actors involved
have: the patient, caregiver, relatives and health team. The caregiver deserves
special attention given that he does not always have the knowledge and skills
necessary to attend to the patient. This work describes the computational view
of these health scenarios and proposes TV Health of Things, a smart solution
for the ADS environment based on the Brazilian digital TV model and the
Internet of Things technology.  This solution has as substrate an embedded
system associated with a digital TV (hub) that collects data from several
sensors existing in the ADS environment, besides serving as an interface for
the patient to several interactive applications.  The data collected feed
intelligent ontology models that allow the local inference (hub) of information
necessary for the decision making of the ADS actors: patient, caregiver, health
team, family, etc.  Another intelligent aspect of TV Health of Things relates
to learning machines via the \nextsaude[] system, of which TV Health of Things
is a component.  A context-aware concept, enriched with concepts from the
OpenIoT (Internet Platform of Things) platform, was defined for this ADS
scenario.  A prototype is already operational with its formally specified
functional and engineering architectures.  Thus, patients, caregivers,
relatives, health teams and managers can selectively receive alerts,
information, etc. via mobile devices and other mechanisms integrated into the
system.  TV Health of Things resulted in an Embrapii-funded project and is
being negotiated for deployment in a health care plan.



\textbf{Keywords}: home care. iot. ubiquitous and pervasive computing.
 \end{otherlanguage*}
\end{resumo}
