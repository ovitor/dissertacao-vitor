\begin{resumo}[Abstract]
 \begin{otherlanguage*}{english}

According to IBGE, the life expectation of Brazillian people increased from
45,5 to 75,5 years between 1940 and 2015. Because of inherent fragility
of age, the elderly demands more hospital care, with more recovery time
which increases the money that the Estate spend in each patient. The
Home Care - health care modality that treats the patient in his own
home - presents as more adequate option for the patient, besides the
reduction of hospital expenses. One of the problems identified in the
scenario of home care it's the need to help the patient and the caregiver,
once that the caregiver not always have necessary knowlegde to attend
the patient. His does not always have the caregiver figure daily or in
some situations, the caregiver doesn't exist. The Information Technology
can provide mechanisms and instruments for this type of service.
Considering that the television is an equipment present in practically all
Brazillian residences, this work proposes TV Health of Things, a smart
solution for the ADS environment based on the Brazilian model of digital TV
and technology Internet of Things. TV Health of Things is a ``Set-Top Box''
(digital converter) system that collects data from various sensors in the
home care environment and serves as an interface to a variety of interactive
applications. These collected data feed ontology models that allow the
inference of information necessary for the decision making of the ADS actors
(patient, caregiver, health team, family, etc). Therefore, the various actors
mentioned can receive alerts, information, etc. via TV, web applications or 
mobile devices integrated into the system. The proposal is a component of 
\nextsaude[], a health interoperability platform based on the OpenEHR
standard. It makes use of archetypes to integrate with the \nextsaude[] platform
and thus to share services available by the platform (regulation, pharmacy
etc). TV Health of Things is a context-aware (\textit{context-aware concept}) 
architecture, enriched with concepts from the \textit {OpenIoT} platform. 
A prototype is already operational and its IoT version has its formally 
specified functional and engineering architectures.


\textbf{Keywords}: home care. iot. ubiquitous and pervasive computing.
 \end{otherlanguage*}
\end{resumo}
