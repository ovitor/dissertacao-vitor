\setlength{\absparsep}{18pt} % ajusta o espaçamento dos parágrafos do resumo
\begin{resumo}
A expectativa de vida do brasileiro teve aumento de 30 anos entre 1940 e 2015 (IBGE), passando de 45,5 para 75,5 anos, além disso, a população passa por um processo de envelhecimento desde a década de 1960. Os idosos, por conta da fragilidade inerente à idade, necessitam de maiores cuidados na hospitalização e mais tempo de recuperação, o que aumenta o gasto com saúde por parte do Estado. 
Nesse sentido, a assistência domiciliar à saúde (ADS) - uma modalidade de atenção ao paciente realizado em seu próprio domicilio, que surgiu no Brasil na década de 1940 com o principal objetivo de liberar leitos nas unidades hospitalares - torna-se uma opção viável. O paciente conta com os seguintes atores para axiliá-lo: familiares, os amigos mais próximos, a equipe médica e o cuidador.
Porém, um dos problemas identificados nesse cenário de ADS é a necessidade de auxiliar o paciente e o cuidador, tendo em vista que, geralmente, este não possui os conhecimentos necessários para tornar o dia-a-dia de ambos mais agradável.
Além disso, como acompanhar de forma satisfatória o paciente em seu próprio lar? Infelizmente, o acompanhamento por parte da equipe médica não é constante, os familiares, muitas vezes, não podem ou não querem ajudar e, por fim, o paciente não conta com a figura do cuidador diariamente ou, em algumas situações, essa figura é inexistente. 
O avanço nas áreas de tecnologia da informação e sistemas embarcados nos proporciona dispositivos menores, com mais poder de processamento e com mais opções de conectividade permitindo inserí-los nos mais diversos ambientes, incluindo o ambiente domiciliar. Nesse sentido, um Set-Top Box foi utilizado como concentrador, recebendo toda a carga dos diversos sensores instalados no ambiente domiciliar.
Com essa tomada de decisão, identificamos um outro problema. De posse dos dados coletados, é necessário disponbilizá-los para o provedor de saúde (Estado ou empresas privadas). Além disso, os atores envolvidos (equipe médica, familiares, amigos e cuidador) precisam de informações em tempo real sobre determinado evento.
Portanto, para auxiliar o paciente e o cuidador no domicílio, mensagens e alertas são exibidos na televisão através do STB. Já para a transmissão dos dados do STB para o provedor de saúde ocorre de maneira padronizada com o uso do padrão OpenEHR. 
Como prova de contexto, parte dessa solução apresentada foi implementada.

 \textbf{Palavras-chave}: latex. abntex. editoração de texto.
\end{resumo}