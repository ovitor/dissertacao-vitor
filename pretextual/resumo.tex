\setlength{\absparsep}{18pt} % ajusta o espaçamento dos parágrafos do resumo
\begin{resumo}

A expectativa de vida do brasileiro teve aumento de 30 anos entre 1940 a 2015
(IBGE), passando de 45,5 para 75,5 anos. Para além desse aspecto, o processo de
envelhecimento da população, desde a década de 1960, foi acentuado de forma
crescente. Os idosos, por conta da fragilidade inerente à idade, podem
necessitar de maiores cuidados hospitalares, com maior tempo de recuperação, o
que aumenta o gasto com saúde por parte do Estado. Nesse sentido, a assistência
domiciliar à saúde (ADS) - uma modalidade de atenção ao paciente realizado em
seu próprio domicilio, que surgiu no Brasil na década de 1940 com o principal
objetivo de liberar leitos nas unidades hospitalares – apresentou-se como opção
viável para o atendimento ao paciente. À redução dos gastos hospitalares,
estudos revelam que o espaço doméstico se constitui como local mais adequado
para a recuperação do paciente, que pode contar com vários atores 
para auxiliá-lo:  familiares, amigos, a equipe médica e o cuidador. 
Porém, um dos problemas identificados nesse cenário de ADS é a necessidade de
auxiliar o paciente e o cuidador, tendo em vista que, geralmente, este não
possui os conhecimentos necessários para atender o doente. O acompanhamento por
parte da equipe médica também não é constante, e os familiares, muitas vezes,
não podem ou não querem ajudar e, por fim, o paciente nem sempre conta com a
figura do cuidador diariamente ou, em algumas situações, essa figura é
inexistente. Para enfrentar tais questões, a tecnologia de informação e sistemas
embarcados, mostram-se instrumentos eficazes para esse tipo de atendimento, ao
proporcionar dispositivos menores, com maior poder de processamento e com mais
opções de conectividade permitindo inseri-los nos mais diversos ambientes,
incluindo o ambiente domiciliar. Nesse sentido, um Set-Top Box foi utilizado
como concentrador, recebendo toda a carga dos diversos sensores instalados no
ambiente domiciliar. Com a aplicação deste dispositivo, identificamos um outro
problema. De posse dos dados coletados, percebeu-se a necessidade de
disponbiliza-los para o provedor de saúde (Estado ou empresas privadas). Além
disso, os atores envolvidos (equipe médica, familiares, amigos e cuidador)
precisam de informações em tempo real sobre determinado evento. Portanto, para
auxiliar o paciente e o cuidador no domicílio, mensagens e alertas são exibidos
na televisão através do STB. Já a transmissão dos dados do STB para o provedor
de saúde ocorre por meio do padrão OpenEHR. Como prova de conceito, parte desse
processo foi implementado.

 \textbf{Palavras-chave}: atenção domiciliar. computação ubíqua. computação pervasiva.
 \end{resumo}
