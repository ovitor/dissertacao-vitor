\setlength{\absparsep}{18pt} % ajusta o espaçamento dos parágrafos do resumo
\begin{resumo}

A expectativa de vida do brasileiro aumentou de 45,5 para 75,5 anos entre 1940
e 2015, segundo dados do IBGE. Idosos podem necessitar de maiores cuidados
hospitalares com maior tempo de recuperação, o que aumenta o gasto com saúde.
Nesse sentido, a Assistência Domiciliar à Saúde (ADS) - uma modalidade de
atenção ao paciente realizado em seu próprio domicílio, – apresenta-se como
opção mais adequada ao paciente. Além de reduzir gastos hospitalares a ADS
permite o contato mais intenso com familiares na recuperação do paciente.
Estudos realizados por profissionais de saude identificaram problemas no
cenario de ADS, tais como a dificuldade que possuem os diversos atores
envolvidos: o paciente, o cuidador, familiares, equipe de saude e gestores. O
cuidador mereceu especial atenção tendo em vista que ele nem sempre possui os
conhecimentos e as habilidades necessárias para atendendimento do doente. Este
trabalho descreve a visão computacional desses cenários de saúde e propõe a TV
Health das Coisas, uma solução inteligente para o ambiente de ADS baseada no
modelo brasileiro de TV digital e na tecnologia Internet das Coisas. Esta
solução tem como substrato um sistema embarcado associado à uma TV digital
(\textit{hub}) que coleta dados de diversos sensores existentes no ambiente de ADS, além
de servir como interface para o paciente a diversas aplicações interativas. Os
dados coletados alimentam modelos inteligentes de ontologia que permitem a
inferência local (\textit{hub}) de informações necessárias à tomada de decisão dos
atores do ADS: paciente, cuidador, equipe de saúde, familiares, etc. Outro
aspecto inteligente do TV Health das Coisas diz respeito a aprendizagem de
máquinas via o sistema \nextsaude, do qual a TV Health das Coisas é um
componente. Foi definida para este cenário de ADS uma arquitetura orientada a
contexto (\textit{context-aware concept}), enriquecida com conceitos da
plataforma \textit{OpenIoT} (Plataforma de Internet das Coisas).  Um protótipo
já se encontra operacional com suas arquiteturas funcional e de engenharia
formalmente especificadas.  Assim, paciente, cuidador, familiares, equipes de
saúde e gestores podem, de forma seletiva, receber alertas, informações etc, via
TV dispositivos móveis e outros mecanismos integrados ao sistema. A TV Health
das Coisas resultou em um projeto financiado pela Embrapii e está sendo
negociado para implantação em um plano de saúde.

 \textbf{Palavras-chave}: atenção domiciliar. internet das coisas. computação ubíqua. computação pervasiva.
 \end{resumo}
