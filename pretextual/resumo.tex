\setlength{\absparsep}{18pt} % ajusta o espaçamento dos parágrafos do resumo
\begin{resumo}

A expectativa de vida do brasileiro aumentou de 45,5 para 75,5 anos entre 1940 a
2015 (IBGE). Os idosos, por conta da fragilidade inerente à idade, podem
necessitar de maiores cuidados hospitalares, com maior tempo de recuperação, o
que aumenta o gasto com saúde por parte do Estado. Nesse sentido, a Assistência
Domiciliar à Saúde (ADS) - uma modalidade de atenção ao paciente realizado em
seu próprio domicílio, – apresenta-se como opção mais adequada ao paciente, além
da redução dos gastos hospitalares. Um dos problemas identificados nesse
cenário de ADS é a necessidade de auxiliar o paciente e o cuidador, tendo em
vista que, geralmente, o cuidador não possui os conhecimentos necessários para
atender o doente e o acompanhamento por parte da equipe médica também não é 
constante. Vale ressaltar que o paciente nem sempre conta com a figura do
cuidador diariamente ou, em algumas situações, essa figura é inexistente.
Naturalmente, a Tecnologia de Informação pode fornecer mecanismos e instrumentos
eficazes para esse tipo de atendimento. Considerando que o televisor é um
equipamento presente em praticamente todos as residências, este trabalho propõe
a TVHealthDasCoisas (THECA), uma solução inteligente para o ambiente de ADS
baseada no modelo brasileiro de TV digital e na tecnologia Internet das
Coisas. A THECA é um sistema embarcado em um ``\textit{Set-Top Box}'' (conversor
digital) que coleta dados de diversos sensores existentes no ambiente de ADS,
além de servir como interface a diversas aplicações interativas. Estes dados
coletados alimentam modelos de ontologia que permitem a inferência de
informações necessárias à tomada de decisão dos atores do ADS: paciente,
cuidador, equipe de saúde, familiares, etc. Assim, os diversos atores citados
podem receber alertas, informações etc, via TV ou dispositivos móveis integrados
ao sistema. A THECA é um componente do NextSAÚDE, uma plataforma de
interoperabilidade em saúde baseada no padrão OpenEHR. Ela faz uso de arquétipos
para integrar-se ao NextSAÚDE e, assim, compartilhar serviços disponíveis pela
plataforma (regulação, farmácia etc) e pelos outros diversos módulos existentes
no projeto (Dengosa, Vite, SisAPP etc). A THECA é uma arquitetura orientada a
contexto (\textit{context-aware concept}), enriquecida com conceitos da
plataforma \textit{OpenIoT} (Internet das Coisas). Um protótipo da THECA já se
encontra operacional e sua versão IoT tem suas arquiteturas funcional e de
engenharia formalmente especificadas.

 \textbf{Palavras-chave}: atenção domiciliar. computação ubíqua. computação pervasiva.
 \end{resumo}
