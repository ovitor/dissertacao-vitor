\setlength{\absparsep}{18pt} % ajusta o espaçamento dos parágrafos do resumo
\begin{resumo}

A expectativa de vida do brasileiro aumentou de 45,5 para 75,5 anos entre 1940
e 2015, segundo dados do IBGE. Idosos podem necessitar de maiores cuidados
hospitalares com maior tempo de recuperação, o que aumenta o gasto com saúde e
a necessidade de novas soluções.  Nesse sentido, a Assistência Domiciliar à
Saúde (ADS), uma modalidade de atenção ao paciente realizado em sua própria
residência, apresenta-se como uma opção mais adequada. Além de reduzir gastos
hospitalares a ADS permite o contato mais intenso do paciente com familiares em
sua recuperação.  No entanto, existem problemas no cenário de ADS. Os diversos
atores envolvidos (o cuidador, familiares, equipe de saúde e gestores), possuem
limitações ou dificuldade em prestar atendimento ao paciente. O cuidador, por
exemplo, nem sempre possui habilidades e/ou conhecimentos adequados. Este
trabalho descreve a visão computacional de cenários ADS e propõe a TV Health
das Coisas, uma solução inteligente e orientada a contexto para o ambiente de
ADS, baseada no modelo brasileiro de TV digital e na tecnologia Internet das
Coisas (IoT). Esta solução tem como substrato de \textit{hardware} um sistema embarcado
associado à uma TV digital (\textit{hub} de comunicação) que coleta dados de
diversos sensores existentes no ambiente de ADS. Estes são tratados por módulos
inteligentes no  \nextsaude, uma plataforma de tomada de decisão em saúde da
qual a TV Health das Coisas é um componente. Além disso, esta solução permite
que a TV sirva de interface para o paciente, e/ou cuidador, a alertas, a
recomendações e a outras aplicações interativas.  Assim, paciente, cuidador,
familiares, equipes de saúde e gestores podem, de forma inteligente, receber
alertas, informações etc, via TV, dispositivos móveis e outros mecanismos
integrados ao sistema.  A principal contribuição deste trabalho é a definição
de uma arquitetura para ambientes ADS orientada a contexto
(\textit{context-aware concept}), enriquecida com conceitos da plataforma
\textit{OpenIoT}, um \textit{Middleware} para Internet das Coisas. Foram
formalizados os aspectos funcionais e a visão de engenharia da arquitetura
proposta.  Um protótipo de \textit{hardware}, \textit{software} embarcado e
aplicações associadas foi implementado como prova de conceito da arquitetura da
TV Health das Coisas, tendo sido disponibilizadas APIs para agregação de novas
aplicações.  Há expectativa da TV Health das Coisas evoluir para um modelo de
referência no desenvolvimento de aplicações em ambientes ADS. Esta proposta
motivou um projeto aprovado pelo polo de Inovação Embrapii/IFCE e está sendo
cogitada sua implantação em um plano de saúde com abrangência nacional.

 \textbf{Palavras-chave}: atenção domiciliar. internet das coisas. context-aware.
 \end{resumo}
