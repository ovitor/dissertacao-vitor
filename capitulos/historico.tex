% \subsection{Histórico}\label{subsec:historico-tv-health}

% O Governo Brasileiro instituiu, através do decreto número 4901 de 26 de novembro
% de 2003, o Sistema Brasileiro de Televisão Digital (SBTVD) no  país. Além da
% melhoria técnica, com a troca do sinal analógico para o sinal  digital,
% promovendo, assim, melhoria na qualidade da imagem e som, a inserção  da TV
% Digital trouxe o conceito de interatividade, permitindo que o  telespectador
% saia da condição de espectador e passe a interagir com a  televisão. Além disso,
% através do decreto, o Governo tenta promover a inclusão social ao ``propiciar a
% criação de uma rede universal de educação à distância'' 
% \cite{digitaltv2015decree}.

% A proposta do SBTVD contém um forte viés social, uma vez que o Brasil conta com
% uma grande quantidade de lares com televisão que, adicionado à interatividade 
% possibilitada pela TV digital, o telespectador passa a receber um conjunto
% de novas funcionalidades através do \stb. Com isso, a população ganha novas 
% formas de acesso a serviços básicos.

% Desde então, houve estudos em diversas áreas para viabilizar tecnicamente o
% que o decreto propusera. As pesquisas apresentaram aplicações em áreas diversas
% como educação e serviços do Governo (receita federal, \textit{status} de benefícios
% etc) e área da saúde. Por conta dessa  abordagem, foi incluído no projeto
% \nextsaude[], um conjunto de aplicações voltada para o atendimento domiciliar à
% saúde que fizesse uso da TV Digital, e de sua interatividade característica.

% Adicionado a isso, surgiu a possibilidade de inserção das aplicações
% descritas  na subseção \ref{subsec:aplicacoes-tv-health} nos STBs
% disponibilizados aos beneficiários do programa Bolsa Família. Porém, o sinal
% analógico não foi completamente desligado, a adesão ao núcleo da tecnologia
% desenvolvida para a interatividade não foi completamente adotada  pela indústria
% e a distribuição dos STBs para beneficiários do Bolsa Família não levou em
% consideração as aplicações desenvolvidas, por questões financeiras.

% Esse histórico é um relato do que nos levou a acreditar na força do programa
% SBTVD, em virtude de seu viés social e inovador, trazendo consigo a
% interatividade. Entretanto, a recusa dessa tecnologia por parte da indústria e 
% sua indiferença no que concerne às importantes contribuições para a sociedade, 
% nos incentivou a pesquisar novas tendências capazes de suprir com outras
% soluções e estratégias tecnológicas o atendimento domiciliar à saúde.


