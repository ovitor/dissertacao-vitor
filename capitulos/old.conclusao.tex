\chapter{Conclusões e trabalhos futuros} \label{cap:conclusao}
% ---

Históricamente, a TV Health das Coisas nasceu do METAL (Mecanismo de Comunicação entre Concessionárias e Clientes baseada na TV Digital), um projeto de pesquisa e 
desenvolvimento no setor elétrico fomentado pela Agência Nacional de
Energia Elétrica (ANEEL) em parceria com a ENEL (antiga Companhia Energética do
Ceará - COELCE), tendo como base o Sistema Brasileiro da TV Digital (SBTVD). 

Em seguida, a ideia do projeto METAL evoluiu para o TV Health, um projeto que
conservava a mesma estrutura de \textit{hardware} do METAL, baseada na TV Digital,
mas tinha como foco a área de ADS. A principal motivação era, além da presença
quase universal da TV nas residências do país, a política federal vigente que prometia
dotar os STBs de aplicações interativas com interesses visivelmente sociais,
em especial nas áreas de educação e saúde. Na verdade, esta foi a razão do Brasil
ter decidido desenvolver um modelo próprio de TV Digital em 2003.

O TV Health foi desenvolvido no contexto do \nextsaude, um projeto financiado
pela Fundação Cearense de Apoio ao Desenvolvimento Científico e Tecnológico
no Ceará (FUNCAP). Até 2015, esperava-se que este módulo do \nextsaude
atendesse a demanda do Governo Federal que acabou sendo adiada e continua até
hoje indefinida: a expectativa do uso de aplicações interativas em toda a rede
de beneficiários do Bolsa Família. Nesse sentido, várias reuniões foram feitas
pelo autor deste trabalho e seu orientador na PUC Rio, mais precisamente no
laboratório Telemídia, autor do \middleware[] Ginga, um resultado do SBTVD que
tornou-se o padrão ITU-T número.

Diante deste impasse, relativo ao destino do SBTVD, deu-se início ao TV Health
das Coisas, novo projeto que, se de um lado mantinha a TV Digital e o seu associado
STB, por outro lado o foco da pesquisa passa a ser o ambiente de ADS. Assim,
toda a experiência acumulada pelo autor deste trabalho nos projetos METAL e
\nextsaude foi canalizada na concepção da TV Health das Coisas que, como o próprio
título caracteriza, faz uso da tecnologia \textit{Internet of Things} (IoT).

Foi, portanto, este o processo percorrido pelo TV Health das Coisas, uma plataforma
que disponibiliza uma arquitetura para o ambiente de ADS e implementa um protótipo 
como prova de conceito da proposta. Para tanto, buscou-se referência no OpenIoT,
uma plataforma da comunidade Europeia. Uma das razões dessa escolha deve-se a
uma série características conceituais do OpenIoT que vão ao encontro do Lariisa,
um projeto do qual se originou o \nextsaude e, por consequência, a TV Health
das Coisas. Por exemplo, tanto o OpenIoT quanto o Lariisa fazem uso de ontologias
e são fortemente orientados a contexto (\textit{context-aware concept}). 

Além da correlação arquitetural da TV Health das Coisas e o OpenIoT, a proposta
desse trabalho apresenta-se como uma plataforma mais leve. A disponibilidade de
APIs facilita o desenvolvimento de novas aplicações, e sua intengração a plataforma
\nextsaude[] agrega novas funcionalidades específicas da área de saúde. Na TV
Health das Coisas é viável o acesso a uma série de serviços de saúde, tais como 
o Registro Eletrônico de Saúde (RES), regulação (marcação de consultas, leitos etc.),
farmácia (acesso a remédios), entre outros.

Finalmente, caso haja uma decisão do Governo Federal em retomar a política em relação
ao programa Bolsa Família, a TV Health das Coisas poderá trazer benefícios a milhões 
de brasileiros, em especial das classes menos favorecidas, como preconizado no decreto 
4096 que instituiu o SBTVD. Neste ínterim, a TV Health das Coisas continua como
uma plataforma de pesquisa e desenvolvimento tendo sido aprovado pelo polo de Inovação Embrapii/IFCE e está sendo cogitada sua implantação em um plano de saúde com abragência nacional.


% Tomando como base a assertiva que um dos grandes meios de entretenimento e
% informação da população brasileira é a televisão e o advento do Sistema
% Brasileiro de Televisão Digital com a interatividade inerente a ele, foi
% pensado que a televisão pudesse ser utilizada também como um meio de interação
% com o usuário. Para fornecer mais funcionalidades foi necessário a aplicação do
% STB com poder de processamento e conexão à Internet disponibilizados na casa do
% usuário.   

% O projeto de pesquisa \nextsaude[] (número do projeto: 6424611/2014) composto
% por três módulos: (1) \textit{hardware}, (2) gerenciamento de ontologias e (3)
% aplicação,  se propôs a desenvolver soluções especializadas e gerar inovações
% tecnológicas de interoperabilidade sintática, integração semântica de dados,
% tomada de decisão automatizada e o consequente aconselhamento aos atores
% envolvidos (do usuário ao gestor) em sistemas público de saúde. Como prova de
% conceito, o projeto fez uso do STB e sensores em conjunto com a televisão digital
% para focar na  assistência domiciliar à saúde para doentes, idosos e cuidadores.

% O autor desse trabalho foi responsável pela execução do módulo de
% \textit{hardware} que era constituído por um sistema embarcado, além da aplicação de
% captura de dados do paciente em sua residência (assistência domiciliar à
% saúde), seja ela através de sensores ou por meio de aplicativos com entrada de
% dados. 

% Ao término do projeto \nextsaude[], o módulo de \textit{hardware} foi renomeado para
% TV-Health e era composto de um \stb[], aplicações interativas e sistema de alertas 
% e avisos. Os outros dois módulos receberam o nome de plataforma \nextsaude.

% Além disso, foram realizadas reuniões com o laboratório Telemídia da PUC - Rio
% no sentido de distribuir as aplicações interativas de saúde desenvolvidas para o TV
% Health nos STBs entregues aos beneficiários do programa Bolsa Família do Governo Federal.
% Essa ação colocaria as aplicações em uso para milhões de brasileiros,
% melhorando assim o acesso a informação e saúde.

% %Além disso, a coordenação do projeto disponibilizou a plataforma
% %\nextsaude[] e todos os seus subsistemas para que secretarias de saúde dos
% %municípios pudessem implantar o sistema em suas gestões.

% %Durante a execução do projeto, 

% Por fim, percebeu-se a necessidade de um ambiente de atenção domiciliar à saúde
% de qualidade, a importância do cuidador no auxílio ao idoso/doente além da
% qualidade de vida que é necessária para terceira idade ou doentes crônicos.
% Através das pesquisas, nota-se a importância da tecnologia da
% informação para prover uma solução mais impactante para o usuário. 

% A percepção de instrumentalização dos atores (cuidador, equipe de saúde, idoso
% ou doente) através da tecnologia também surgiu nesse período. Aplicações
% móveis, interface \web[] e acompanhamento do doente através da Internet foram
% tópicos importantes para o projeto.

% No período da entrega da primeira versão do TV Health, a partir da
% pesquisa do estado da arte de tecnologias emergentes, percebeu-se que a Internet das Coisas
% irá revolucionar a nossa sociedade. Diversos estudos apontam para uma transformação
% profunda nas áreas da indústria e de serviços.

% Não será diferente com o ambiente de atenção domiciliar como nós o conhecemos. Através
% de ``coisas'' conectadas dotadas de sensores, poderemos melhorar a experiência
% do usuário. Será possível, então, entregar um serviço bem mais transparente
% para o usuário final, alinhando-o com as novas soluções propostas ao
% redor do mundo.

% A arquitetura proposta para atenção domiciliar à saúde faz uso de tecnologias
% já conhecidas, tais como sistema sensível a contexto, sistemas embarcados e
% ontologias, incluindo-se também a Internet das Coisas. Auxiliar doente, cuidador,
% parentes e equipe de saúde através do STB e dispositivos móveis foi um dos
% objetivos deste trabalho. 

% A arquitetura considera a utilização do STB como um \textit{hub} de
% comunicação. Todas as ``coisas'', ou seja, sensores coletando sinais vitais do
% paciente, sensores coletando informações do ambiente e dispositivos móveis
% contribuindo para um sistema repleto de informações sobre o ator (idoso
% ou doente). Além disso a arquitetura contempla a tomada de decisão, tanto
% local, quanto na nuvem, através da utilização de ontologias e aprendizagem de
% máquinas.

% Com essas características o objeto de estudo foi aprovado como um
% projeto a ser executado em parceria com o Polo Emprapii de Inovação/IFCE e
% a empresa cearense EXATA. Uma alteração do objeto de estudo
% está em fase de negociação para que um plano de saúde privado faça uso 
% da TV Health das Coisas em unidades hospitalares.

\section{Produção científica}\label{sec:producao}  

Durante este projeto de mestrado, os seguintes trabalhos científicos foram
aceitos e publicados, a saber:

\begin{itemize}
  \item \textbf{LOPES, V. C. M.}; QUEIROZ, E.; FREITAS, N.; OLIVEIRA, M.; MONTEIRO, O. TV-Health:
  A Context-Aware health Care Application for Brazilian Digital TV. In ACM.
  \textit{Proceedings of the 22nd Brazilian Symposium on Multimedia and the Web (Webmedia)}. 
  Teresina, Brasil. 2016, pp. 103-106.

  \item \textbf{LOPES, V. C. M.}; ROCHA, E.; QUEIROZ, E.; FREITAS, N.; VIANA, D.; OLIVEIRA, M. VITESSE 
  - more intelligence with emerging technologies for health systems. In: IEEE. 
  \textit{2016 7th International Conference on the Network of the Future (NOF)}. Buzios, Brasil. 2016, pp. 1-3.
\end{itemize}

Além do seguinte artigo aceito para publicação:

\begin{itemize}
  \item \textbf{LOPES, V. C. M.}; MOTA, H.; OLIVEIRA, M.; CARVALHO, G. Towards an Emergency/Urgency
    approach based on the Brazilian Digital TV. \textit{Multi Conference on Computer Science 
    and Information Systems (MCCSIS)}. Ilha da Madeira, Portugal, 2016.
\end{itemize}

\section{Trabalhos futuros} \label{sec:trabalhos-futuros}

A TV Health das Coisas pode também ser visto como uma visão computacional do
trabalho realizado por \citeauthor{paz2014}, que apresentou a visão de saúde
dos cenários de ADS. 

Como trabalhos futuros, cogita-se a total integração dos serviços do \nextsaude
ao protótipo desenvolvido, aprimoramento de APIs e a inclusão de novos métodos de 
aquisição de dados.

Acredita-se que a arquitetura proposta na TV Health
das Coisas fornece uma base sólida e moderna para futuramente ajudar na 
concepção de um modelo de referência que facilite o desenvolvimento de aplicações
para ambientes específicos de ADS. 


% Uma vez definida a arquitetura e as características gerais que um sistema de
% atenção domiciliar à saúde atual deve ter, pode-se propor a inclusão de
% novas funcionalidades à arquitetura apresentada, tais como novos métodos
% de aquisição de dados com a ajuda de processamento de linguagem natural.

% A necessidade de gerenciar corretamente todos os dispositivos que a Internet
% das Coisas proporciona cria uma linha de pesquisa em que é necessário aprimorar
% a comunicação e gerência de IoT e Computação em Nuvem.

% Para finalizar, é possível realizar estudos relacionados ao tratamento em domicílio de
% doenças específicas - tais como Parkinson, pessoas com pressão alta,
% diabetes etc. O estudo em aprendizagem de máquinas permite a detecção
% antecipada do surgimento de uma doença específica.
