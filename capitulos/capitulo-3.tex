\chapter{Trabalhos relacionados}\label{cap:trabalhos-relacionados}

Neste capítulo é apresentado os trabalhos relacionados à pesquisa. Peculiaridades
de cada uma das soluções encontradas, seus pontos fortes e fracos de acordo com
a visão do autor. Na seção \ref{sec:comparacao} uma tabela comparativa é exposta.

\section{Estado da arte} \label{sec:estado-da-arte}

O trabalho de \citeonline{koch2006home} é uma leitura do estado atual e dos
tópicos mais estudados em relação a atenção domiciliar. Segundo o trabalho,
esta é uma das áreas que mais cresce. Com o intuito de diminuir custos,
permitir que o indivíduo tenha mais controle sobre seu estado de saúde e a
preferência de envelhecer em casa e próximo de seus familiares faz com que a
área esteja em alta. Alinhado a isso, o rápido crescimento da tecnologia da
informação e da comunicação auxilia a melhoria de serviços na área.

Palavras-chave como \textit{``home monitoring''}, \textit{``home
telemedicine''} , \textit{``information systems and home care''} estão no topo
de uma lista de termos buscados pela pesquisadora. Dentre os trabalhos
publicados, grande parte encontra-se nos Estados Unidos da América, Reino Unido
e Japão. Adicionado a isso, os estudos publicados têm como foco, pacientes com
doenças crônicas ou idosos.

É possível extrair do trabalho, ainda, que as linhas de pesquisas, a partir de
2000, envolve casas inteligentes e sensores vestíveis - principalmente para
detecção de quedas. Técnicas e soluções para vídeo consultas é um tópico
importante para a área. Outros tópicos importantes neglicenciados pelos
trabalhos são as padronizações para comunicar sistemas incompatíveis e a falta
de guias práticos para criação de casas com potencial atendimento em atenção à
saúde a distância. A área de segurança de dados também cresce com o intuito de
garantir a privacidade e confidencialidade dos dados previnindo questões legais
e éticas.

A revisão sistemática de \citeonline{pare2007systematic} sobre tele
monitoramento de pacientes com doenças crônicas apresenta como resultado
estudos que obtiveram sucesso em detecção antecipada de sintomas, periodicidade
nos medicamentos etc.  corroborando assim, com o estudo realizado por
\citeonline{koch2006home}.

\section{Outras propostas} \label{sec:solucoes-semelhantes}

O trabalho de \citeonline{isern2011agent} (TS-1) descreve uma arquitetura
descentralizada baseada em agentes para o \textit{home care}. A arquitetura é
dividida em três camadas: (1) Camada de conhecimento, representando um Registro
Eletrônico de Saúde\footnote{Documento estrutural responsável por 
manter um conjunto de informações de saúde e assistência de um paciente durante 
toda a sua vida.} (RES), ontologias e bancos de dados de atividades que podem
ser realizadas por outros atores descritos no sistema. (2) Camada de abstração
de dados é responsável por recuperar os dados de fontes e tecnologias
diferentes, como por exemplo os dados das ontologias, de RES e de base de dados
relacionais, esta camada é uma espécie de \textit{middleware}. (3) A camada
baseada em agentes contém a interface web que os atores (equipe de saúde e
pacientes) utilizam para interagir com o sistema. Além disso conta com os
agentes, que são entidade que agem de maneira semi automática no sentido de
realizar diversas tarefas do sistema. 

O trabalho, no entanto, não explica de que maneira será possível acompanhar o
doente no seu dia-a-dia, além de não aplicar ou permitir, na descrição da
arquitetura, que haja um acompanhamento constante.

A arquitetura funcional, descrita no trabalho de \citeonline{capozzi2010agent} (TS-2),
envolve a criação de duas áreas de interesse. Parte da arquitetura é chamada de
``Unidade de tratamento remoto'' (\textit{Replace Care Unit (RCU)}), responsável
pela interação do usuário, no caso o doente, com o sistema implantado no centro
de tratamento, nomeado de ``Unidade Central de Tratamento de Saúde''
(\textit{Health Care Center Unit - HCCU}). Nesta camada estão descritos os
sistemas utilizados pela equipe de saúde que acompanha o doente.

Além de usar dispositivos médicos para adquirir os sinais vitais, o trabalho
enfatiza, ainda, a solução criada para melhor acquisitar os dados médicos do
paciente. Foi utilizado reconhecimento de voz através de ligações para um
sistema implantado no HCCU. Na comunicação entre essas partes, trafegam, ainda,
alertas, lembretes, informações para alimentar o RES do paciente e informações
das atividades diárias do doente.

Do ponto de vista computacional, \citeonline{capozzi2010agent} optaram
por uma arquitetura baseada em agentes, um agente servidor, responsável por ser
um ponto de encontro dos agente móveis - (computadores de propósito geral ou
\smartphones[] para repassar os dados adquiridos pelos dispositivos). Há ainda,
nos agentes móveis, espaço para a utilização de domínios de conhecimentos
através de ontologias.
 
Apesar da inovação ao utilizar reconhecimento de voz para realizar a aquisição
de dados, o trabalho não faz nenhum estudo relacionado a usabilidade dessa
solução no cotidiano do doente. Além disso, não ficou claro se os dados
coletados trafegam dentro do sistema através de um protocolo conhecido e
aberto, permitindo que o sistema se comunique com outros sistemas de saúde.

O trabalho de \citeonline{bajo2010thomas} (TS-3) descreve a solução THOMAS
(\textit{MeTHods, Techniques and Tools for Open Multi-Agent Systems}), uma
arquitetura modular composta por serviços. Ela é dividida em três camadas,
Facilitador de Serviços (1), Sistema de Gerenciamento Organizacional e Núcleo
da Plataforma. O Facilitador de Serviços age como um \gateway[] para que outras
partes do sistema tenham acesso e para que duas entidades possam se comunicar,
permitindo a utilização de um serviço. O sistema de Gestão da Organização é
responsável pelo ciclo de vida de uma organização. Para o sistema, uma
organização pode ser o atendimento em domicílio e é estruturado por meio de
unidades, por exemplo, a unidade de alerta, unidade de tratamento, unidade de
localização  etc. No Núcleo da Plataforma é onde reside os serviços necessários
para uma arquitetura multi-agente, permitindo que vários agentes se comuniquem
entre si.

O trabalho de \citeonline{costa2012sensor} (TS-4) descreve uma plataforma de três
camadas, a camada de baixo nível de nome ``Aquisição de dados e
processamento'', realiza funções básicas de sistemas de monitoramento, coleta e
processamento de dados dos sensores. A camada intermediária chamada de
``Detecção de atividades'' tem o propósito de detectar atividades humanas a
partir dos dados coletados na primeira camada. A camada ``Agendamento e tomadas
de decisão'' provê agendamento de alertas, consultas etc.  além de servir de
interface para o restante do sistema.

A aquisição de dados dar-se prioritariamente a partir de câmeras, uma vez que
os autores do trabalho em questão já trabalharam com esse tipo de dispositivo.
Outra fonte de dados são os \smartphones, possibilitando por exemplo, se
determinado indivíduo sofreu uma queda. A camada intermediária utiliza
algoritmos de visão computacional para detectar atividades do dia-a-dia,
repassando estas informações para a terceira camada. O agendamento utiliza as
informações das outras camadas para realizar suas ações. 

Apesar do descrito, o trabalho não informa se há a utilização de outros
sensores no ambiente que não as câmeras. Também não deixa claro quais
técnicas são utilizadas para o agendamento e as tomadas de decisão.

O trabalho de \citeonline{palumbo2014sensor} (TS-5) mostra a infraestrutura de redes
de sensores necessários para atender um ambiente de atenção domiciliar.
Sensores médicos são utilizados no intuito de monitorar o paciente
continuamente. Com isso, o sistema pode emitir alarmes referentes a possíveis
problemas eminentes ou a longo prazo (de acordo com os dados coletados).
Adiciona-se a essa abordagem um \middleware[] com padrão OSGi\footnote{OSGI -
\textit{Open Services Gateway initiative}. Descreve um padrão aberto para
especificações que definem um sistema dinâmico de componentes para Plataforma
\textit{Java}.} e uma aplicação sensível a contexto.

Neste trabalho, além do \middleware, há uma infraestrutura de software composta
de serviços responsáveis por interligar várias funcionalidades, tais como o
serviço de armazenamento, responsável por persistir os dados remotamente e o
monitoramento inteligente que permite o acesso ao reconhecimento de atividades
e ao contexto.  

O sistema permite, ainda, uma visita remota a partir da utilização de um robô,
neste momento são discutidas as atividades do doente assim como os dados
adquiridos pelos sensores.

Apesar de não aparecer explicitamente os termo ``computação ubíqua'' ou ``Internet
das Coisas'', percebe-se a tentativa de utilizar técnicas e tecnologias referentes a
estes assuntos. O trabalha espera ainda, que o paciente esteja sempre próximo
de um \smartphone[] que é usado como \gateway[] para o restante do sistema na nuvem.
No entanto, não há uma abordagem na utilização de técnicas de inferência como
aprendizagem de máquinas ou ontologias.

\citeonline{spinsante2012remote} (TS-6) apresentam uma arquitetura sem fio centrada no
domicílio para monitorar a saúde do paciente, através dela, o usuário têm pouca
ou nenhuma interação com os dispositivos médicos, facilitando assim a
utilização do sistema pelos idosos. O trabalho utiliza um \framework[] OSGi para
lidar com os dispositivos médicos e uma aplicação interativa instalada no
\stb[] conectado a tela da TV, além disso, um serviço na nuvem recebe os
dados adquiridos pelos sensores. O \gateway[] é a figura central da arquitetura
proposta, ele é responsável por receber os dados dos sensores médicos através
da tecnologia \bluetooth[] e transmití-los para um repositório virtual na nuvem.
Já as aplicações interativas instaladas no STB tem acesso aos dados através de
uma API. A aplicação remota disponibiliza os dados através de uma interface
\web, que pode ser acessada pela equipe médica.

O trabalho não esclarece, no entanto, qual dispositivo desempenhará o papel do
\gateway[], podendo ser um \tablet[], um \smartphone[] ou um \textit{Personal
Digital Assistant} (PDA). Além disso, não utiliza nenhum tipo de inferência ou
inteligência no sistema para que se possa alertar o paciente ou a equipe de
saúde sobre uma possível anomalia.

\citeonline{niiranen2002personal} (TS-7) utiliza a melhoria tecnológica proveniente da
mudança da TV analógica para a TV digital para a criação da ``TV Digital para
cuidados médicos''. Focando na funcionalidade interativa da televisão digital,
o trabalho cria uma aplicação para \stb[] para que o usuário utilize através
do controle remoto, entrando com valores das medições realizadas por ele ou um
cuidador. Além de serem apresentados na tela da televisão, os dados são salvos
em um banco de dados para posterior processamento. Este trabalho foca na parte
técnica da televisão digital e na interação que ela proporciona, não
mencionando termos como sensores, informações de contexto ou inteligência
artificial.

O trabalho de \citeonline{chan2008mobile} (TS-8) aborda a criação de um ambiente de
monitoramento remoto baseado em sistemas de tecnologia da informação modernos.
Para isso, uma arquitetura de vários agentes para dispositivos móveis foi
desenvolvida, a arquitetura permite a coleta de dados e posterior recomendações
para pacientes e equipe de saúde. A solução é dividida em 3 áreas de
comunicação, (1) \textit{Body Area Network} (BAN), responsável pela comunicação
entre os sensores médicos com comunicação \bluetooth[] e o dispositivo móvel do
paciente.  (2) \textit{Personal Area Network} (PAN), responsável pela
comunicação do dispositivo móvel (\smartphone) com a rede de telefonia móvel
(3G/4G), um agente é responsável por enviar para o servidor na nuvem os dados
através de web services. (3) \textit{Wide Area Network} (WAN), permite à equipe
de saúde, acompanhar o estado do paciente através de um dispositivo móvel ou um
computador de propósito geral (interface \web).

Apesar da descrição com viés para a tecnologia de telecomunicação,
\citeonline{chan2008mobile} apresenta características que encontramos em
trabalhos descritos anteriormente, tais como sensores médicos, monitoramento
remoto e acompanhamento por parte da equipe de saúde. Tópicos como sistemas
inteligentes, aplicações sensíveis a contexto e TV digital, no entanto, não
entram no escopo do trabalho.

\section{Comparação entre soluções} \label{sec:comparacao}

A tabela \ref{tab:trabalhos-semelhantes} a seguir apresenta um comparativo
simplificado do que foi exposto na seção anterior. O termo \texttt{TS} refere-se
a ``trabalhos semelhantes'' enquanto que o termo \texttt{SP} refere-se a
``solução proposta''.

\tabela{trabalhos-semelhantes}{Resumo das características das soluções semelhantes apresentadas
anteriormente}{0.9}

