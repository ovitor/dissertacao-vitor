\chapter{Fundamentação Teórica}\label{cap:fundamentacao-teorica}

Este capítulo apresenta a fundamentação teórica, separada em aspectos de saúde e
tecnológicos. As seções de saúde abordam termos como Assistência Domiciliar à
Saúde e suas particularidades, além de explanar, brevemente, sobre a história da
hospitalização no Brasil. A seção \ref{sec:aspectos-tecnologicos} trata das
tecnologias e conceitos tecnológicos utilizados neste trabalho.

\section{Aspectos de Saúde}\label{sec:aspectos-de-saude}

Segundo \citeonline{inaia2008}, uma das primeiras instituições voltadas para o
cuidado com a saúde foi a fundação da Santa Casa de Misericórdia de Santos, em
1543, cuja principal atividade era prestar assistência de cunho caritativo a
pessoas pobres e desabrigados. Tal estrutura permanece inalterada até o final do
século XIX, e início do século XX.

Com o Governo Getúlio Vargas, a partir de 1930, ocorre no Brasil o processo de
industrialização, que trouxe crescimento rápido e desordenado às cidades,
principalmente São Paulo e Rio de Janeiro. As transformações econômicas e
sociais resultantes desse processo, a falta de saneamento básico, a pobreza etc
foram motivos para que parte da população reivindicasse mais atenção do Governo
em relação aos cuidados de saúde \cite{carvalho1984}.

\citeonline{carvalho1984} indica, ainda, que apesar dessa pressão, não existiu
uma política de saúde clara por parte das autoridades. Muitas vezes, algumas
ações se voltavam para a criação de condições sanitárias mínimas, que se
mostravam limitadas frente às reais necessidades da população. Dessa forma, as
décadas subsequentes não foram significativas no tocante a uma ampliação dos
serviços de saúde oferecidos à população.

Conforme análise de \citeonline{inaia2008}, ainda nos anos 1970, surgem
tentativas de universalizar o acesso à assistência à saúde. Alguns programas e
sistemas foram iniciados, sendo válido citar, (i) Sistema Único e
Descentralizado de Saúde, o SUDS e (ii) o Sistema Único de Saúde, o SUS.
Entretanto, em virtude da vigência da ditadura militar, implantada em 1964, que
levou o país a vivenciar um estado de exceção, tais propostas não conseguiram se
concretizar. Assim, somente no período da redemocratização, ocorrida em meados
dos anos 1980, é que o SUS foi criado oficialmente pela Constituição Federal de
1988, Lei 8080/90\footnote{ Acessível em
\url{http://www.planalto.gov.br/ccivil_03/leis/L8080.htm}}, com o  objetivo de
garantir à população Brasileira, o acesso universal às ações e  serviços de
saúde.

Paralelo à criação do Sistema Único de Saúde (SUS), ocorreu o avanço
tecnológico que alcançou a prática médica, aperfeiçoando, com isso a
infraestrutura hospitalar. Dessa forma, os hospitais deixaram de ser espaços
para abrigarem pobres desamparados e passaram a proporcionar tratamentos mais
elaborados. O hospital passa a oferecer  procedimentos cirúrgicos, atendimentos
de urgência, internações, tornando a instituição complexa, com uma atuação de
caráter menos íntimo e acolhedor. Além disso, estudiosos começam a identificar
a possibilidade de tratamentos e cuidados com a  saúde que não estejam,
necessariamente, vinculados ao ambiente hospitalar.

Como consequência, surgiram diversas mudanças no atendimento, onde a Assistência
Domiciliar à Saúde (ADS) se tornou uma modalidade disponível.

\subsection{Assistência Domiciliar à Saúde}
\label{subsec:assistencia-domiciliar-a-saude}

A Assistência Domiciliar à Saúde (ADS) divide-se basicamente em grupos de
enfermagem e fisioterapia - nas modalidades mais básicas - até um atendimento
multiprofissional, possibilitando um apoio ao paciente como um todo. A ADS pode
ser provida tanto pelo setor privado quanto pelo setor público
\cite{amaral2001assistencia}.

Os primeiros registros da ADS no Brasil surgem em 1967, na cidade de São
Paulo, no Hospital do Servidor Público. O principal objetivo dessa abordagem
era a liberação de leitos no hospital, levando para o domicílio procedimentos
básicos, de baixa complexidade clínica.

No começo da década de 90, segundo \citeonline{tavolari2000desenvolvimento},
houve um aumento na quantidade de empresas privadas provendo o serviço de ADS,
com atuação de cinco empresas que prestavam esse tipo de  serviço. Já em 1999,
esse número subiu, consideravelmente; para mais de 180
\cite{tavolari2000desenvolvimento}.

\citeonline{amaral2001assistencia} define a ADS como uma sequência de serviços
residuais a serem oferecidos, depois que o indivíduo já recebeu atendimento
primário e prévios, ou seja, aquele que já recebeu atendimento primário com
consequente diagnóstico e tratamento.

\citeonline{amaral2001assistencia} lembram, ainda, que o atendimento domiciliar
pode acelerar a recuperação do  paciente e promover a redução de custos
hospitalares, além de ser uma solução mais  humanista para os portadores de
doenças crônicas ou de longa duração, frente à  hospitalização. Dessa forma, a
assistência domiciliar à saúde tem como objetivos principais: (1) humanização
no atendimento; (2) maior rapidez na recuperação do paciente, devido à
proximidade com os seus familiares; (3) diminuição do risco de infecção
hospitalar; (4) Otimização de leitos hospitalares para pacientes que deles
necessitem; e (5) Redução do custo/dia da internação.

\subsubsection{Envolvidos}\label{subsubsec:envolvidos}

Para o entendimento geral da modalidade ADS, faz-se necessário separar e
explicar a atuação de cada um dos envolvidos. O paciente, para quem está
voltado, fundamentalmente, todo o sistema, é aquele que sofre algum problema
físico ou mental. A família é responsável por prover um ambiente propício à
melhora do paciente.

Outra figura importante na atenção ao paciente é a equipe multiprofissional -
composta de médicos, enfermeiros, psicólogos, terapeutas, assistentes sociais,
farmacêuticos, cuidadores e outros - visando propiciar, através da integração
das diversas áreas de conhecimentos, a melhoria efetiva do paciente.

O cuidador, muitas vezes, é um familiar, alguém próximo à família ou alguém
contratado. Seu papel principal é cuidar do paciente, ajudando nas tarefas
diárias, como alimentação, lazer, socialização, limpeza do paciente, entre
outros \cite{amaral2001assistencia}.

%Embora as pesquisas indiquem que o tratamento domiciliar ajuda na recuperação do
%paciente, devido sua presença no seio familiar, também sabemos das dificuldades
%que os cuidadores enfrentam. exemplo quando são familiares, que muitas vezes, 
%negam a dar atenção, ou desempenhar determinadas funções.

\subsubsection{Terminologia}\label{subsubsec:terminologia}

Apesar de não haver uma definição formal, a Assistência Domiciliar à Saúde pode
ser separada em três modalidades, diferenciadas, principalmente, pelo grau de 
atenção dispensada ao paciente. 

É defendido por Tavolari, Fernandes e Medina, que o termo Assistência Domiciliar
à Saúde é genérico e referente a todo e qualquer procedimento de saúde realizado
em domicílio, não importando o grau de complexidade. Já o termo Internação
Domiciliar é aplicado quando, dos procedimentos realizados, o cuidado intensivo
e multiprofissional é perceptível, caracterizando-se ainda, pelo transporte de
parte da estrutura hospitalar para o domicílio do paciente. O paciente, nesse
caso, é categorizado com complexidade alta ou moderada.
Já no Atendimento Domiciliar, o paciente encontra-se num estado de menor 
complexidade médica, e a atenção a ele dispensada pode ou não ser realizada por
uma equipe multiprofissional \cite{tavolari2000desenvolvimento}.

\figuradupla{ads-tavalori}{Representação gráfica das categorias defendida 
por Tavolari, Fernandes e Medina}{ads-giacomozzi}{Representação gráfica das 
categorias defendida por Giacomozzi}

\citeonline{giacomozzi2006pratica} faz uma inversão das categorias propostas
por \citeonline{tavolari2000desenvolvimento} e  define Atenção Domiciliar à
Saúde como um termo mais genérico, englobando o atendimento, a visita e as
internações domiciliares \cite{giacomozzi2006pratica}. Segundo a autora, a
atenção domiciliar é ``um componente do \textit{continuum} dos cuidados à
saúde, pois os serviços de saúde são oferecidos ao indivíduos e sua família
[...] minimizando os efeitos das incapacidades ou doenças, incluindo aquelas
sem perspectiva de cura.'' Já o termo Assistência Domiciliar à Saúde, é formado
por atividades de cunho ambulatorial, adicionando a essa categorização a
modalidade Visita Domiciliar, voltada para verificar a realidade do paciente,
além de realizar ações educativas.

A despeito da diversidade de categorização apresentada pelos autores, ao nosso
trabalho interessa tanto aqueles pacientes que inspiram maiores cuidados,
correndo, inclusive, risco de vida, quanto aqueles que demandam menos
preocupações - mas que, invariavelmente, estão sob os cuidados fora do
ambiente hospitalar.

\section{Aspectos Tecnológicos}\label{sec:aspectos-tecnologicos}

Este trabalho foi amparado em tecnologias bem conhecidas, permitindo assim, 
que os objetivos fossem alcançados.

\subsection{Sistemas Embarcados}\label{subsec:sistemas-embarcados}

Sistemas Embarcados estão inseridos em nossos cotidianos. Aparelhos como
\smartphones[], \tablets[] - com alto poder de processamento,
produtos comumente encontrados em nossas casas, a exemplo do forno de 
micro-ondas, geladeiras e máquinas de lavar roupas, até computadores de bordo e
controle de freios ABS (\textit{Anti-lock Breaking System})\footnote{ABS:
Tecnologia  de freio considerada segura pois seu funcionamento impede que, em
uma brecada brusca, as rodas  deslizem, tornando difícil controlar o veículo} em
nossos carros, contêm sistemas embarcados. Essa diversidade de aplicações
explicita a importância dessa área.

Nos exemplos citados anteriormente, temos uma classificação quanto à sua
funcionalidade, porém, sua definição não é simples, nem muito menos taxativa,
uma vez que possuem uma grande complexidade em sua composição. Dessa forma,
podemos definir, a princípio, sistemas embarcados como qualquer
dispositivo/equipamento que disponha de um sistema programável e que seu
objetivo não seja o de um computador de propósito geral\footnote{Computador de
propósito geral é aquele em que é possível executar as mais diversas tarefas,
tais como, navegar na Internet, escrever documentos, executar jogos etc, ou
seja, não há uma função específica a ser realizada. Pelo exposto, percebe-se a
complexidade da definição de sistemas embarcados, uma vez que, muitos deles, 
desempenham funções que antes cabiam apenas ao computador de propósito geral.}
\cite{wolf2012computers}.

Marwedel define: ``Sistemas Embarcados (como) sistemas de processamento de
informações incorporados à produtos.'', e, adicionado à esta definição, lista 5
características que devem levar em consideração: 

\begin{itemize}
  \item Segurança - ou seja, os dados confidenciais transmitidos ou recebidos pelo
  equipamento devem permanecer confidenciais e a comunicação deve ser autêntica;
  \item Seguro - característica indicativa de que o sistema não causará nenhum dano àqueles que o utilizam;
  \item Disponibilidade - sistema disponível para executar as funções para as quais
  foi programado; 
  \item Confiabilidade - probabilidade que o sistema tem de que não
  falhará em sua execução; e
  \item Manutenibilidade, caso o sistema venha a falhar, deverá 
  ser consertado em uma determinada janela de tempo.
\end{itemize}

Além disso, propriedades como sensores coletando informações do ambiente
físico e atuadores controlando o ambiente no qual estão inseridos também
caracterizam um sistema embarcado \cite{marwedel2010embedded}.

\subsubsection{Arquitetura}\label{subsubsec:arquitetura}

A literatura divide o sistema embarcado em duas áreas, o \hardware[] e o 
\software. O primeiro é composto pelas partes físicas do sistema, 
tal como o processador, memória, interfaces de entrada e saída etc. O segundo é
composto pelos componentes lógicos do sistema, ou seja, os programas que 
irão executar as funções previamente definidas do sistema embarcado.

No processo de \design[] do sistema embarcado considera-se, dentro dos
requisitos de \hardware, quais dos processadores disponíveis no mercado atende
melhor a necessidade do projeto, assim como é necessário identificar a
quantidade de memória a ser utilizada pelo \software[]. 

Ainda referente ao processo de design é necessário determinar quais serão as
interfaces de entrada (responsáveis por receber os dados para posterior
processamento) e as interfaces de saída (responsáveis por apresentar os dados
processados). Em paralelo, os requisitos de \software[] são alinhados para que o
\hardware[] seja melhor aproveitado \cite{wolf2012computers}.

A depender da finalidade do produto, esses requisitos diferem bastante. Por
exemplo, o sistema embarcado responsável por controlar uma máquina de lavar é
simples (em termos de funcionalidade, processamento, consumo de energia etc),
uma vez que aplica um componente microcontrolador de baixo poder de
processamento (tal qual um microprocessador PIC de 16 bits) e um circuito
auxiliar para atender as especificações de entrada e saída do sistema, assim
como uma possível comunicação com outros sistemas.

Já o sistema embarcado em um \smartphone[] realiza diversas funções, tem um
alto nível de processamento de dados e consome muita energia. Para atender a
estes requisitos, a equipe responsável pelo \design[] utiliza vários
processadores de alto poder computacional (tal qual processadores ARM), como
interface de entrada e saída uma tela sensível ao toque e faz uso, ainda, de
diversos sensores para ajudar na usabilidade do dispositivo.

Semelhante aos \smartphones, o sistema embarcado utilizado neste trabalho, o
Set-Top Box (STB), processa grande quantidade de dados, além de realizar
diferentes funções. Em que pese tais diferenças, o \hardware[] utilizado é
bastante semelhante.

\subsection{Computação ubíqua e pervasiva}\label{subsec:computacao}

A convergência das diversas tecnologias pode ser observado no nosso dia-a-dia
através de: informações constantes que recebemos nos dispositivos móveis, nos
meios de comunicação, do monitoramento em tempo real do tráfego nas cidades de
grande porte, dos relógios com tecnologia embarcada (\textit{smartwatches}), das
vestimentas inteligentes etc.

Percebemos, também, o distanciamento ou o desaparecimento da figura ``computador
pessoal (PC)'', em nosso cotidiano. A tecnologia está difundida ao ponto do
termo  ``era pós pc'' ser encontrado em diversos estudos
\cite{bonilla2011inclusao,  chen2011pospc, press1999personal} nas áreas de
sistemas embarcados e  tecnologias móveis.

À essa tecnologia ``em todo lugar'' e ao distanciamento do usuário com o 
computador pessoal deu-se o nome de computação ubíqua. Mark Weiser, considerado 
o pai desse termo, antecipou que encontraríamos tecnologia nos diversos objetos
do nosso cotidiano, tais como etiquetas de roupas e alimentos, cadeiras, 
geladeiras, lixeiras, interruptores de luzes etc \cite{weiser1991computer}. Essa 
previsão de Weiser já se concretizou em vários dispositivos.

Através de sensores e meios de comunicação, esses objetos ganham novas funções.
A patente de Yang registra uma lixeira inteligente que abre sua tampa de acordo
com a proximidade do usuário \cite{yang2005trash}. Já a pesquisa de Wang
demonstra a utilização de sensores em uma geladeira. A partir de um sistema de
gerenciamento da geladeira é possível ter informações das comidas e recomendar
receitas para o usuário de acordo com os alimentos disponíveis, além de 
verificar quais itens estão próximos do final da validade ou faltantes e gerar 
uma lista de compras personalizada \cite{hou2013}.

% Os dispositivos \smartphones, \tablets, relógios inteligentes e até mesmo roupas
% inteligentes são realidades para muitos. Esses dispositivos permitem que o 
% indivíduo que os carrega, tenha um poder computacional que o permita utilizar 
% serviços que um computador oferece, independentemente da sua localização 
% \cite{de2003computaccao}.

Muitas vezes, a tecnologia inserida nesses objetos é invisível para o usuário
final e essa transparência era defendida por Weiser para apresentar um outro
termo, a ``computação pervasiva'', assim definida pelo autor: ``(...) criar  um
computador tão embarcado, tão natural que o usaríamos sem nem pensar sobre
ele''. Além disso, computação pervasiva está relacionada à capacidade de
dispositivos serem embarcados no mundo físico, obtendo informações do meio para
auxiliar na computação, integrando, dessa forma, o ambiente físico e o mundo
virtual \cite{bolsoni2009computaccao, de2003computaccao}. A figura  
\ref{fig:pervasive-marwedel} demonstra a interligação entre as áreas de sistemas
embarcados, computação ubíqua e pervasiva com as tecnologias de comunicação.

\figurasimples{pervasive-marwedel}{Representação gráfica das influências nas 
áreas de sistemas embarcados, computação ubíqua/pervasiva e meios de 
comunicação.}{12cm}

Segundo Hansmann, a computação pervasiva tem 4 princípios fundamentais,
detalhados a seguir \cite{hansmann2013pervasive}:

\begin{description}
  \item [Descentralização] os diversos dispositivos cooperam entre si e realizam
  pequenas tarefas e funções, contribuindo para o estabelecimento de uma 
  dinâmica rede de comunicação; Podemos citar como exemplo os diversos 
  dispositivos implantados na casa proporcionando conectividade em toda
  a extensão da casa.
  \item [Diversificação] diferente do computador de propósito geral - que 
  proporciona a execução de várias tarefas - cada dispositivo tem um propósito
  específico atendendo a necessidades únicas; podemos citar, como exemplo, um
  sensor para leitura do nível de oxigênio no sangue, um relógio inteligente
  que conta quantos passos uma pessoa deu durante o dia etc; dentro dessa 
  diversidade, cada um desempenha uma função específica.
  \item [Conectividade] os dispositivos devem interagir de maneira transparente
  e aqueles que são móveis, devem mudar entre redes heterogêneas, sem o 
  auxílio de um usuário; a conectividade diz respeito, também, à relação 
  transparente entre dispositivo e usuário; 
  \item [Simplicidade] as funções desempenhadas pelos dispositivos devem ser
  simples e de fácil execução. Existem tipos de dispositivos que não requerem 
  uma interação com o usuário; outros, no entanto, exigem alguma configuração e,
  nestes casos, a interação deve manter-se simples, capaz de oferecer acesso
  rápido ao usuário. 
\end{description}

No que concerne à assistência domiciliar à saúde, as pesquisas nas áreas de
computação ubíqua e pervasiva são as mais diversas. Isso porque o domicílio é um
ambiente propício à aplicação dessas técnicas. A utilização de sensores, como
demonstrado no estudo de Warren, em que sensores  vestíveis são empregados na
leitura de oxigênio do paciente e o envio dos dados à um sistema de
monitoramento pessoal para posterior processamento \cite{warren2002sensors}
exemplifica a utilização da computação pervasiva, uma  vez que os dados
coletados se tornam disponíveis a qualquer momento.

\subsubsection{Internet das Coisas} \label{subsubsec:iot}

No final dos anos 90, \citeonline{ashton2009internet} defendeu que a Internet
das Coisas tinha potencial para mudar o mundo, assim como a Internet mudou.
Desde então pesquisas em diversas áreas, a inovação tecnológica e a atualização
de conceitos já existentes possibilitou o surgimento da tecnologia Internet das
Coisas (\textit{Internet of Things - IoT}).

Na revisão sistemática sobre IoT, \citeonline{li2015internet}, defende IoT como
parte da Internet do futuro e consiste de bilhões de ``coisas'' inteligentes
que se comunicam entre si. Já \citeonline{pretz2013next} define IoT como um
conjunto de coisas conectadas por uma rede sem fio e que são capazes de
interagir entre si, sem interferência humana. Já
\citeonline{guillemin2009internet} define Internet of Things permite que
pessoas e coisas estejam conectadas a qualquer hora, em qualquer lugar com
qualquer coisa e qualquer um, preferencialmente usando uma rede e qualquer
serviço.

\figurasimples[perera2014context]{iot-perera}{Ilustração da definição de IoT.}{8cm}

%Apesar da capacidade de trocar dados entre si, alguns autores como, Dada e
%Thiesse, Broll e Gama et al argumentam que a falta de padrões pode diminuir a
%competitividade de produtos IoT. Por esse motivo, esses mesmos autores
%trabalham em comissões internacionais para padronizar protocolos de
%comunicação.

IoT habilita sistemas a capturar, armazenar e  transmitir dados de modo que
cresce a quantidade de áreas atendidas por esse tipo de tecnologia.
Alimentação, segurança, indústrias, logística, turismo e armazenamento são
algumas das áreas em que a aplicação de IoT é abrangente e traz benefícios
\cite{xu2014ubiquitous}.

A área de atendimento domiciliar é uma importante área de aplicação de IoT.
Segundo \citeonline{xu2014ubiquitous}, IoT é adotado para melhorar a qualidade
do serviço e reduzir custos. Um número cada vez maior de sensores médicos e
dispositivos capazes de capturar sinais vitais do paciente  - como por exemplo
temperatura corporal, nível de glicose no sangue, e pressão sanguínea - estão
surgindo, possibilitando um monitoramento do paciente em tempo real.

\subsection{Aplicações Sensíveis ao Contexto}\label{subsec:contexto}

O dicionário Houaiss define contexto como um conjunto de palavras, frases ou
texto que precede ou se segue a determinada palavra, frase ou texto e que
contribuem para o seu significado. Podemos entender desta definição que contexto
são circunstâncias que acompanham a situação ou determinado fato. A depender do
contexto, nos portamos de maneira diferente, como por exemplo: nos portamos de
uma maneira no ambiente de trabalho (contexto) e de uma maneira diferente quando
estamos na praia (contexto).

% Na comunicação entre pessoas, aproveitamos o contexto para facilitar, ou ainda,
% acelerar a velocidade da conversa, mas não é fácil a utilização de contexto nas
% comunicações homem-máquina.

Apesar de exemplificar de maneira clara o que é contexto, esse tipo de definição
apresentada anteriormente é de difícil aplicação na área da computação.
Outras abordagens com definições mais específicas foram surgindo com o
passar do tempo. No trabalho de Schilit et al são definidos 3 aspectos
importantes para o contexto, (1) onde você está, (2) com quem você está e (3)
quais recursos estão próximos \cite{schilit1994context}. Pascoe define contexto 
como um conjunto de estados de interesse conceituais e físicos para uma entidade 
em particular \cite{pascoe1998adding}.

O desenvolvedor precisa identificar se determinada informação é contexto ou não
para determinada aplicação em determinado momento. Tomando como exemplo uma
aplicação móvel de acompanhamento de um \textit{tour} em um museu a céu aberto, 
tanto as informações de tempo (clima, precipitação, umidade etc) quanto as 
informações de pessoas (quantidades em cada setor) são informações de contexto, 
já em locais fechados, a informação de tempo não representa uma informação de 
contexto.

Davies realizou um trabalho semelhante ao apresentado acima. Foi 
desenvolvido uma aplicação de guia turístico sensível a contexto para a cidade 
de \textit{Lancaster} na Inglaterra. A solução combinava computação móvel, 
comunicação sem fio e sistemas embarcados para prover aplicações interativas
\cite{davies1999caches}.

No sentido de simplificar o desenvolvimento de aplicações sensíveis ao contexto,
Dey define "contexto (como) qualquer informação que pode ser utilizada para
caracterizar a situação de uma entidade. Uma entidade pode ser uma pessoa, um
lugar ou um objeto que é considerado relevante para a interação entre um usuário
e uma aplicação, incluindo o usuário e a aplicação".

É possível, portanto, definir que um sistema computacional é sensível ao contexto
se ele fizer uso do contexto para prover informações ou serviços relevantes para
o usuário de acordo com a sua tarefa no momento \cite{dey2001understanding}.

No que concerne a assistência domiciliar à saúde, os sistemas sensíveis a
contexto podem fazer uso de informações de sensores médicos (leitura de sinais 
vitais do paciente), sensores de propósito geral (leituras no ambiente), 
ou ainda sensores virtuais (conjunto de programas cuja função é varrer as redes 
sociais e extrair informações válidas para determinada situação).

\subsection{Plataforma OpenIoT}\label{subsec:openiot}

O OpenIoT é um esforço conjunto da União Européia para Pesquisa e Inovação (FP7
- \textit{European Union's Research and Innovation}). A plataforma 
\textit{OpenIoT} tem como objetivo principal a implantação de uma infraestrutura 
de \textit{middleware} capaz de integrar soluções de Internet das Coisas.

O projeto enfatiza a convergência de Internet das Coisas e computação em nuvem,
habilitando esses dois tópicos através de interoperabilidade semântica e dados
ligados (\textit{Linked Data}), desse modo, será possível fornecer aos
interessados uma "nuvem de coisas". A figura \ref{fig:arquitetura-openiot} a 
seguir representa  as partes que compõem a plataforma OpenIoT.

\figurasimples[soldatos2015openiot]{arquitetura-openiot}{Arquitetura plataforma 
\textit{OpenIoT}}{12cm}
    
A plataforma foi arquitetada com 3 grandes seções: (1) Aplicações e Utilitários,
(2) Plano virtualizado e (3) Plano físico. Diversos componentes compõem as 
seções, listados abaixo, com uma visão \textit{top-down}:

\begin{description}
  \item [Aplicações e Utilitários]
  \item [Request Presentation] Responsável pelas apresentações das saídas dos serviços,
ou seja, permite a seleção de mashups a partir de uma biblioteca com o intuito
de facilitar a apresentação dos objetos configurados em uma interface web. Para
isso, esse módulo comunica-se diretamente com o módulo Service Delivery and
Utility Manager).

  \item [Request Definition] Conjunto de serviços responsáveis por especificar e
formular requisições - em tempo de execução - para a plataforma OpenIoT.

  \item [Configuration and Monitoring] Habilita uma visão de gerenciamento,
monitoramento e configuração das funcionalidades dos sensores e serviços
acoplados à plataforma OpenIoT.
\end{description}


\begin{description}   
  \item [Plano Virtualizado]   
  \item [Scheduler] Processa as solicitações de serviços a partir do módulo 
  Request Definition. Além disso, tem como outro objetivo encontrar sensores 
  para serem adicionados à plataforma OpenIoT.

  \item [Linked Stream Middleware] Equivalente a base de dados na nuvem. Persiste os
dados provenientes da camada de sensores (reais e virtuais). Nesse módulo também
são guardados as configurações e informações gerais referente ao funcionamento
da rede OpenIot.

  \item [Service Delivery and Utility Manager] Recebe consultas (geralmente SPARQL)
para fornecer streams de dados do serviço requisitado. Além disso, realiza uma
medição de utilização dos serviços fornecidos pela plataforma.
\end{description}

\begin{description}
  \item [Plano Físico]
  \item [Sensor Middleware] extende as funções do Global Sensor Network (GSN), se
tornando o X-GSN. É responsável pela aquisição de dados de sensores reais e/ou
virtuais. Ele atua como um gateway entre o mundo físico e a plataforma OpenIoT.
Tem sua base de funcionamento em sistemas distribuídos - várias instâncias do
sistema que podem ou não pertencer a entidades diferentes.
\end{description}























