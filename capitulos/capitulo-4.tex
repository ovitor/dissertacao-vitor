\chapter{Proposta}\label{cap:proposta}

Neste capítulo serão abordadas as etapas que constituem a estrutura geral do
sistema de informática voltado para o atendimento de paciente no espaço domiciliar. 
Sua descrição é realizada na primeira seção. O TV-Health, parte integrante da 
solução, tem suas características apresentadas no item \ref{sec:tv-health}.

\section{Descrição da solução}
\label{sec:descricao-solucao}

A arquitetura proposta soluciona o problema relatado na seção  
\ref{sec:descricao-problema}. Ela tem como ponto principal a utilização de 
um \stb[], como pode ser visto na figura tal, quando se conectam a ele sensores 
implantados no ambiente, sensores de dispositivos como \smartphones[] e 
\smartwatches[] do paciente e do cuidador, e sensores médicos do doente.

Esse processo é composto de três momentos distintos, mas 
interligados, no qual o primeiro é relativo à aquisição de dados do ambiente
e do paciente, e a transmissão desses dados ao stb. O segundo diz respeito
ao processamento dos dados brutos. O terceiro momento o envio dos dados
processados para a plataforma nextsaúde.

Neste sentido, os sensores gerais, relativos ao ambiente domiciliar, coletam informações como 
temperatura e umidade da residência, qualidade do ar e localização do paciente
ou cuidador nos cômodos. A partir desses dados é possível dizer ao software se
determinado espaço está muito quente, se a qualidade do ar não está favorável
à melhoria do paciente etc. Dados extraídos de celulares, como a localização,
também cooperam com o contexto, a exemplo da informação proveniente
do celular do cuidador que informa, à equipe médica ou aos parentes,
se este está junto ao paciente.  

Já os sensores médicos têm como função principal coletar os sinais vitais do
paciente (pressão arterial, frequência cardíaca, níveis de oxigênio no sangue
etc). Em ambos os casos - sensores gerais e sensores médicos -, as informações
coletadas pelos dispositivos repassam  esses dados para o STB, seja através de
interfaces com tecnologia  \textit{Bluetooth}, \textit{NFC} ou \textit{Wi-Fi}.

Essa quantidade de dados enviada ao STB deverá então ser processada e encaminhada
para a plataforma NextSaúde. O processamento é referente ao tratamento do dado
bruto recebido e a tradução para o padrão OpenEHR entendido pela plataforma
NextSaúde. Em seguida a transmissão dos dados é feita através de uma
arquitetura  cliente/servidor. Já no sistema na nuvem, o dado relacionado ao
paciente é extraído e persistido.

Para permitir ao STB tomar decisões no âmbito domiciliar, parte dos dados
referentes ao contexto são tratados e processados diretamente no equipamento.
Já os dados referentes aos sinais vitais são enviados para plataforma. Dessa 
maneira, ocorre um processamento nos dados para tomada de decisões a nível
clínico, por exemplo, indicar uma consulta na especialidade cardiológica.

Pelo exposto, uma funcionalidade importante do sistema é a geração de alertas e
notificações para o paciente e o cuidador. Os atores recebem as notificações
através da tela da televisão conectada ao STB. Para esse sistema, dois tipos de
alertas foram desenvolvidos: o tipo \textit{pop-up} e o caixa de diálogo.
Além disso, a partir das informações de contexto, o sistema pode decidir
entregar a mensagem não na tela da televisão e sim no \smartphone.


\section{TV-Health} \label{sec:tv-health} 

O protótipo TV-Health surgiu no âmbito do projeto NextSaúde. 

\section{Visão de Engenharia} \label{sec:visao-engenharia}

\lipsum[1]

\section{Cenários de uso} \label{sec:cenarios-de-uso}

\lipsum[1]