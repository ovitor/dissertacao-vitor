\chapter{Introdução}\label{cap:introducao}

O Brasil vem passando por um processo de envelhecimento da população e
um aumento da expectativa de vida crescente desde a década de 1960. 
Com os atuais índices, a taxa do envelhecimento populacional atingirá, 
em 2025, cerca de 15\% da população brasileira com indivíduos acima 
de 60 anos \cite{gonccalves2006perfil}. 

Os mais idosos, por conta da fragilidade inerente à idade, necessitam de 
cuidados especiais na hospitalização, além de demandarem mais tempo na 
recuperação. Muitas vezes, há uma demanda de vários profissionais de uma equipe 
de saúde multidisciplinar para a total recuperação do doente.

Essa situação afeta políticas públicas dos governos municipais, estaduais e
federal que, segundo a Constituição Federal, devem prover atendimento
hospitalar universal à população, o que torna o gasto público com hospitalização
de idosos maior a cada ano \cite{da2013gastos}.

Esse processo ocasiona mudanças  nos paradigmas de atendimento à saúde. Nessa
perspectiva, surge a atenção domiciliar ou \textit{home care}, modelo definido
como o tratamento do paciente em seu próprio lar, com a presença ou não de um
cuidador\footnote{Neste trabalho, a figura do cuidador refere-se tanto a
familiares, amigos ou um profissional remunerado.}, figura responsável por
acompanhar o idoso em suas atividades diárias, papel de fundamental importância
no acompanhamento do paciente em seu cotidiano.

Esse método traz benefícios, pois o  paciente encontra-se em um ambiente
conhecido, podendo contar com a presença de seus familiares
\cite{hermann2007atendimento, day2010beneficios}. Destaque-se que o ambiente
domiciliar é seguro, livre de infecção hospitalar, portanto mais propício para
uma completa recuperação do paciente.

Pesquisas realizadas indicam uma mudança gradativa no modelo de tratamento
de idosos. A escolha da atenção domiciliar em substituição à hospitalização 
mostra-se positiva nos planos social, psicológico e econômico, como exposto
no trabalho de \citeonline{santos2014}.

No que concerne às políticas públicas desenvolvidas pelo Estado,  a vantagem é
a redução de custos com internação. Estudos na área da saúde pública revelam
que é possível economizar ao se substituir a internação hospitalar por uma
abordagem em atenção domiciliar nos casos de menor gravidade, ou seja, casos
em que o paciente não corre risco de morte \cite{bourdette1993health}.  

Na atenção domiciliar é de grande importância o acompanhamento do paciente pelo
seu responsável.  Essa tarefa é realizada por diversos atores, além do
cuidador. A equipe de saúde, por exemplo, se desloca até o domicílio do idoso e
executa a aferição de seus sinais vitais. Além de conversar com o próprio
paciente, a equipe de saúde também conversa com o cuidador e verifica se o
ambiente domiciliar está propício à melhoria do doente.

Nesse contexto, vale observar que o aparelho de TV está presente em,
praticamente, todos os  domicílios brasileiros \cite{ibge2015tv}.  Com isso, o
Governo Brasileiro instituiu, via decreto número 4901 de 26 de novembro de
2003, o Sistema Brasileiro de TV Digital (SBTVD) no país.  Além da
melhoria técnica, com a troca do sinal analógico para o sinal  digital,
promovendo, assim, melhoria na qualidade da imagem e som, a inserção  da TV
Digital trouxe o conceito de interatividade \cite{digitaltv2015decree}.

O ambiente domiciliar está cada vez mais relacionado com o avanço tecnológico
nas áreas de sistemas embarcados e tecnologia da informação e comunicação. A
miniaturização dos dispositivos possibilita o surgimento de novas aplicações e
soluções voltadas para áreas antes não atendidas por essas tecnologias.  A
partir dessa evolução, surge a Internet das Coisas (\textit{Internet of
Things}, IoT), bilhões de ``coisas'' inteligentes que se comunicam entre si
\cite{li2015internet}. Os objetos ou ``coisas'' são pequenos sistemas
embarcados com sensores e/ou atuadores com um meio de comunicação. 

A Assistência Domiciliar à Saúde (ADS) se beneficia do aperfeiçoamento na
tecnologia e do surgimento da IoT ao ser atendida por dispositivos implantados
no ambiente doméstico.  Pesquisas realizadas \cite{triantafyllidis2013,moreira2016} 
apontam para a utilização de equipamentos que monitorem o cotidiano do paciente
e os auxilie nas formas de alertas para o paciente e o cuidador, além de
informar à equipe de saúde ou a emergência sobre questões pertinentes do quadro 
observado auxiliando na tomada de decisão.

A comunicação que antes ocorria principalmente entre homens e máquinas na ``Internet
dos computadores'' muda com o advento da IoT, e passa a ser principalmente entre
máquinas, sendo elas (as coisas) os maiores produtores e consumidores de dados.
Com isso, é possível perceber que a tecnologia IoT traz consigo a possibilidade de
produzir uma grande quantidade de dados e permite que a interação homem-máquina
seja atualizada, proporcionando novas formas de interação, o que se adequa a 
um ambiente domiciliar do futuro.

O projeto Lariisa, uma arquitetura para tomada de decisão em governança
de sistemas públicos de saúde, trata, em uma de suas instâncias, da questão do
ambiente domiciliar. No Lariisa, todos são tomares de decisão, desde o paciente
ao gestor, passando por todos os agentes de saúde (enfermeiro, fisioterapeuta,
médicos etc.). Nele, são levados em consideração os cinco domínios de inteligência:
Gestão de conhecimento, Normatização sistêmica, Clínico e Epidemiológico,
Administrativa e Conhecimento compartilhado \cite{oliveira2010context}.

Este trabalho descreve a perspectiva computacional de cenários de ADS e propõe a TV
Health das Coisas, uma solução inteligente e orientada a contexto para o
ambiente de ADS, inspirada no projeto Lariisa, baseada no modelo brasileiro de
TV digital e na tecnologia Internet das Coisas (IoT). Esta solução tem como
substrato de \hardware[] (\stb) um sistema embarcado associado à uma TV digital
(\textit{hub} de comunicação) que coleta dados de diversos sensores existentes
no ambiente de ADS. Estes são tratados por módulos inteligentes no
\nextsaude[], uma plataforma de interoperabilidade em saúde baseada no padrão
OpenEHR, para compartilhar serviços internos disponíveis (regulação, farmácia,
leitos etc.) \cite{mota2016}, da qual a TV Health das Coisas é um componente.
Além disso, esta solução permite que a TV sirva de interface para o paciente,
e/ou cuidador, a alertas, a recomendações e a outras aplicações interativas.

O STB é o dispositivo utilizado como \textit{hub} de comunicação na arquitetura
proposta, conectando sensores e outros componentes com tecnologias e conceitos
de IoT para  prover uma solução inteligente de ADS. Além
disso, tem poder de processamento para executar outras funções.

Através dos sensores é possível coletar dados referentes ao paciente e ao
contexto em que ele está inserido. Os dados coletados são repassados ao STB que
os concentra e realiza um processamento. Estes dados vão alimentar também
modelos de ontologias, permitindo a inferência de informações necessárias para
à tomada de decisão por parte dos atores  envolvidos (paciente, cuidador,
equipe de saúde e familiares). Dessa maneira, o sistema pode auxiliar os atores
envolvidos, emitindo alertas, informações, mensagens, entre outros, por meio da
TV ou dispositivos móveis integrados ao sistema. 

\section{Motivação para a Dissertação}\label{sec:motivacao}

Cresce cada vez mais a preocupação com os idosos, haja vista o aumento da
expectativa de vida. Uma preocupação dos sistemas públicos é a qualidade de
vida na velhice considerando que, nessa idade, os problemas de saúde são
acentuados. A idade provoca condições que impedem a autonomia, aumentam as
dificuldades de locomoção, e intensificam a preocupação com remédios, entre
outros.

Em muitos casos, cresce o número de internações hospitalares e,
consequentemente, os gastos do Estado - que precisam ser mantidos, mas que
podem ser mitigados ou melhor aproveitados.  Visando enfrentar essa realidade,
a comunidade acadêmica começa a se perguntar sobre os cuidados aos idosos no
espaço domiciliar.

A linha de pesquisa de computação aplicada à saúde é vista como uma parte
importante para a melhoria na qualidade dos serviços prestados na área. A
comunidade acadêmica vem possibilitando o avanço no campo da ADS, desenvolvendo
sistemas inteligentes que auxiliem no tratamento e ajudem os envolvidos.

A TV Health das Coisas é voltada, prioritariamente, à proposição de soluções
tecnológicas, modernas, com IoT, para ADS. A expectativa é que este trabalho
contribua fortemente com as políticas públicas nacionais que diz respeito a
questão do atendimento domiciliar e, eventualmente, com soluções privadas
relativas ao tema.

\section{Descrição do problema}\label{sec:descricao-problema}

Tendo em vista o cenário de atenção domiciliar, no qual o paciente
encontra-se em tratamento e deve ser observado constantemente, 
se faz necessário um acompanhamento por parte do cuidador, que, ao perceber 
a piora do paciente realiza algum procedimento e em alguns casos entra 
em contato com o socorro médico ou a equipe de saúde responsável pelo paciente.

O cuidador é, geralmente, um familiar, um amigo, ou um profissional remunerado.
Não, necessariamente, o cuidador detém uma formação na área de saúde apropriada
para a função que desempenha, ou não disponibiliza toda a sua atenção ao
paciente. Do exposto, percebe-se a dificuldade do cuidador de realizar um
procedimento de saúde mais elaborado ou auxiliar o paciente durante 24 horas do
dia.

Já a equipe de saúde, devido a não proximidade com o doente, carece de informações
prévias para um socorro direcionado ao paciente, diminuindo assim, as chances de
sucesso do atendimento realizado.

Segundo \citeonline{santos2014}, constata-se que a ADS é possível e desejável,
desde que alguns obstáculos sejam superados.  O auxílio ao paciente, ao
cuidador ou, ainda, à equipe de saúde que o acompanha, necessita de soluções
eficientes e acessíveis.

Urge, portanto, soluções tecnológicas que possam mitigar os problemas envolvendo
a atuação dos diversos atores no ambiente ADS, em especial o cuidador,
foco principal deste trabalho.

\section{Objetivo Geral}\label{sec:objetivos}

Desenvolver uma plataforma sensível ao contexto para ambientes de Assistência
Domiciliar à Saúde (ADS), observando em sua arquitetura seus aspectos
funcionais e visão de engenharia, baseada na tecnologia de Internet das Coisas
(IoT), com foco na análise arquitetural da plataforma OpenIoT.

\section{Objetivos Específicos}

\begin{itemize}
  \item Realizar estudo sobre o estado da arte de IoT aplicado a Assistência
    Domiciliar à Saúde;
  \item Modelar cenários ADS dentro de uma perspectiva computacional;
  \item Identificar aspectos funcionais que atendam os requisitos do ambiente
    de ADS;
  \item Descrever elementos estruturais que apresentem a visão de engenharia da
    solução proposta;
  \item Analisar as correlações funcionais entre a plataforma OpenIoT e os
    requisitos a serem observados na concepção da arquitetura TV Health das
    Coisas;
  \item Apresentar uma arquitetura sensível a contexto para ambientes de
    Assistência Domiciliar à Saúde (ADS) baseada em IoT, a partir, dos aspectos
    funcionais identificados na visão de engenharia;
  \item Implementar um protótipo como prova de conceito da arquitetura proposta
    atendendo aos requisitos do ambiente ADS;
  \item Desenvolver APIs para agregação de novas aplicações a plataforma TV
    Health das Coisas;
\end{itemize}

\section{Estrutura da Dissertação}\label{sec:estrutura}

O restante da dissertação é dividido em 5 capítulos:
\nameref{cap:fundamentacao-teorica}, \nameref{cap:trabalhos-relacionados},
\nameref{cap:quatro}, \nameref{cap:aspectos-de-implementacao} e
\nameref{cap:conclusao}. 

No capítulo \ref{cap:fundamentacao-teorica}, são descritos conceitos de Atenção
Domiciliar à Saúde, Sistemas Embarcados, Internet das Coisas -
(\textit{Internet of Things, IoT}) e Aplicações Sensíveis ao Contexto,
separados nas seções \nameref{sec:ads}, \nameref{sec:sistemas-embarcados},
\nameref{sec:iot} e \nameref{sec:contexto}. Além de descrever os trabalhos
sobre a arquitetura \nameref{sec:lariisa} e a \nameref{sec:openiot}.

No capítulo \ref{cap:trabalhos-relacionados} é apresentado o estado da arte de
sistemas para a ADS, assim como outros trabalhos relacionados à proposta
apresentada.

No capítulo \ref{cap:quatro} são descritos os cenários de ADS e a arquitetura TV
Health das Coisas, principal contribuição do trabalho. Além disso, é apresentada a visão
funcional da TV Health das Coisas, com uma descrição do percurso metodológico na
descrição da arquitetura, além da visão de engenharia da solução.

O capítulo \ref{cap:aspectos-de-implementacao} aborda detalhes da implementação
referente a plataforma TV Health das Coisas, tais como os casos de uso do módulo 
de notificações, do módulo de auxílio e do módulo de botão de pânico.

Por fim, no capítulo \ref{cap:conclusao} é exposta a conclusão com as 
considerações finais e os trabalhos futuros na perspectiva da TV Health das Coisas
tornar-se um modelo de referência para o desenvolvimento de aplicações para ambientes
de ADS.

