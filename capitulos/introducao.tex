\chapter{Introdução}\label{cap:introducao}

O Brasil vem passando por um processo de envelhecimento da população e
um aumento da expectativa de vida crescente desde a década de 1960. 
Com os atuais índices, a taxa do envelhecimento populacional atingirá, 
em 2025, cerca de 15\% da população brasileira com indivíduos acima 
de 60 anos. 

Os mais idosos, por conta da fragilidade inerente à idade, necessitam de 
cuidados especiais na hospitalização, além de demandarem mais tempo na 
recuperação. Muitas vezes há uma demanda de vários profissionais de uma equipe 
médica multidisciplinar para se recuperar totalmente.

Essa situação afeta políticas públicas dos governos municipais, estaduais e federal -
que, segundo a Constituição Federal, deve prover atendimento hospitalar
universal à população - tornando o gasto público com hospitalização de idosos
maiores a cada ano.

Portanto, esse processo que a população brasileira vem sofrendo, ocasiona mudanças 
nos paradigmas de atendimento à saúde. Nessa perspectiva, surge a atenção domiciliar
%Uma outra abordagem para casos semelhantes é a atenção domiciliar
ou \textit{home care}, modelo definido como o tratamento do paciente 
em seu próprio lar, com a presença ou não de um cuidador - figura responsável
por acompanhar o idoso em suas atividades diárias, ao assumir um papel 
de fundamental importância no acompanhamento do paciente em seu cotidiano. 

Pesquisas mostram que esse método traz benefícios pois o 
paciente encontra-se em um ambiente conhecido e está na presença contínua de
seus familiares.

Assim, pesquisas realizadas indicam uma mudança gradativa no modelo de
tratamento de idosos. A escolha da atenção domiciliar em detrimento da
hospitalização traz benefícios sociais, psicológicos e
econômicos para o paciente. %, mas também benefícios econômicos para o Estado.

No que concerne às políticas públicas desenvolvidas pelo Estado, 
a vantagem é de reduzir os custos com internação. Estudos revelam 
que é possível reduzir custos de internação hospitalar com uma 
abordagem em atenção domiciliar nos casos de menor gravidade, 
ou seja, casos em que o paciente não corre risco de morte. 
% o aumento de casos em que o paciente pode ser tratado v

\section{Motivação para a Dissertação}\label{sec:motivacao}

Observo na sociedade atual, através de notícias divulgadas, nos meios
de comunicação, de conversas informais, de filmes etc que está cada
vez mais crescente a preocupação com os idosos, haja vista o
aumento da expectativa de vida. Me pergunto como é possível essas pessoas
terem uma qualidade de vida na velhice se sabemos que, nessa idade,
os problemas de saúde são acentuados, como falta de autonomia,
dificuldades de locomoção, preocupação com remédios etc.

Em muitos casos, cresce o número de internações hospitalares e,
consequentemente, os gastos do Estado. Visando enfrentar essa 
realidade, a comunidade acadêmica, % a partir de meados dos anos tais,
começa a se pesquisar sobre os cuidados aos idosos no espaço domiciliar.

A linha de pesquisa de computação aplicada à saúde é vista como uma parte
importante para a melhoria na qualidade dos serviços prestados na área.
Com a pesquisa realizada, a comunidade acadêmica possibilitou o avanço
de atenção domiciliar, pesquisando e desenvolvendo sistemas inteligentes
que auxiliem no tratamento e ajudem os envolvidos.

\section{Descrição do problema}\label{sec:descricao-problema}



A partir do descrito, se pode identificar que a internação domiciliar
de pacientes idosos é possível. O auxílio ao idoso, ao cuidador
ou ainda, à equipe médica que o acompanha carece de soluções eficientes
e acessíveis financeiramente.

\section{Objetivos Geral e Específicos}\label{sec:objetivos}

Em seu sentido mais particular, os seguintes objetivos específicos são:

\begin{itemize}
	\item ...; 
	\item ...;
\end{itemize}

\section{Produção científica}\label{sec:producao}
Durante este projeto de mestrado, os seguintes trabalhos científicos foram aceitos e publicados, a saber:

\begin{itemize}
	\item \textbf{Einstein, A.}, 1905. \textbf{The photoelectric effect}. Ann. Phys, 17(132), p.4;
	\item \textbf{Einstein, A.}, 1904. \textbf{Zur allgemeinen molekularen Theorie der Wärme}. Annalen der Physik, 319(7), pp.354-362.
\end{itemize}

\section{Estrutura da Dissertação}\label{sec:estrutura}
\lipsum[1]
