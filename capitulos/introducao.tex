\chapter{Introdução}\label{cap:introducao}

O Brasil vem passando por um processo de envelhecimento da população e
um aumento da expectativa de vida crescente desde a década de 1960. 
Com os atuais índices, a taxa do envelhecimento populacional atingirá, 
em 2025, cerca de 15\% da população brasileira com indivíduos acima 
de 60 anos \cite{gonccalves2006perfil}. 

Os mais idosos, por conta da fragilidade inerente à idade, necessitam de 
cuidados especiais na hospitalização, além de demandarem mais tempo na 
recuperação. Muitas vezes há uma demanda de vários profissionais de uma equipe 
médica multidisciplinar para a total recuperação do doente.

Essa situação afeta políticas públicas dos governos municipais, estaduais e
federal - que, segundo a Constituição Federal, devem prover atendimento
hospitalar universal à população - tornando o gasto público com hospitalização
de idosos maiores a cada ano \cite{da2013gastos}.

Portanto, esse processo que a população brasileira vem atravessando, ocasiona
mudanças  nos paradigmas de atendimento à saúde. Nessa perspectiva, surge a
atenção domiciliar ou \textit{home care}, modelo definido como o tratamento do
paciente em seu próprio lar, com a presença ou não, de um
cuidador\footnote{Neste trabalho, a figura do cuidador refere-se tanto a
familiares, amigos ou um profissional com remuneração.} - figura responsável por
acompanhar o idoso em suas atividades diárias, ao assumir um papel de
fundamental importância no acompanhamento do paciente em seu cotidiano.

Pesquisas indicam que esse método traz benefícios, pois o  paciente encontra-se
em um ambiente conhecido, podendo contar com a presença de seus familiares
\cite{hermann2007atendimento, day2010beneficios}.

Assim, estudos realizados indicam uma mudança gradativa no modelo de tratamento
de idosos. A escolha da atenção domiciliar em substituição da hospitalização 
mostra-se positivo nos planos social, psicológico e econômico. 

No que concerne às políticas públicas desenvolvidas pelo Estado,  a vantagem é
de reduzir os custos com internação. Estudos revelam  que é possível economizar,
ao se substituir a internação hospitalar por uma  abordagem em atenção 
domiciliar nos casos de menor gravidade,  ou seja, casos em que o paciente não 
corre risco de morte \cite{bourdette1993health}.  

Na atenção domiciliar, é de grande importância o acompanhamento do paciente.
Essa tarefa é realizada por uma equipe médica que se desloca até o domicílio do
idoso e executa a aferição de seus sinais vitais, além de conversar com o
próprio paciente, eles também conversam com o cuidador e verificam, ainda, se o
ambiente domiciliar está propício à melhoria do doente.

A televisão está presente em praticamente todos os  domicílios brasileiros. Com
isso, o Governo Brasileiro instituiu, através do decreto número 4901 de 26 de
novembro de 2003, o Sistema Brasileiro de Televisão Digital (SBTVD) no país.
Além da melhoria técnica, com a troca do sinal analógico para o sinal  digital,
promovendo, assim, melhoria na qualidade da imagem e som, a inserção  da TV
Digital trouxe o conceito de interatividade, permitindo que o  telespectador
saia da condição de espectador e passe a interagir com a  televisão. Além disso,
através do decreto, o Governo tenta promover a inclusão social ao ``propiciar a
criação de uma rede universal de educação à distância''
\cite{digitaltv2015decree}.

O avanço tecnológico nas áreas de sistemas embarcados e tecnologia da informação
e comunicação permite o desenvolvimento e a produção de dispositivos cada vez
mais potentes e eficientes no processamento e no consumo de energia. Além disso,
a miniaturização dos dispositivos possibilitam o surgimento de novas aplicações
e soluções voltadas para áreas antes não atendidas por essas tecnologias.

A área da assistência domiciliar à saúde se beneficia do aperfeiçoamento na 
tecnologia ao ser atendida por dispositivos implantados no ambiente doméstico. 
Pesquisas realizadas apontam para a utilização de equipamentos que monitorem o 
cotidiano do paciente e os auxilie nas formas de alertas para o paciente e 
cuidador, além de informar à equipe de saúde ou a emergência sobre questões 
pertinentes ao paciente.

Neste trabalho, é proposta a TVHealthDasCoisas (THECA), um sistema embarcado em
um \stb[] (STB) - dispositivo responsável por receber o sinal da  televisão
digital e apresentá-lo na TV - foi utilizado como ponto principal em uma
arquitetura composta por ele (STB), sensores e outros equipamentos para  prover
uma solução inteligente de atenção domiciliar.

O STB, além de realizar suas tarefas básicas explicadas anteriormente, tem poder
de processamento para executar outras funções. Ele conta com interface de
comunicação Wi-Fi, portas \textit{ethernet} e HDMI e o padrão RCA para
conexão com a TV, além de portas \textit{USB}. Sendo assim, uma vez conectado à
Internet, o STB servirá de \textit{gateway} na casa do usuário, comunicando-se
com sistemas de informática para saúde e com o próprio paciente.

Através dos sensores é possível coletar dados referentes ao paciente e ao
contexto em que ele está inserido, como por exemplo, se ele está acamado ou se
possui certa autonomia, ou ainda, se o doente está acompanhado.  Em seguida, os
dados coletados são repassados ao STB que  os concentra e realiza um
processamento. Estes dados, vão alimentar ainda, modelos de ontologias, 
permitindo a inferência de informações necessárias para à tomada de decisão
por parte dos atores  envolvidos (paciente, cuidador, equipe de saúde e 
familiares). 

Dessa maneira, o sistema pode auxiliar os atores envolvidos, emitindo alertas,
informações, mensagens etc por meio da televisão ou dispositivos móveis
integrados ao sistema. A THECA comunica-se com o NextSAÚDE - uma plataforma
de interoperabilidade em saúde baseada no padrão OpenEHR - para compartilhar
serviços internos disponíveis (regulação, farmácia, leitos etc).

Portanto, a THECA é uma arquitetura orientada a contexto (\textit{context-aware
concept}), enriquecida com tecnologia da Internet das Coisas (\textit{IoT}), que
através de um STB, coleta dados e apresenta alertas e informações processadas e
comunica-se com a plataforma NextSAÚDE através de arquétipos do padrão OpenEHR.
Até o momento uma parte desse processo já foi implementada, tal como as 
notificações dos alertas enviados ao doente e cuidador. A fase seguinte 
dirá a respeito ao envio constante dos dados para a equipe médica.

% Vale ressaltar que este dispositivo é o resultado de um projeto de pesquisa, do
% qual participei de forma efetiva, que recebeu fomento da  Agência Nacional de
% Energia Elétrica (ANEEL), em parceria com a Empresa CRAFF,  sediada no Município
% de Fortaleza.

\section{Motivação para a Dissertação}\label{sec:motivacao}

Observo na sociedade atual, através de notícias divulgadas nos meios
de comunicação, de conversas informais, de filmes etc, que está cada
vez mais crescente a preocupação com os idosos, haja vista o
aumento da expectativa de vida. Me pergunto como é possível essas pessoas
terem uma qualidade de vida na velhice se sabemos que, nessa idade,
os problemas de saúde são acentuados, tais como, falta de autonomia,
dificuldades de locomoção, preocupação com remédios, entre outros.

Em muitos casos, cresce o número de internações hospitalares e,
consequentemente, os gastos do Estado. Visando enfrentar essa 
realidade, a comunidade acadêmica, % a partir de meados dos anos tais,
começa a se perguntar sobre os cuidados aos idosos no espaço domiciliar.

A linha de pesquisa de computação aplicada à saúde é vista como uma parte
importante para a melhoria na qualidade dos serviços prestados na área. Por meio
de pesquisas, a comunidade acadêmica possibilitou o avanço no campo da atenção
domiciliar, desenvolvendo sistemas inteligentes que auxiliem no tratamento e
ajudem os envolvidos.

\section{Descrição do problema}\label{sec:descricao-problema}

Tendo em vista o cenário de atenção domiciliar no qual o paciente
encontra-se em tratamento e deve ser observado constantemente, 
se faz necessário um acompanhamento por parte do cuidador, que, ao perceber 
a piora do paciente realiza algum procedimento e em alguns casos, entra 
em contato com o socorro médico ou a equipe médica responsável pelo paciente.

O cuidador, como já afirmamos, é geralmente, um familiar, amigo, ou profissional
remunerado, sem formação médica apropriada para a função que desempenha  e sem,
necessariamente, disponibilizar todo o seu horário para o paciente. Do exposto,
percebe-se a dificuldade do cuidador de realizar um procedimento médico mais
elaborado ou auxiliar o paciente durante 24 horas do dia.

Já a equipe médica, devido a não proximidade com o doente, carece de informações
prévias para um socorro direcionado ao paciente, diminuindo assim, as chances de
sucesso do atendimento realizado.

%Outro aspecto que não pode ser ignorado, diz respeito ao custo financeiro de
%uma internação hospitalar. Estudos mostram que o fato de o paciente não pagar
%a diária de um hospital, já reduz os seus gastos em torno de 72\%. Isso também
%traz benefícios para o Estado. Por outro lado...
% vê o que os artigos falam a respeito do paciente. 

A partir das observações citadas anteriormente, constata-se que a assistência
domiciliar é possível e desejável, desde que alguns obstáculos sejam superados.
Portanto, o auxílio ao paciente, ao cuidador ou ainda, à equipe médica que o
acompanha necessita de soluções eficientes e acessíveis.

\section{Objetivos Geral e Específicos}\label{sec:objetivos}

Oferecer uma solução que auxilie, através de sistemas de informática, tecnologia
de Internet das Coisas e os meios de comunicação (Televisão Digital e 
\textit{smartphones}) o cuidador, o doente e à equipe de saúde no tratamento 
do paciente em domicílio.
% por meio das tecnologias tais e tais..
%Em seu sentido mais particular, os seguintes objetivos específicos são:

\begin{itemize}
  \item Permitir um acompanhamento constante da equipe médica ao paciente;
  \item Facilitar as atividades diárias do cuidador em relação aos procedimentos
  destinados ao doente;
\end{itemize}

\section{Produção científica}\label{sec:producao}  

Durante este projeto de mestrado, os seguintes trabalhos científicos foram
aceitos e publicados, a saber:

\begin{itemize}
	% \item \textbf{Einstein, A.}, 1905. \textbf{The photoelectric effect}. Ann. Phys, 17(132), p.4;
	% \item \textbf{Einstein, A.}, 1904. \textbf{Zur allgemeinen molekularen Theorie der Wärme}. Annalen der Physik, 319(7), pp.354-362.
  \item produção 1
\end{itemize}

\section{Estrutura da Dissertação}\label{sec:estrutura}
