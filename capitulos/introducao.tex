\chapter{Introdução}\label{cap:introducao}

O Brasil vem passando por um processo de envelhecimento da população e
um aumento da expectativa de vida crescente desde a década de 1960. 
Com os atuais índices, a taxa do envelhecimento populacional atingirá, 
em 2025, cerca de 15\% da população brasileira com indivíduos acima 
de 60 anos \cite{gonccalves2006perfil}. 

Os mais idosos, por conta da fragilidade inerente à idade, necessitam de 
cuidados especiais na hospitalização, além de demandarem mais tempo na 
recuperação. Muitas vezes há uma demanda de vários profissionais de uma equipe 
médica multidisciplinar para se recuperar totalmente.

Essa situação afeta políticas públicas dos governos municipais, estaduais e
federal - que, segundo a Constituição Federal, devem prover atendimento
hospitalar universal à população - tornando o gasto público com hospitalização
de idosos maiores a cada ano \cite{da2013gastos}.

Portanto, esse processo que a população brasileira vem atravessando, ocasiona
mudanças  nos paradigmas de atendimento à saúde. Nessa perspectiva, surge a
atenção domiciliar %Uma outra abordagem para casos semelhantes é a atenção
ou \textit{home care}, modelo definido como o tratamento do paciente
em seu próprio lar, com a presença, de cuidador\footnote{Neste trabalho, a
figura do cuidador refere-se tanto a familiares, amigos ou um profissional com
remuneração.} - figura responsável por acompanhar o idoso em suas atividades
diárias, ao assumir um papel de fundamental importância no acompanhamento do
paciente em seu cotidiano.

Pesquisas mostram que esse método traz benefícios pois o  paciente encontra-se
em um ambiente conhecido, podendo contar com a presença de seus familiares
\cite{hermann2007atendimento, day2010beneficios}.

Assim, estudos realizados indicam uma mudança gradativa no modelo de tratamento
de idosos. A escolha da atenção domiciliar em detrimento da hospitalização traz
benefícios sociais, psicológicos e econômicos para o paciente. %, mas também
benefícios econômicos para o Estado.

No que concerne às políticas públicas desenvolvidas pelo Estado,  a vantagem é
de reduzir os custos com internação. Estudos revelam  que é possível economizar
substituindo a internação hospitalar por uma  abordagem em atenção domiciliar
nos casos de menor gravidade,  ou seja, casos em que o paciente não corre risco
de morte \cite{bourdette1993health}.  % o aumento de casos em que o paciente 
% pode ser tratado v

Nesse sentido, foi modelado um sistema computacional para atender idosos com
doenças crônicas em situação de atenção domiciliar. Como forma de validação, 
uma prova de conceito com um sistema baseado em um \stb[] (STB) da TV Digital e
sensores foi desenvolvido.

\section{Motivação para a Dissertação}\label{sec:motivacao}

Observo na sociedade atual, através de notícias divulgadas nos meios
de comunicação, de conversas informais, de filmes etc, que está cada
vez mais crescente a preocupação com os idosos, haja vista o
aumento da expectativa de vida. Me pergunto como é possível essas pessoas
terem uma qualidade de vida na velhice se sabemos que, nessa idade,
os problemas de saúde são acentuados, tais como, falta de autonomia,
dificuldades de locomoção, preocupação com remédios, entre outros.

Em muitos casos, cresce o número de internações hospitalares e,
consequentemente, os gastos do Estado. Visando enfrentar essa 
realidade, a comunidade acadêmica, % a partir de meados dos anos tais,
começa a se perguntar sobre os cuidados aos idosos no espaço domiciliar.

A linha de pesquisa de computação aplicada à saúde é vista como uma parte
importante para a melhoria na qualidade dos serviços prestados na área. Por meio
de pesquisas, a comunidade acadêmica possibilitou o avanço no campo da atenção
domiciliar, desenvolvendo sistemas inteligentes que auxiliem no tratamento e
ajudem os envolvidos.

\section{Descrição do problema}\label{sec:descricao-problema}

Tendo em vista o cenário de atenção domiciliar no qual o paciente
encontra-se em tratamento e deve ser observado constantemente, 
se faz necessário um acompanhamento por parte do cuidador, que, ao perceber 
a piora do paciente realiza algum procedimento e em alguns casos, entra 
em contato com o socorro médico ou a equipe médica responsável pelo paciente.

O cuidador, como já afirmamos, é geralmente, um familiar, amigo, ou profissional
remunerado, sem formação médica apropriada para a função que desempenha  e sem,
necessariamente, disponibilizar todo o seu horário para o paciente. Do exposto,
percebe-se a dificuldade do cuidador de realizar um procedimento médico mais
elaborado ou auxiliar o paciente durante 24 horas do dia.

Já a equipe médica, devido a não proximidade com o doente, carece de informações
prévias para um socorro direcionado ao paciente, diminuindo assim, as chances de
sucesso do atendimento realizado.

%Outro aspecto que não pode ser ignorado, diz respeito ao custo financeiro de
%uma internação hospitalar. Estudos mostram que o fato de o paciente não pagar
%a diária de um hospital, já reduz os seus gastos em torno de 72\%. Isso também
%traz benefícios para o Estado. Por outro lado...
% vê o que os artigos falam a respeito do paciente. 

A partir das observações citadas anteriormente, constata-se que a assistência
domiciliar é possível e desejável, desde que alguns obstáculos sejam superados.
Portanto, o auxílio ao paciente, ao cuidador ou ainda, à equipe médica que o
acompanha necessita de soluções eficientes e acessíveis.

\section{Objetivos Geral e Específicos}\label{sec:objetivos}

Oferecer uma solução que auxilie, através de sistemas de informática, o 
cuidador, o doente e à equipe médica no tratamento do paciente em domicílio.

%Em seu sentido mais particular, os seguintes objetivos específicos são:

\begin{itemize}
  \item Permitir um acompanhamento constante da equipe médica ao paciente;
  \item Facilitar as atividades diárias do cuidador em relação aos procedimentos
  destinados ao doente;
\end{itemize}

\section{Produção científica}\label{sec:producao}  

Durante este projeto de mestrado, os seguintes trabalhos científicos foram
aceitos e publicados, a saber:

\begin{itemize}
	\item \textbf{Einstein, A.}, 1905. \textbf{The photoelectric effect}. Ann. Phys, 17(132), p.4;
	% \item \textbf{Einstein, A.}, 1904. \textbf{Zur allgemeinen molekularen Theorie der Wärme}. Annalen der Physik, 319(7), pp.354-362.
\end{itemize}

\section{Estrutura da Dissertação}\label{sec:estrutura}
