\chapter{Fundamentação Teórica}\label{cap:fundamentacao-teorica}

Este capítulo apresenta a fundamentação teórica, separados em aspectos
de saúde e tecnológicos. As seções de saúde abordam termos como atenção
domiciliar e urgência e emergência, além de explanar, brevemente, sobre
a história da hospitalização no Brasil. 

\section{Aspectos de Saúde}\label{sec:aspectos-de-saude}

Segundo o livro ``Humanização nos Hospitais do Brasil'' \cite{inaia2008} de Inaiá  Mello,
temos a informação que a instituição Hospital está presente em nosso país
desde a criação da Santa casa de Misericórdia de Santos, em 1543, permanecendo
incólume até o final do século XIX, quando houve uma transformação, tornando
o hospital uma organização complexa e dispendiosa.

No Governo Getúlio Vargas, nos anos de 1933 à 1940, diversos Institutos de
Previdência foram criados. Eles eram encarregados, principalmente, de prestar
assistência médica-hospitalar. Já em 1953, ainda no Governo de Vargas, surge 
o Serviço de Assistência Médica Domiciliar e de Urgência (SAMDU), seu 
objetivo era o atendimento domiciliar, ambulatorial e auxílio aos Institutos 
de Previdência.

Ainda segundo Inaiá, a década de 1970, foi marcada pelas tentativas, do poder público, no 
sentido de universalizar o acesso à assistência à saúde. Alguns Programas e 
Sistemas foram iniciados, sendo válido citar, (1) Sistema Único e Descentralizado 
de Saúde, o SUDS e o Sistema Único de Saúde, o SUS.

O SUS foi criado oficialmente pela Constituição Federal de 1988, Lei 8080/90
\footnote{Acessível em \url{http://www.planalto.gov.br/ccivil_03/leis/L8080.htm}},
com o objetivo de garantir à população Brasileira, o acesso universal às
ações e serviços de saúde.

Com o surgimento da medicina científica, a melhoria na tecnologia e na
infraestrutura hospitalar, os hospitais deixaram de ser espaços para 
abrigarem pobres e doentes e passaram a proporcionar tratamentos mais
elaborados. Como consequência, surgiram diversas mudanças no atendimento,
onde a Assistência Domiciliar à Saúde (ADS) se tornou uma modalidade
disponível.

\subsection{Assistência Domiciliar à Saúde}\label{subsec:atencao-domiciliar}

A Assistência Domiciliar à Saúde (ADS) divide-se basicamente em grupos de
enfermagem e fisioterapia - nas modalidades mais básicas - até um atendimento
multiprofissional, possibilitando um apoio ao paciente como um todo. A ADS
pode ser provida tanto pelo setor privado quanto pelo setor público \cite{amaral2001assistencia}.

Os primeiros registros da ADS no Brasil surgem em 1967, na cidade de São
Paulo, no Hospital do Servidor Público. O principal objetivo dessa abordagem
era a liberação de leitos no hospital, levando para o domicílio procedimentos
básicos, de baixa complexidade clínica.

Já na década de 90, houve um aumento considerável na quantidade de empresas
privadas provendo o serviço de ADS, apenas 5 empresas prestavam esse tipo de 
serviço, já em 1999, esse número subiu para mais de 180.

Amaral et al define a ADS como uma sequência de serviços residuais a serem
oferecidos, depois que o indivíduo já recebeu atendimento primário e prévios,
ou seja, aquele que já recebeu atendimento primário com consequente diagnóstico
e tratamento.

Amaral lembra, ainda, que o atendimento domiciliar pode acelerar a recuperação do 
paciente, e promover a redução de custos hospitalares. Além de ser uma solução mais 
humanista para os portadores de doenças crônicas ou de longa duração, frente à 
hospitalização.

Com estudos e casos de sucesso, percebeu-se que a assistência domiciliar à saúde  Portanto, a ADS
surge como uma alternativa viável a internação hospitalar clássica.



\subsection{Urgência e Emergência}\label{subsec:urgencia-emergencia}
\lipsum[1]

\subsection{Atenção domiciliar}\label{subsec:atencao-domiciliar}

\lipsum[1]

\section{Aspectos Tecnológicos}\label{sec:aspectos-tecnologicos}
