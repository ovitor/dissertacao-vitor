\chapter{Fundamentação Teórica}\label{cap:fundamentacao-teorica}

Este capítulo apresenta a fundamentação teórica, separada em aspectos de saúde e
tecnológicos. As seções de saúde abordam termos como Assistência Domiciliar à
Saúde e urgência e emergência, além de explanar, brevemente, sobre a história da
hospitalização no Brasil.

\section{Aspectos de Saúde}\label{sec:aspectos-de-saude}

Segundo Inaiá Mello, no livro ``Humanização nos Hospitais do Brasil''
\cite{inaia2008}, uma das primeiras instituições voltadas para o cuidado com a
saúde, foi a fundação da Santa  Casa de Misericórdia de Santos, em 1543, cuja
principal atividade era prestar assistência de cunho caritativo a pessoas pobres
e desabrigados. Tal estrutura permanece inalterada até o final do século XIX,
e início do século XX.

Com o Governo Getúlio Vargas, a partir de 1930, ocorre no Brasil o processo de
industrialização, que trouxe crescimento rápido e desordenado das cidades,
principalmente São Paulo e Rio de Janeiro. As transformações econômicas e
sociais resultantes desse processo, a falta de saneamento básico, a pobreza etc
foram motivos para que parte da população reivindicasse mais atenção do Governo
em relação aos cuidados de saúde \cite{carvalho1984}.

Carvalho indica, ainda, que apesar dessa pressão, não existiu uma política de
saúde clara por parte das autoridades. Muitas vezes, algumas ações se voltavam
para a criação de  condições sanitárias mínimas, que se mostravam limitadas
frente às reais necessidades da população. Dessa forma, as décadas subsequentes
não foram significativas no tocante a uma ampliação dos serviços de saúde
oferecidos à população.

Conforme análise de Inaiá Mello, ainda nos anos 1970, surgem tentativas de
universalizar o acesso à assistência à saúde. Alguns Programas e  Sistemas foram
iniciados, sendo válido citar, (i) Sistema Único e Descentralizado  de Saúde, o
SUDS e (ii) o Sistema Único de Saúde, o SUS.  Entretanto, em virtude da vigência
da ditadura militar, implantada em 1964,  que levou o país a vivenciar um estado
de exceção, tais propostas não conseguiram  se concretizar. Assim, somente no
período da redemocratização, ocorrida em meados dos anos 1980,  é que o SUS foi
criado oficialmente pela Constituição Federal de 1988, Lei 8080/90
\footnote{Acessível em
\url{http://www.planalto.gov.br/ccivil_03/leis/L8080.htm}}, com o objetivo de
garantir à população Brasileira, o acesso universal às ações e serviços  de
saúde.

Paralelo à criação do Sistema Único de Saúde (SUS), ocorreu o avanço tecnológico
que alcançou a prática médica, aperfeiçoando, com isso a infraestrutura
hospitalar.  Dessa forma, os hospitais deixaram de ser espaços para abrigarem
pobres desamparados  e passaram a proporcionar tratamentos mais elaborados. O
hospital passa a oferecer  procedimentos cirúrgicos, atendimentos de urgência,
internações, tornando a instituição dispendiosa. Os estudiosos começam a
identificar a possibilidade de tratamentos e cuidados com a  saúde que não
estejam, necessariamente, vinculados ao ambiente hospitalar.

Como consequência, surgiram diversas mudanças no atendimento, onde a Assistência
Domiciliar à Saúde (ADS) se tornou uma modalidade disponível.

\subsection{Assistência Domiciliar à Saúde}\label{subsec:assistencia-domiciliar-a-saude}

A Assistência Domiciliar à Saúde (ADS) divide-se basicamente em grupos de
enfermagem e fisioterapia - nas modalidades mais básicas - até um atendimento
multiprofissional, possibilitando um apoio ao paciente como um todo. A ADS pode
ser provida tanto pelo setor privado quanto pelo setor público
\cite{amaral2001assistencia}.

Os primeiros registros da ADS no Brasil surgem em 1967, na cidade de São
Paulo, no Hospital do Servidor Público. O principal objetivo dessa abordagem
era a liberação de leitos no hospital, levando para o domicílio procedimentos
básicos, de baixa complexidade clínica.

Já na década de 90, segundo Tavolari, houve um aumento considerável na
quantidade de empresas privadas provendo o serviço de ADS, apenas 5 empresas
prestavam esse tipo de  serviço, já em 1999, esse número subiu para mais de 180
\cite{tavolari2000desenvolvimento}.

Amaral et al define a ADS como uma sequência de serviços residuais a serem
oferecidos, depois que o indivíduo já recebeu atendimento primário e prévios,
ou seja, aquele que já recebeu atendimento primário com consequente diagnóstico
e tratamento.

Amaral e Tavolari lembram, ainda, que o atendimento domiciliar pode acelerar a
recuperação do  paciente e promover a redução de custos hospitalares, além de
ser uma solução mais  humanista para os portadores de doenças crônicas ou de
longa duração, frente à  hospitalização. Dessa forma, a asistência domiciliar à
saúde, tem como objetivos principais: (1) humanização no atendimento; (2) maior
rapidez na recuperação do paciente, devido à proximidade com os seus familiares;
(3) diminuição do risco de infecção hospitalar; (4) Otimização de leitos
hospitalares para pacientes que deles necessitem; e (5) Redução do custo/dia da
internação.

\subsubsection{Nomenclatura}\label{subsubsec:nomenclatura}

Não há 

\subsection{Urgência e Emergência}\label{subsec:urgencia-emergencia}
\lipsum[1]

\subsection{Atenção domiciliar}\label{subsec:atencao-domiciliar}

\lipsum[1]

\section{Aspectos Tecnológicos}\label{sec:aspectos-tecnologicos}
