\chapter{Conclusões e trabalhos futuros} \label{cap:conclusao}
% ---

Neste capítulo são apresentadas as conclusões e considerações finais acerca da
arquitetura apresentada. Além disso, descrevemos o que achamos ser importante
como trabalhos futuros.

\section{Conclusão} \label{sec:conclusao}

A proposta descrita nesta dissertação teve como ponto de partida o projeto de
pesquisa e desenvolvimento do qual o autor participou ativamente, chamado
Mecanismo de Comunicação entre Concessionárias e Clientes baseada na TV Digital
– METAL (processo: PD-0039-0062-2012), fomentado pela Agência Nacional de
Energia Elétrica (ANEEL) em parceria com a ENEL (antiga Companhia Energética do
Ceará - COELCE) e executado pela empresa cearense CRAFF. O projeto destinava-se
a pesquisa e desenvolvimento de uma plataforma em nuvem para gerenciamento e um
\stb[] com aplicações interativas para atender ao público alvo da fornecedora de
energia elétrica no Estado do Ceará. Como resultado, a empresa disponibilizou o
sistema na nuvem, aplicações interativas através do STB que permitiam o cliente
ter acesso aos serviços da companhia através da sua televisão.

Contribuiu para a elaboração da arquitetura proposta o também projeto de
pesquisa \nextsaude[] (processo: 6424611/2014). Este projeto, que o autor também
participou, se propôs a utilizar modernas tecnologias computacionais na tomada
de decisão pelos diversos atores envolvidos (do usuário ao gestor) em sistemas
público de saúde com ênfase na interoperabilidade sintática e integração 
semântica de dados.

Tomando como base a assertiva que um dos grandes meios de entretenimento e
informação da população brasileira é a televisão e o advento do Sistema
Brasileiro de Televisão Digital com a interatividade inerente ao sistema, foi
pensado que a televisão pudesse ser utilizada também, como um meio de interação
com o usuário. Para fornecer mais funcionalidades foi necessário a aplicação do
STB com poder de processamento e conexão à Internet disponbilizados na casa do
usuário.   

Como prova de conceito, portanto, o projeto teve a seguinte estratégia inicial:
aplicar o avanço da tecnologia da informação no ambiente de atenção domiciliar
à saúde, auxiliando os atores envolvidos (idoso/doente, cuidador, equipe de
saúde etc). À utilização do STB, ao desenvolvimento e a implantação de avisos e
alertas através da tela da televisão, as aplicações interativas com cunho
social e aplicado à saúde deram o nome de TV-Health, já a plataforma na nuvem
recebeu o nome do projeto - plataforma \nextsaude, estes foram os entregáveis do
projeto. Além disso, a coordenação do projeto disponibilizou a plataforma
Nextsaúde e todos os seus subsistemas para que secretarias de saúde dos
municípios pudessem implantar o sistema em suas gestões.

Durante a execução do projeto, percebeu-se a necessidade de um ambiente de
atenção domiciliar à saúde de qualidade, a importância do cuidador no auxílio
ao idoso/doente além da qualidade de vida que é necessária para terceira idade
ou doentes crônicos. Além disso, através das pesquisas, ficou claro a
importância da tecnologia da informação para prover uma solução mais impactante
para o usuário. 

A percepção de instrumentalização dos atores (cuidador, equipe de saúde, idoso
ou doente) através da tecnologia também surgiu nesse período. Aplicações
móveis, interface web e acompanhamento do doente através da Internet foram
tópicos importantes para o projeto.

Com o término do projeto \nextsaude[] e a entrega da primeira versão do TV Health,
percebeu-se a partir da pesquisa do estado da arte de tecnologias emergentes
como a Internet das Coisas revolucionaria o ambiente de atenção domiciliar como
nós o conhecemos. Através de ``coisas'' conectadas dotadas de sensores,
poderíamos melhorar a experiência de usuário. Seria possível, então, entregar
um serviço bem mais transparente para o usuário final e estaria alinhado com as
novas soluções propostas ao redor do mundo.

Neste trabalho foi descrito, portanto, uma arquitetura para atenção domiciliar
à saúde, que faz uso de tecnologias já conhecidas, tais como um sistema
sensível a contexto e sistemas embarcados, como também, Internet das Coisas.
Auxiliar doente, cuidador, parentes e equipe de saúde através do STB e
dispositivos móveis foi um dos objetivos deste trabalho. 

A arquitetura considera a utilização do STB como um hub de comunicação. Todas
as ``coisas'', ou seja, sensores coletando sinais vitais do paciente, sensores
coletando informações do ambiente e dispositivos móveis contribuem para um
sistema repleto de informação sobre o ator principal (idoso ou doente). Além
disso a arquitetura contempla a tomada de decisão, tanto local, quanto na
nuvem, através da utilização de ontologias e aprendizagem de máquinas.

%\section{Contribuições}\label{sec:resultados-praticos}

%Além da implementação das ideias discutidas durant

\section{Trabalhos futuros} \label{sec:trabalhos-futuros}

Uma vez definida a arquitetura e as características gerais que um sistema de
atenção domiciliar à saúde atual deve ter, podemos propor a inclusão de
novas funcionalidades a arquitetura apresentada, tais como novos métodos
de aquisição de dados, com a ajuda de processamento de linguagem natural.

A necessidade de gerenciar corretamente todos os dispositivos que a Internet
das Coisas proporciona cria uma linha de pesquisa em que é necessário aprimorar
a comunicação e gerência de IoT e Computação em Nuvem.

Para finalizar, pode-se realizar estudos relacionados ao tratamento em domicílio de
doenças específicas - tal como o Parkinson, portadores de pressão alta,
diabetes etc - englobando o uso de aprendizagem de máquinas com o intuito de detectar
antecipadamente o surgimento de uma doença específica.
