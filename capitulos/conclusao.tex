\chapter{Conclusões e trabalhos futuros} \label{cap:conclusao}
% ---

Historicamente, a TV Health das Coisas nasceu do METAL (Mecanismo de
Comunicação entre Concessionárias e Clientes baseada na TV Digital), um projeto
de pesquisa e desenvolvimento no setor elétrico fomentado pela Agência Nacional
de Energia Elétrica (ANEEL) em parceria com a ENEL (antiga Companhia Energética
do Ceará - COELCE), tendo como base o Sistema Brasileiro da TV Digital (SBTVD). 

Em seguida, a ideia do projeto METAL evoluiu para o TV Health, um projeto que
conservava a mesma estrutura de \hardware[] do METAL, baseada na TV Digital,
mas tinha como foco a área de ADS. A principal motivação era, além da presença
quase universal da TV nas residências do país, a política federal vigente que prometia
dotar os STBs de aplicações interativas com interesses visivelmente sociais,
em especial nas áreas de educação e saúde. Na verdade, esta foi a razão do Brasil
ter decidido desenvolver um modelo próprio de TV Digital em 2003.

O TV Health foi desenvolvido no contexto do \nextsaude, um projeto financiado
pela Fundação Cearense de Apoio ao Desenvolvimento Científico e Tecnológico
no Ceará (FUNCAP). Até 2015, esperava-se que este módulo do \nextsaude[]
atendesse a demanda do Governo Federal que acabou sendo adiada e continua até
hoje indefinida: a expectativa do uso de aplicações interativas em toda a rede
de beneficiários do Bolsa Família. Nesse sentido, várias reuniões foram feitas
pelo autor deste trabalho e seu orientador na PUC Rio, mais precisamente no
laboratório Telemídia, autor do \middleware[] Ginga, um resultado do SBTVD que
tornou-se o padrão ITU-T H.761.

Diante deste impasse, relativo ao destino do SBTVD, deu-se início ao TV Health
das Coisas, novo projeto que, se de um lado mantinha a TV Digital e o seu associado
STB, por outro lado o foco da pesquisa passa a ser o ambiente de ADS. Assim,
toda a experiência acumulada pelo autor deste trabalho nos projetos METAL e
\nextsaude[] foi canalizada na concepção da TV Health das Coisas que, como o próprio
título caracteriza, faz uso da tecnologia \textit{Internet of Things} (IoT).

Foi, portanto, este o processo percorrido pelo TV Health das Coisas, uma plataforma
que disponibiliza uma arquitetura para o ambiente de ADS e implementa um protótipo 
como prova de conceito da proposta. Para tanto, buscou-se referência no OpenIoT,
uma plataforma da comunidade Europeia. Uma das razões dessa escolha deve-se a
uma série características conceituais do OpenIoT que vão ao encontro do Lariisa,
um projeto do qual se originou o \nextsaude[] e, por consequência, a TV Health
das Coisas. Por exemplo, tanto o OpenIoT quanto o Lariisa fazem uso de ontologias
e são fortemente orientados a contexto (\textit{context-aware concept}). 

Além da correlação arquitetural da TV Health das Coisas e o OpenIoT, a proposta
desse trabalho apresenta-se como uma plataforma mais leve. A disponibilidade de
APIs facilita o desenvolvimento de novas aplicações, e sua integração a plataforma
\nextsaude[] agrega novas funcionalidades específicas da área de saúde. Na TV
Health das Coisas é viável o acesso a uma série de serviços de saúde, tais como 
o Registro Eletrônico de Saúde (RES), regulação (marcação de consultas, leitos etc.),
farmácia (acesso a remédios), entre outros.

Finalmente, caso haja uma decisão do Governo Federal em retomar a política em
relação ao programa Bolsa Família, a TV Health das Coisas poderá trazer
benefícios a milhões de brasileiros, em especial das classes menos favorecidas,
como preconizado no decreto 4901 que instituiu o SBTVD. Neste ínterim, a TV
Health das Coisas continua como uma plataforma de pesquisa e desenvolvimento
tendo sido aprovado pelo polo de Inovação Embrapii/IFCE e está sendo cogitada
sua implantação em um plano de saúde com abrangência nacional.

\section{Produção científica}\label{sec:producao}  

Durante este projeto de mestrado, os seguintes trabalhos científicos foram
aceitos e publicados, a saber:

\begin{itemize}
  \item \textbf{LOPES, V. C. M.}; QUEIROZ, E.; FREITAS, N.; OLIVEIRA, M.; MONTEIRO, O. TV-Health:
  A Context-Aware health Care Application for Brazilian Digital TV. In ACM.
  \textit{Proceedings of the 22nd Brazilian Symposium on Multimedia and the Web (Webmedia)}. 
  Teresina, Brasil. 2016, pp. 103-106.

  \item \textbf{LOPES, V. C. M.}; ROCHA, E.; QUEIROZ, E.; FREITAS, N.; VIANA, D.; OLIVEIRA, M. VITESSE 
  - more intelligence with emerging technologies for health systems. In: IEEE. 
  \textit{2016 7th International Conference on the Network of the Future (NOF)}. Buzios, Brasil. 2016, pp. 1-3.
\end{itemize}

Além do seguinte artigo aceito para publicação:

\begin{itemize}
  \item \textbf{LOPES, V. C. M.}; MOTA, H.; OLIVEIRA, M.; CARVALHO, G. Towards an Emergency/Urgency
    approach based on the Brazilian Digital TV. \textit{Multi Conference on Computer Science 
    and Information Systems (MCCSIS)}. Ilha da Madeira, Portugal, 2016.
\end{itemize}

\section{Trabalhos futuros} \label{sec:trabalhos-futuros}

A TV Health das Coisas pode também ser visto como uma perspectiva computacional do
trabalho realizado por \citeauthor{santos2014}, que apresentou a visão de saúde
dos cenários de ADS. 

Como trabalhos futuros, cogita-se a total integração dos serviços do \nextsaude[]
ao protótipo desenvolvido, aprimoramento de APIs e a inclusão de novos métodos de 
aquisição de dados.

Acredita-se que a arquitetura proposta na TV Health das Coisas fornece uma base
sólida e moderna para futuramente ajudar na concepção de um modelo de
referência que facilite o desenvolvimento de aplicações para ambientes
específicos de ADS. 

