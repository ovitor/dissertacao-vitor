% Copyright 2004 by Till Tantau <tantau@users.sourceforge.net>.
%
% In principle, this file can be redistributed and/or modified under
% the terms of the GNU Public License, version 2.
%
% However, this file is supposed to be a template to be modified
% for your own needs. For this reason, if you use this file as a
% template and not specifically distribute it as part of a another
% package/program, I grant the extra permission to freely copy and
% modify this file as you see fit and even to delete this copyright
% notice. 

\documentclass{beamer}

% There are many different themes available for Beamer. A comprehensive
% list with examples is given here:
% http://deic.uab.es/~iblanes/beamer_gallery/index_by_theme.html
% You can uncomment the themes below if you would like to use a different
% one:
\usepackage[brazil]{babel}
\usepackage[utf8]{inputenc}
\usepackage[T1]{fontenc}
\usepackage[alf]{abntex2cite}
\usepackage{xcolor}
\usepackage{amsmath}    % Pacote para multilines equations
\usepackage{multirow}
\usepackage{scalefnt} 
\usepackage{tabularx}
\usepackage{subcaption}
\usepackage{booktabs}
\newcommand{\tabspace}{\hspace{2em}}
\usepackage{etoolbox,lipsum}
\usepackage[version=4]{mhchem}
\usepackage{collcell}
\usepackage{capt-of}% or use the larger `caption` package
\usepackage{siunitx}
\sisetup{output-exponent-marker=\ensuremath{\mathrm{e}}}
\usepackage{multirow}

\newcommand\infe{\textcolor{gray}{\ding{55}}}
\newcommand\supe{\textcolor{gray}{\ding{55}}}
\newcommand\equi{\ding{51}}

\usepackage{pifont}

\setbeamercolor{block title}{bg=blue!20,fg=black}
\setbeamercolor{block body}{bg=blue!10,fg=black}

\DeclareMathOperator*{\argmin}{min}
\DeclareMathOperator*{\argmax}{max}
\DeclareMathOperator{\sign}{sign}
\renewcommand{\vec}[1]{\boldsymbol{#1}}

\newcommand{\tabela}[3]{
\begin{table}[!htbp]
  \scalefont{#3}
  \begin{center}
  \caption{#2}\label{tab:#1}
  %\vspace{-1em}
    \input{tabelas/#1}
  %\vspace{-1em}
  \end{center}
\end{table}
}
\newcommand{\algoritmo}[2]{
\begin{algorithm}
\caption{#2}\label{alg:#1}
  \input{algoritmos/#1}
\end{algorithm}
}
\usepackage{algorithm}
\usepackage{algpseudocode}
\usepackage{tikz}
\makeatletter
\def\BState{\State\hskip-\ALG@thistlm}
\renewcommand{\ALG@name}{Algoritmo}
\renewcommand{\listalgorithmname}{Lista de algoritmos}
\makeatother

% Declaracoes em Português
\algrenewcommand\algorithmicend{\textbf{fim}}
\algrenewcommand\algorithmicdo{\textbf{faça}}
\algrenewcommand\algorithmicwhile{\textbf{enquanto}}
\algrenewcommand\algorithmicfor{\textbf{para}}
\algrenewcommand\algorithmicif{\textbf{se}}
\algrenewcommand\algorithmicthen{\textbf{então}}
\algrenewcommand\algorithmicelse{\textbf{senão}}
\algrenewcommand\algorithmicreturn{\textbf{devolve}}
\algrenewcommand\algorithmicfunction{\textbf{função}}



\algrenewcommand\algorithmicforall{\textbf{para todo}}
\algrenewcommand\algorithmicloop{\textbf{loop}}
\algrenewcommand\algorithmicrepeat{\textbf{repetir}}
\algrenewcommand\algorithmicuntil{\textbf{até}}
\algrenewcommand\algorithmicprocedure{\textbf{procedimento}}
\algrenewcommand\algorithmicrequire{\textbf{Exige:}}
\algrenewcommand\algorithmicensure{\textbf{Garante:}}
\newcommand{\Not}{\textbf{não }}





% Rearranja os finais de cada estrutura
%\algrenewtext{Procedure}{\textbf{Procedimento~} }
\algrenewtext{goto}{\textbf{vai para~} }
\algrenewtext{EndWhile}{\algorithmicend\ \algorithmicwhile}
\algrenewtext{EndFor}{\algorithmicend\ \algorithmicfor}
\algrenewtext{EndIf}{\algorithmicend\ \algorithmicif}
\algrenewtext{EndFunction}{\algorithmicend\ \algorithmicfunction}

% O comando For, a seguir, retorna 'para #1 -- #2 até #3 faça'
\algnewcommand\algorithmicto{\textbf{até}}
\algrenewtext{For}[3]%
{\algorithmicfor\ #1 $\gets$ #2 \algorithmicto\ #3 \algorithmicdo}
% \setitemize{label=\usebeamerfont*{itemize item}%
%   \usebeamercolor[fg]{itemize item}
%   \usebeamertemplate{itemize item}}

\graphicspath{ {figuras/graficos/} {figuras/imagens/}} 

\setbeamertemplate{caption}[numbered]



%\usetheme{AnnArbor}
%\usetheme{Antibes}
%\usetheme{Bergen}
%\usetheme{Berkeley}
%\usetheme{Berlin}
%\usetheme{Boadilla}
%\usetheme{boxes}
\usetheme{CambridgeUS}
%\usetheme{Copenhagen}
%\usetheme{Darmstadt}
%\usetheme{default}
%\usetheme{Frankfurt}
%\usetheme{Goettingen}
%\usetheme{Hannover}
%\usetheme{Ilmenau}
%\usetheme{JuanLesPins}
%\usetheme{Luebeck}
% \usetheme{Madrid}
%\usetheme{Malmoe}
%\usetheme{Marburg}
%\usetheme{Montpellier}
%\usetheme{PaloAlto}
%\usetheme{Pittsburgh}
% \usetheme{Rochester}
%\usetheme{Singapore}
%\usetheme{Szeged}
%\usetheme{Warsaw}
\usecolortheme{seahorse}
\title[Qualificação]{Título do Trabalho}

% A subtitle is optional and this may be deleted
%\subtitle{A Dual Approach}

\author[Vitor C. M. Lopes]{\textbf{Aluno:} Vitor de Carvalho Melo Lopes~~~~~~~~~~\;\\\textbf{Orientador:} Prof. Dr. Antônio Mauro Barbosa de Oliveira}
% - Give the names in the same order as the appear in the paper.
% - Use the \inst{?} command only if the authors have different
%   affiliation.

\institute[PPGCC/IFCE] % (optional, but mostly needed)
{%
  Programa de Pós-Graduação em Ciência da Computação (PPGCC)\\
  Instituto Federal de Educação, Ciência e Tecnologia do Ceará (IFCE)  
}
% - Use the \inst command only if there are several affiliations.
% - Keep it simple, no one is interested in your street address.

\date{\today}
% - Either use conference name or its abbreviation.
% - Not really informative to the audience, more for people (including
%   yourself) who are reading the slides online

\subject{Theoretical Computer Science}
% This is only inserted into the PDF information catalog. Can be left
% out. 

% If you have a file called "university-logo-filename.xxx", where xxx
% is a graphic format that can be processed by latex or pdflatex,
% resp., then you can add a logo as follows:

% \pgfdeclareimage[height=0.5cm]{university-logo}{university-logo-filename}
% \logo{\pgfuseimage{university-logo}}

% Delete this, if you do not want the table of contents to pop up at
% the beginning of each subsection:
% \AtBeginSection[]
% {
%   \frame{\tableofcontents[currentsection]\frametitle{Agenda}}
% }

\robustify{\textit}
\robustify{\color}

% \AtBeginSection[]{
%     \begin{frame}
%     \vfill
%     \centering
%     \begin{beamercolorbox}[sep=8pt,center,shadow=true,rounded=true]{title}
%         \usebeamerfont{title}\insertsectionhead\par%
%     \end{beamercolorbox}
%     \vfill
%     \end{frame}
% }

\setbeamertemplate{navigation symbols}{} % Remove os itens inferiores do slides
% Let's get started
\begin{document}

\begin{frame}
  \titlepage
\end{frame}

\begin{frame}{Agenda}
  \tableofcontents[hideothersubsections]
  % You might wish to add the option [pausesections]
\end{frame}

% Section and subsections will appear in the presentation overview
% and table of contents.

\section{Introdução}

% slide  
\subsection{Linhas gerais}
\begin{frame}{Introdução}
  \begin{itemize}
    \item \Large Envelhecimento da população e as abordagens para o tratamento
    clínico;
  \end{itemize}
\end{frame}

% slide  
\subsection{Motivação}
\begin{frame}{Motivação}
  \begin{itemize}
    \item \Large Alto número de idosos;
    \item \Large Proporcionar soluções de atenção à sociedade;
  \end{itemize}
\end{frame}

% slide 
\subsection{Descrição do problema}
\begin{frame}{Descrição do problema}
  \begin{itemize}
    \item Descrição do problema;
  \end{itemize}
\end{frame}

% slide  
\section{Objetivos Geral e Específicos}

\begin{frame}{Objetivos Geral e Específicos}
\begin{block}{Objetivo Geral}
  \begin{itemize}
    \item Oferecer uma solução que auxilie, através de sistemas de 
    informática, o cuidador, o doente e à equipe médica no tratamento do 
    paciente em domicílio.
  \end{itemize}
\end{block}
\vspace{-.5em}

\begin{block}{Objetivos Específicos}
  \begin{itemize}
    \item Permitir um acompanhamento constante da equipe médica ao paciente;
    \item Facilitar as atividades diárias do cuidador em relação aos 
    procedimentos destinados ao doente;
  \end{itemize}
\end{block}

\end{frame}

\section{Fundamentação Teórica}

% slide 
\subsection{Aspectos de Saúde}
\begin{frame}{Aspectos de Saúde}
  \begin{itemize}
    \item Hospitalização no Brasil 
    % \vspace{0.5em}
    \item Assistência Domiciliar à Saúde:
      \begin{itemize}
        \item Terminologia 
        \item Envolvidos;
      \end{itemize}
  \end{itemize}
\end{frame}

% slide
\subsection{Aspectos de Saúde}
\begin{frame}{Terminologia ilustrada}
  \begin{figure}[H]
    \begin{center}  
    \caption{(a) Representação gráfica das categorias defendida por Tavolari,
    Fernandes e Medina; (b) Representação gráfica das categorias defendida
    por Giacomozzi.}
      \begin{subfigure}[b]{.45\textwidth}
        % \resizebox{6cm}{5cm}{
         \begin{tikzpicture}[font=\sffamily]
         \begin{scope}[shift={(0cm,0cm)}, fill opacity=0.8]
             \draw (0,0) ellipse (3.5cm and 2.6cm);
             \draw (-1.7,-0.6) ellipse (1.2cm and 1cm);
             \draw (1.5,-0.6) ellipse (1.5cm and 1cm);

             \node at (0,1.4) [align=center]{Assistência Domiciliar\\ à Saúde};
             \node at (-1.7,-0.6) [align=center]{\tiny Internação\\ \tiny Domiciliar};
             \node at (1.5,-0.6)[align=center]{\tiny Atenção Domiciliar\\ \tiny à Saúde};
             \end{scope}
         \end{tikzpicture} 
         % } % close resizebox
        \caption{}
        \label{fig:ads-categoria-tavoloari}
      \end{subfigure}
      ~
      \begin{subfigure}[b]{.45\textwidth}
        % \resizebox{6cm}{5cm}{
        \begin{tikzpicture}[font=\sffamily]
        \begin{scope}[shift={(0cm,0cm)}, fill opacity=0.8]
            \draw (0,0) ellipse (3.5cm and 2.6cm); % atencao domiciliar
            \draw (-1.7,-0.6) ellipse (1.2cm and 1cm); % internacao domiciliar
            \draw (1.5,-0.1) ellipse (1.5cm and 1cm); % assistencia domiciliar
            \draw (0.1, -1.7) circle (0.8cm); % visita domiciliar

            \node at (0,1.4) [align=center]{Atenção Domiciliar\\ à Saúde};
            \node at (-1.7,-0.6) [align=center]{\tiny Internação\\ \tiny Domiciliar};
            \node at (1.5,-0.1)[align=center]{\tiny Assistência\\ \tiny Domiciliar\\ \tiny à Saúde};
            \node at (0.1,-1.7)[align=center]{\tiny Visita\\ \tiny Domiciliar};
            \end{scope}
        \end{tikzpicture}
        % } % close resizebox
        \caption{}
        \label{fig:ads-categoria-giacamozzi}
      \end{subfigure}
      \par
      Fonte: Próprio autor.
    \end{center}
  \end{figure}
\end{frame}

% slide
% \subsection{Aspectos Tecnológicos}

% \begin{frame}{SVM de margem flexível}
% \begin{figure}[H]
%     \begin{center}  
        % %\includegraphics[width=.45\textwidth]{figuras/imagens/conjunto-de-dados-overlap.pdf}
%     \end{center}
%     \caption{Conjunto de dados com sobreposição entre classes. Exemplos positivos são apresentados como discos na cor preta e negativos na cor cinza.}
% \end{figure}
% \end{frame}

% \begin{frame}{Treinamento}
  % \begin{itemize}
    % \item Formulação \textit{primal} (margem flexível):
    % \vspace{-0.5em}
    % \begin{eqnarray}
             % \argmin_{  \vec{w},  w_0, \vec{\xi}} \left \{ \frac{1}{2} \| \vec{w} \| + C \sum_{i=1}^{l} \xi_i \right \} , &                    \label{eq-svm-primal}     \\ 
      % \text{sujeito a~~~~~} y_i (\vec{w}^T\vec{x}_i - w_0) \geq 1 - \xi_i, & \forall i.         \label{eq-svm-restricao-primal}
    % \end{eqnarray}
    % \begin{itemize}
        % \item em que $\vec{\xi} = \{\xi_i \geq 0 \}_{i=1}^{l}$ são variáveis de folga e $C$ controla o compromisso entre erros de classificação e a complexidade do modelo.
      % \end{itemize}
  % \end{itemize}
% \end{frame}

% \begin{frame}{Treinamento}
  % \begin{itemize}
    % \item Formulação \textit{dual} (margem flexível):
    % \begin{eqnarray} \label{eq:problema_dual_com_varivel_folga}
         % \argmax_{\vec{\alpha}}  \left\{\sum_{i=1}^l \alpha_i - \frac{1}{2} \sum_{i=1}^l \sum_{j=1}^l \alpha_i \alpha_j y_i y_j \vec{x}_i^T\vec{x}_j \right\}, & \\ \label{eq:main_constraint} \text{sujeito a~~~~~}  \sum_{i=1}^l \alpha_i y_i = 0,  & \\
       % \label{eq:second_constraint}  0 \leq \alpha_i \leq C, & \forall i.
      % \end{eqnarray} 
      % \begin{itemize}
        % \item onde $\vec{\alpha} = \{\alpha_i\}_{i=1}^{l}$ são multiplicadores de Lagrange.  
      % \end{itemize}
  % \end{itemize}
% \end{frame}

% \begin{frame}{Utilização (versão \textit{dual})}
%   \begin{itemize}
%     \item Os valores de $\vec{w}^\star$ e $w^\star_0$ podem ser computados através de
%     \begin{eqnarray}
%     \vec{w}^\star &=& \sum_{i=1}^{l}\alpha^\star_i y_i \vec{x}_i, ~~~~~~~\text{e}\\  
%     w_0^\star     &=& 1 - \vec{x}_i^T\vec{w}^\star, ~~~\forall i.
%     \end{eqnarray}
%     \item Função discriminante:
%     \begin{equation}\label{eq:svm-regra-decisao-2}
%     f(\vec{x}) =   \sum_{i=1}^{l}\alpha^\star_i y_i \vec{x}_i^T\vec{x} - w_0^\star
%     \end{equation}
%   \end{itemize}
% \end{frame}

\section{Metaheurísticas}

% \subsection{Introdução}

% \begin{frame}{Metaheurísticas}
%   \begin{itemize}
%     \item Subcampo primário de otimização estocástica;
%     \vspace{0.5em}
%     \item Aplicação em problemas de diferentes tipos;
%     \vspace{0.5em}
%     % \item Tipos de metaheurísticas:
%     % \begin{itemize}
%     %   \item Metaheurísticas construtivas;
%     %   \item Metaheurísticas de relaxamento;
%     %   \item Metaheurísticas de busca;
%     %   \item Metaheurísticas evolutivas.
%     % \end{itemize}
%     % \vspace{0.5em}
%     \item Balanço dinâmico entre diversificação e intensificação.
%   \end{itemize}
% \end{frame}


% \subsection{Recozimento Simulado}

% \begin{frame}{Recozimento Simulado}
%   \begin{itemize}
%     \item Recozimento (\textit{Annealing})
%     \begin{itemize}
%       \item Resfriamento rápido conduz a produtos de maior energia interna
%       \item Resfriamento lento conduz a produtos de menor energia
%     \end{itemize}
%     \item Probabilidade de mudança de $E_i$ para $E_j$
%       \begin{equation}
%       P(E_j) = \exp \left ( - \frac{E_j - E_i}{k_B T} \right ),
%       \end{equation}
%       em que $k_B$ é uma constante de Boltzmann e $T$ é a temperatura
%     \item \citeonline{metropolis1953} propôs melhorias no processo de \textit{Annealing}
    
%     \end{itemize}
% \end{frame}

% \begin{frame}{Conceitos iniciais}
% \begin{itemize}
%     \item Inicialmente proposto por~\citeonline{kirkpatrick1983};
%     \vspace{0.5em}
%     \item Proposto separadamente por~\citeonline{vcerny1985};
%     \vspace{0.5em}
%     \item Recozimento Simulado~(\textit{Simulated Annealing}, SA):
%     \begin{itemize}
%       \item Metaheurística de otimização de exploração local;
%       \item Baseado no processo de recozimento (\textit{annealing});
%       \item Reduz a probabilidade de ficar preso em mínimos locais.
%     \end{itemize}
%     \vspace{0.5em}
    
%     % \tabela{sa-equivalencias}{Analogias entre \textit{annealing} e um problema de otimização.}{.7}
    
%   \end{itemize}
% \end{frame}

\begin{frame}{Implementação típica}
  \begin{figure}[H]
    \begin{center}  
      % %\includegraphics[width=0.73\textwidth]{figuras/imagens/sa-flowchart.pdf}
    \end{center}
    \caption{Fluxograma canônico de um SA.}
  \end{figure}
\end{frame}


% \begin{frame}{Algoritmo canônico}
%   \scriptsize
%   \input{algoritmos/sa}
% \end{frame}

\begin{frame}[noframenumbering]{Implementação típica}
  \begin{itemize}
    \item Representação da solução;
    \vspace{0.5em}
    \item Parâmetros de configuração do SA:
    \begin{itemize}
      \item Função de energia $\textsc{E}(\boldsymbol{\theta})$;
      \item Função de agendamento de temperaturas $\textsc{W}(T)$;
      \item Temperatura inicial $T_0$ e final $T_f$; e
      \item Estrutura de vizinhança.
    \end{itemize}
    \vspace{0.5em}
    \item Soluções infactíveis:
    \begin{itemize}
      \item Penalização;
      \item Eliminação;
      \item Reparo das soluções; e
      %\item Decodificadores de soluções; e
      \item Métodos híbridos.
    \end{itemize}
  \end{itemize}
\end{frame}


\subsection{Algoritmos Genéticos}

\begin{frame}{Conceitos iniciais}
  \begin{itemize}
    \item Método computacional de busca probabilística;
    \vspace{0.5em}
    \item Proposto por~\citeonline{holland1975} e popularizado por~\citeonline{goldberg1989};
    \vspace{0.5em}
    \item Inspirado nos mecanismos de evolução natural e na genética;
    \vspace{0.5em}
    \item Metaheurística populacional.
    \vspace{0.5em}
    % \item Termos utilizados:
    %   \begin{itemize}
    %     \item Gene;
    %     \item Cromossomo;
    %     \item Indivíduo;
    %     \item Aptidão (\textit{fitness}).
    %   \end{itemize}
  \end{itemize}
\end{frame}

\begin{frame}{Implementação típica}
  \begin{figure}[H]
    \begin{center}  
      % %\includegraphics[width=0.73\textwidth]{figuras/imagens/ga-flowchart.pdf}
    \end{center}
    \caption{Fluxograma canônico de um GA.}
  \end{figure}
\end{frame}

\begin{frame}[noframenumbering]{Implementação típica}
  \begin{itemize}
    \item Componentes necessários:
      \begin{itemize}
        \item Representação genética do problema; 
        \item Função de aptidão;
        \item Mecanismo de seleção; 
        \item Operadores de mutação e reprodução e
        \item Critério de parada.
      \end{itemize}
  \end{itemize}
\end{frame}

\section{Métodos de Treinamento de SVMs}

\subsection{Programação Quadrática}

\begin{frame}{Métodos de Treinamento de SVMs}
  
  \begin{itemize}
    \item Programação quadrática(QP)
    \begin{itemize}
      \item Métodos matemáticos clássicos~\cite{nocedal2006};
      \item Ponto interior, conjunto ativo, Lagrangiano aumentado, etc.
      \item Requer o armazenamento de uma matriz de dimensões $l \times l$.
    \end{itemize}
    \vspace{0.5em}
    \item \textit{Sequential Minimal Optimization} (SMO)
    \begin{itemize}
      \item Método iterativo, proposto por~\citeonline{platt1998};
      \item Particiona o problema de otimização nos menores subproblemas;
      \item Menor subproblema possível envolve dois multiplicadores;
      \item Número excessivo de combinações: $l(l-1)$.
    \end{itemize}
    
  \end{itemize}
\end{frame}


\begin{frame}{Métodos de Treinamento de SVMs}
  
  \begin{itemize}
    \item \textit{Kernel} Adatron (KA)
    \begin{itemize}
      \item Método de aprendizado online, proposto por~\citeonline{anlauf1989}
      % \item Baseado no \textit{Adaptive Perceptron Algorithm} (Adatron);
      \item Topologia similar a arquitetura de uma rede neural;
      \item Utiliza apenas informação de primeira ordem da função custo;
      % \item Possui convergência garantida em relação à solução ótima.
      \item A restrição apresentada na Equação~(\ref{eq:main_constraint}) não é respeitada.
    \end{itemize}
    \vspace{0.5em}
    \item \textit{Linear Particle Swarm Optimization} (LPSO)
    \begin{itemize}
      \item Baseado em no LPSO, proposto por~\citeonline{paquet2003a};
      \item Decompoe o problema de otimização em um subconjunto de trabalho;
      \item Repete a otimização até que o valor da função objetivo não mude;
      \item A restrição apresentada na Equação~(\ref{eq:main_constraint}) não é respeitada.
    \end{itemize}
    
  \end{itemize}
\end{frame}

% \begin{frame}{Programação quadrática  (QP)}
%   \begin{itemize}
%     \item Métodos matemáticos clássicos~\cite{nocedal2006};
%     \vspace{0.5em}
%     \item Principais métodos:
%     \begin{itemize}
%       \item Método do ponto interior;
%       \item Método de conjunto ativo;
%       \item Lagrangiano aumentado;
%       \item Gradiente conjugado; e
%       \item Extensões do algoritmo simplex.
%     \end{itemize}
%     \vspace{0.5em}
%     % \item Projetados para aproveitar da parte quadrática da função objetivo
%     % \vspace{0.5em}
%     \item Requer o armazenamento de uma matriz de dimensões $l \times l$
%   \end{itemize}
% \end{frame}

% \subsection{\textit{Sequential Minimal Optimization}}

% \begin{frame}{\textit{Sequential Minimal Optimization} (SMO)}
%   \begin{itemize}
%     \item Proposto por~\citeonline{platt1998};
%     \vspace{0.5em}
%     \item Método iterativo;
%     \vspace{0.5em}
%     \item Particiona o problema de otimização nos menores subproblemas;
%     \vspace{0.5em}
%     \item Menor subproblema possível envolve dois multiplicadores;
%     \vspace{0.5em}
%     \item Número excessivo de combinações $l(l-1)$.
% %     \begin{equation}
% % \begin{split}
% %  \Omega  = &  + \frac{1}{2}K_{pp}\alpha_{p}^2 + \frac{1}{2}K_{qq}\alpha_{q}^2 + y_p y_q K_{pq}\alpha_{p} \alpha_{q}  \\
% %                 &  + y_p \alpha_p v_p + y_q \alpha_q v_q - \alpha_p - \alpha_q   \\
% %                 &  + \underbrace{\sum_{i \neq p,q} \alpha_i - \frac{1}{2} \sum_{i \neq p,q} \sum_{j \neq p,q} \alpha_i \alpha_j y_i y_j K_{ij}}_{\Omega_{\text{Const.}}}
% % \end{split} 
% % \end{equation}sujeito a
% % \vspace{-2em}
% % \begin{eqnarray}
% % \label{eq:const1} y_p \alpha_p + y_q \alpha_q  = \text{Const.} & & \\
% % \label{eq:const2} \alpha_p,\alpha_q \in [0, C] & & 
% % \end{eqnarray}
% %   em que $v_j \equiv \sum_{i \neq p,q} y_i \alpha_i K_{ij} = u_j + w_0 - y_p \alpha_p K_{pj} - y_q \alpha_q K_{qj}$
%   \end{itemize}
% \end{frame}

% \subsection{\textit{Kernel} Adatron}

% \begin{frame}{\textit{Kernel} Adatron (KA)}
%   \begin{itemize}
%     \item Baseado no \textit{Adaptive Perceptron Algorithm} (Adatron)
%     \vspace{0.5em}
%     \item Proposto por~\citeonline{anlauf1989}
%     \vspace{0.5em}
%     \item Características:
%     \begin{itemize}
%       \item Método de aprendizado online;
%       \item Utiliza apenas informação de primeira ordem da função custo;
%       \item Possui convergência garantida em relação à solução ótima; 
%       \item Topologia similar a arquitetura de uma rede neural.
%     \end{itemize}
%     % \item A restrição~(\ref{eq:main_constraint}) não é respeitada.
%   \end{itemize}
% \end{frame}


% \subsection{Otimização Linear por Enxame de Partículas}

% \begin{frame}{Otimização Linear por Enxame de Partículas (LPSO)}
%   \begin{itemize}
%     \item Soluciona o problema de otimização quadrática
%     \vspace{0.5em}
%     \item Baseado em no LPSO~\cite{paquet2003b}
%     \vspace{0.5em}
%     \item Proposto por~\citeonline{paquet2003a}
%     \vspace{0.5em}
%     \item Funcionamento:
%     \begin{itemize}
%       \item Decompoe o problema de otimização em um subconjunto de trabalho
%       \item Repete a otimização até que o valor da função objetivo não mude
%     \end{itemize}
%     \vspace{0.5em}
%     \item A restrição~(\ref{eq:main_constraint}) não é respeitada.
%   \end{itemize}
% \end{frame}

\subsection{Proposta 1: \textit{Genetic Algorithms for Training Easily}}
\setbeamertemplate{enumerate items}[default]
\begin{frame}{Proposta 1: \textit{Genetic Algorithms for Training Easily}}
  \begin{itemize}
    \item \textit{Genetic Algorithms for Training Easily} (GATE) SVMs
    \item Algoritmos Genéticos mono objetivo
    \item Soluciona o problema de otimização quadrática
    \item As entradas do GATE são o conjunto de dados de treinamento e o parâmetro de regularização $C$ (Similarmente ao QP, SMO, KA)
  \end{itemize}
  \vspace{-0.8em}
  \begin{block}{Algoritmo GATE}
    \begin{enumerate}
      \item Gerar \textbf{população inicial} de soluções aleatóriamente
      \item Aplicar o \textbf{algoritmo de ajuste} (se necessário)
      \item Repetir até que o critério de parada for alcançado
      \begin{enumerate}[i.]
        \item Selecionar os melhores indivíduos através de \textbf{elitismo}
        \item Aplicar os operadores de \textbf{recombinação} e \textbf{mutação}
        \item Se necessário, aplicar o algoritmo de ajuste nos indivíduos mutados
      \end{enumerate}
      \item Computar o \textbf{viés} a partir dos multiplicadores de Lagrange
    \end{enumerate}
  \end{block}
\end{frame}

\begin{frame}{Proposta 1: \textit{Genetic Algorithms for Training Easily}}
  \begin{itemize}
    \item Representação genética: direta
    \item População inicial: gerada aleatoriamente
    \item Função de aptidão: rearranjo da função objetivo
    \item Mecanismo de seleção: roleta padrão
    \item Operador de recombinação: reproddução aritmética
    \item Operador de mutação: \textit{swap}
    \item Critério de parada: número de gerações e monitor de estagnação
    \item Cálculo do viés: $w_0 = \frac{1}{l}\sum_{i=1}^{l}\left(y_{i} + \sum_{j=1}^{l}y_{j}\alpha_{j}\mathbf{x}_{i}^{T}\mathbf{x}_{j} \right )$
    \item Violação das restrições:
    \begin{itemize}
      \item População inicial e Operador de mutação
      \item \textbf{algoritmo de ajuste}: redução iterativa dos valores dos multiplicadores
    \end{itemize}
    
  \end{itemize}
\end{frame}

% \begin{frame}{Representação genética e População Inicial}
% \begin{block}{Representação genética} %OK
%   \begin{itemize}
%     \item Vetor de genes reais, em que cada gene representa um valor de um multiplicador de Lagrange relacionado a certo padrão;
%     \vspace{0.5em}
%     %\item Therefore, each individual has as many support vectors as the number of genes not equal to ``zero''.
%     \item Com um conjunto de treinamento com $l$ padrões, um indivíduo $\vec{\alpha}$ também terá $l$ genes, tal qual o indivíduo
%     \vspace{0.5em}
%     \begin{equation}
%     \vec{\alpha} = [ \alpha_1~~\alpha_2~\ldots~\alpha_i~\ldots~\alpha_l ].
%   \end{equation}
%   \vspace{0.1em}
%   \end{itemize}
% \end{block}
% \vspace{-0.5em}
% \begin{block}{População Inicial}
%   \begin{itemize}
%     \item Genes são gerados aleatóriamente entre no intervalo $[0,C]$;
%     \vspace{0.5em}
%     \item Indivíduos infactíveis são reparados por um \textbf{algoritmo de ajuste}. 
%   \end{itemize}
% \end{block}
% \end{frame}

% \begin{frame}{Violação das restrições e aplicação do Algoritmo de ajuste}
% \begin{block}{Violação das restrições} %OK
%   \begin{itemize}
%     \item Restrições:
%     \vspace{-0.5em}
%     \begin{align*}
%     0 \leq \alpha_i \leq C \\ 
%     \sum_{i=1}^l \alpha_i y_i =  0
%     \end{align*}
%     \vspace{-0.6em}
%     \item Os indivíduos podem violar a segunda restrição depois da
%     \begin{itemize}
%       \item Geração da população inicial; e
%       \item Aplicação do operador de mutação.
%     \end{itemize}
%   \end{itemize}
% \end{block}
% \vspace{-0.5em}
% \begin{block}{Aplicação do Algoritmo de ajuste}
% \begin{itemize}
%     \item Alternativa comum em metaheurísticas~\cite{michalewicz1995a};
%     \item Redução iterativa dos valores dos multiplicadores de Lagrange.
% \end{itemize}
% \end{block}
% \end{frame}

\begin{frame}{Algoritmo de ajuste}
\begin{figure}[!htbp]
  \begin{center}
    % %\includegraphics[width=0.8\textwidth]{figuras/imagens/Creation.pdf}
    \caption{Exemplo de violação da restrição~(\ref{eq:main_constraint}) para um indivíduo gerado aleatoriamente e seu reparo através da função de ajuste.}\label{fig:creation-ga}
\end{center}
\end{figure}
\end{frame}

% \begin{frame}[noframenumbering]{Algoritmo de ajuste}
%   \scriptsize
%   \input{algoritmos/funcao-ajuste}
% \end{frame}

% \begin{frame}{Função de aptidão e Mecanismo de seleção}
% \begin{block}{Função de aptidão} %OK
%   \begin{itemize}
%     \item A função de aptidão é uma forma rearranjada da Equação~\ref{eq:problema_dual_com_varivel_folga}, já que nossa proposta se apresenta em termos de minimização
%   \end{itemize}
% \vspace{-.5em}
%   \begin{equation}
%   \mathrm{F}(\boldsymbol{\alpha}) = - \sum_{i=1}^l \alpha_i + \frac{1}{2} \sum_{i=1}^l \sum_{j=1}^l \alpha_i \alpha_j y_i y_j \mathbf{x}_i^T \mathbf{x}_j.
%   \end{equation}
% \end{block} 
% \begin{block}{Mecanismo de seleção} %OK
%   \begin{itemize}
%     \item Seleção por roleta padrão;
%     % \item Proposta por~\citeonline{dejong1975};
%     \item Escolhe indivíduos de acordo com o valor de uma função objetivo baseada no valor do \textit{fitness}.
%   \end{itemize}
% \vspace{-.5em}
% \end{block} 
% \end{frame}

% \begin{frame}{Operadores de recombinação e mutação}
%   \begin{block}{Operador de recombinação}
%     \begin{itemize}
%       \item Recombinação aritmética cria descendentes que são uma média ponderada dos dois progenitores;
%       %\item Offsprings are feasible with respect to linear constraints and bounds presented in Eq.~(\ref{eq:main_constraint}) and Eq.~(\ref{eq:second_constraint}), respectively.
%       \item O parâmetro $\gamma$ é um valor escolhido aleatoriamente no intervalo $[0,1]$
%       %\item Let $g_{1i}$ and $g_{2i}$ be the $i$-th gene of the first parent and the second one, respectively; so the operator generates children as follows.
%     \end{itemize}
%   \vspace{-1em}
%   \begin{eqnarray}
%   \label{eq:ga-aritimetico-a} \vec{p} = \gamma \cdot \vec{x} + (1 - \gamma) \cdot \vec{y}, & &  \\
%   \label{eq:ga-aritimetico-b} \vec{q} = \gamma \cdot \vec{y} + (1 - \gamma) \cdot \vec{x}. & &
%   \end{eqnarray}
%   \end{block}

%   \begin{block}{Operador de mutação}
%   \begin{itemize}
%     \item Operador de mutação foi definido como o \textit{swap};
%     \item Seleciona-se aleatoriamente dois genes e seus valores são permutados;
%     \item Este operador pode violar a restrição presente na equação~(\ref{eq:main_constraint});
%     \item Usa o algoritmo de ajuste para reparar soluções infactíveis.
%   \end{itemize}
%   \end{block}
% \end{frame}


% \begin{frame}{Critério de parada e Cálculo do viés}
% \begin{block}{Critério de parada}
% \begin{itemize}
% \item O algoritmo para caso uma das duas situações ocorra:
%   \begin{enumerate}[i.]
%   \item O número limite de gerações é atingido; ou 
%   \item O \textit{fitness} se mantem igual para determinado número de gerações.
%   \end{enumerate}
% \end{itemize}
% \end{block}
% \begin{block}{Cálculo do viés}
% \begin{itemize}
% \item Após a obtenção dos multiplicadores de Lagrange através do GA, o viés é calculado da seguinte forma:
% \begin{equation}
% w_0 = \frac{1}{l}\sum_{i=1}^{l}\left(y_{i} + \sum_{j=1}^{l}y_{j}\alpha_{j}\mathbf{x}_{i}^{T}\mathbf{x}_{j} \right )
% \end{equation}
% \vspace{0.2em}
% \end{itemize}
% \end{block}
% \end{frame}



\subsection{Proposta 2: \textit{Simulated Annealing for Training Easily}}

\begin{frame}{Proposta 2: \textit{Simulated Annealing for Training Easily}}
  \begin{itemize}
    \item \textit{Simulated Annealing for Training Easily} (SATE) SVMs
    \item Baseado no método de Recozimento Simulado
    \item Soluciona o problema de otimização quadrática
    \item As entradas do SATE são o conjunto de dados de treinamento e o parâmetro de regularização $C$ (Similarmente ao QP, SMO, KA)
    \item Requisitos de um algoritmo baseado em SA:
      \begin{itemize}
        \item Representação da solução e solução inicial;
        \item Função de energia; e
        \item Função de vizinhança.
      \end{itemize}
    \item Cálculo do viés (mesma forma que o GATE).
  \end{itemize}
\end{frame}

\begin{frame}{Proposta 2: \textit{Simulated Annealing for Training Easily}}
  \begin{itemize}
    \item Representação da solução: direta;
    \item Solução inicial: apenas dois multiplicadores;
    \item Função de energia: rearranjo da função objetivo;
    \item Função de vizinhança: 
    \begin{itemize}
      \item Baseada em três funções auxiliares: troca, adição e modificação;
      \item Prioriza soluções esparsas.
    \end{itemize}
    \item Cálculo do viés: similar ao GATE
    \item Violação das restrições:
    \begin{itemize}
      \item Função de troca
      \item \textbf{algoritmo de ajuste}: similar ao GATE
    \end{itemize}
  \end{itemize}
\end{frame}


% \begin{frame}{Representação da solução e s'olução inicial}
% \begin{block}{Representação da solução} %OK
%   \begin{itemize}
%     \item Vetor de números reais, em que cada posição representa um valor de um multiplicador de Lagrange relacionado a certo padrão;
%     \vspace{0.5em}
%     %\item Therefore, each individual has as many support vectors as the number of genes not equal to ``zero''.
%     \item Com um conjunto de treinamento com $l$ padrões, uma solução $\vec{\alpha}$ também terá $l$ posições
%     \vspace{-.5em}
%     \begin{equation}
%     \vec{\alpha} = [ \alpha_1~~\alpha_2~\ldots~\alpha_i~\ldots~\alpha_l ].
%   \end{equation}
%   \end{itemize}
% \end{block}
% \vspace{-0.5em}
% \begin{block}{Solução inicial}
%   \begin{itemize}
%     \item Tem apenas dois multiplicadores de Lagrange diferentes de zero;
%     \item Cada classe possui um vetor suporte;
%     \item Valores escolhidos de forma aleatória no intervalo $[0,C]$;
%     \item Todas as restrições são respeitadas. 
%   \end{itemize}
% \end{block}
% \end{frame}

% \begin{frame}{Função de energia e Função de vizinhança}
% \begin{block}{Função de energia} %OK
%   \begin{itemize}
%     \item A função de energia é uma forma rearranjada da Equação~\ref{eq:problema_dual_com_varivel_folga}, já que nossa proposta se apresenta em termos de minimização
%   \end{itemize}
% \vspace{-.5em}
%   \begin{equation}
%   \mathrm{E}(\boldsymbol{\alpha}) = - \sum_{i=1}^l \alpha_i + \frac{1}{2} \sum_{i=1}^l \sum_{j=1}^l \alpha_i \alpha_j y_i y_j \mathbf{x}_i^T \mathbf{x}_j.
%   \end{equation}
% \end{block} 
% \begin{block}{Função de vizinhança} %OK
%   \begin{itemize}
%     \item Provê soluções novas e factíveis
%     \item Baseada em três funções auxiliares:
%     \begin{itemize}
%       \item Função de troca;
%       \item Função de adição;
%       \item Função de modificação.
%     \end{itemize}
%     \item Prioriza soluções esparsas.
%   \end{itemize}
% \vspace{-.5em}
% \end{block} 
% \end{frame}

% \begin{frame}{Função de vizinhança}
%   \scriptsize
%   \input{algoritmos/funcao-vizinhanca}
% \end{frame}

% \begin{frame}{Função de modificação e Função de adição}
%   \begin{block}{Função de modificação}
%     \begin{itemize}
%       \item Dois valores inteiros $i$ and $j$ são gerados aleatoriamente no intervalo entre $1$ e $l$ tal que $i \neq j$ e $y_{i} = y_{j}$. 
%       \item Um valor real $\gamma$ no intervalo $[0,1]$ é gerado aleatoriamente para realizar a combinação linear entre $\alpha_i^{mdf}$ e $\alpha_j^{mdf}$

%       %\textsc{Modify}$\left(\boldsymbol{\alpha}\right)$, \textsc{Add}$\left(\boldsymbol{\alpha}\right)$ and \textsc{Swap}$\left(\boldsymbol{\alpha}\right)$. The entire neighborhood function is presented below.
%       %\vspace{-1em}
%     \end{itemize}
%     \vspace{-1em}
%     \begin{eqnarray}\label{eq:sa_modify}
%       \alpha_{i}^{mdf} = \gamma \cdot \alpha_{i}  + (1 - \gamma) \cdot \alpha_{j}, & & \\
%       \alpha_{j}^{mdf} = \gamma \cdot \alpha_{j}  + (1 - \gamma) \cdot \alpha_{i}.  & &
%     \end{eqnarray} 
%   \end{block}
%   \begin{block}{Função de adição}
%     \begin{itemize}
%       \item Duas posições são selecionadas aleatoriamente ($i$ and $j$) pertencentes a diferentes classes e com valores diferentes de zero
%       \item Um valor real aleatorio $\tau \in [0,C]$ é escolhido e atribuido a $\alpha_i$ e $\alpha_j$
%     \end{itemize}
%   \end{block}
% \end{frame}


% \begin{frame}{Função de troca e Cálculo do viés}
% \begin{block}{Função de troca}
% \begin{itemize}
% \item Semelhante ao operador de mutação \textit{swap}.
% \vspace{0.5em}
% \item Este procedimento pode violar a restrição $$\sum_{i=1}^l \alpha_i y_i =  0,$$ apresentada na Equação~(\ref{eq:main_constraint}).
% \vspace{0.5em}
% \item Reparação realizada através do \textbf{algoritmo de ajuste}.
% \end{itemize}
% \end{block}

% \begin{block}{Cálculo do viés}
% \begin{itemize}
% \item Semelhante ao cálculo realizado no GATE.
% \end{itemize}
% \end{block}
% \end{frame}


\section{Simulações Computacionais}

\subsection{Metodologia}
\begin{frame}{Experimento 1: prova de conceito}
  \begin{itemize}
    \item Problema Artificial I (PAI)
      \begin{itemize}
        \item Problema binário;
        \item 100 instâncias em cada classe;
        \item Linearmente separável.
      \end{itemize}
    \item Problema Artificial II (PAII)
      \begin{itemize}
        \item Problema binário;
        \item 400 instâncias em cada classe;
        \item Não linearmente separável.
      \end{itemize}
  \end{itemize}
\end{frame}


\begin{frame}{Experimentos 2 e 3: conjuntos de dados reais}
  \begin{itemize}
    \item Conjuntos de dados binários (tabelas~\ref{tab:conjuntos-de-dados} e ~\ref{tab:mnist});
    \item Normalização de média zero e desvio padrão igual a um;
    \item Avaliação feita através do método \textit{holdout} com divisão $80\%-20\%$;
    \item Hiperparâmetros obtidos através de validação cruzada:
      \begin{itemize}
        \item $k$-\textit{fold}, com $k = 5$;
        \item Variação de $C$ no intervalo $\{2^{-5}, \cdots, 2^{15}\}$.
      \end{itemize}
    \item 30 execuções pareadas;
    \item Avaliação de equivalência realizada pelo teste de Friedman.
  \end{itemize}
\end{frame}


% \begin{frame}{Experimento 1: prova de conceito}
%   \begin{figure}[H]
%     \begin{center}  
%       \begin{subfigure}[b]{.45\textwidth}
        % %\includegraphics[width=\textwidth]{figuras/graficos/artificial/pai/base.pdf}
%         \caption{PAI}
%         \label{fig:simulacoes-pa-i-base}
%       \end{subfigure}
%       ~
%       \begin{subfigure}[b]{.45\textwidth}
        % %\includegraphics[width=\textwidth]{figuras/graficos/artificial/paii/base.pdf}
%         \caption{PAII}
%         \label{fig:simulacoes-pa-ii-base}
%       \end{subfigure}

%     \end{center}
%     \caption{Conjuntos de dados artificiais, utilizados como prova de conceito.}\label{fig:simulacoes-pa}
% \end{figure}
% \end{frame}


% \begin{frame}{Experimento 2: conjuntos de dados reais}
%   \begin{itemize}
%     \item Experimentos com conjuntos de dados reais;
%     \item Experimentos com o conjunto de dados MNIST;
%     \item Variação do hiperparâmetro $C$;
%     \item Variação do Tamanho do conjunto de dados.
%   \end{itemize}
% \end{frame}


% \begin{frame}{Experimento 2: conjuntos de dados reais}
%   \begin{itemize}
%     \item Conjuntos de dados binários (Tabela~\ref{tab:conjuntos-de-dados});
%     \item Normalização de média zero e desvio padrão igual a um;
%     \item Avaliação feita através do método \textit{holdout} com divisão $80\%-20\%$;
%     \item Hiperparâmetros obtidos através de validação cruzada:
%       \begin{itemize}
%         \item $k$-\textit{fold}, com $k = 5$;
%         \item Variação de $C$ no intervalo $\{2^{-5}, \cdots, 2^{15}\}$.
%       \end{itemize}
%     \item 30 execuções pareadas;
%     \item Avaliação de equivalência realizada pelo teste de Friedman.
%   \end{itemize}
% \end{frame}

\begin{frame}[noframenumbering]{Experimento 2: conjuntos de dados reais\footnote[frame]{O método GATE não foi avaliado para os conjuntos de dados RIP e BKM.}}
  % \tabela{conjuntos-de-dados}{Conjuntos de dados utilizados no trabalho.}{.8}
\end{frame}

% \begin{frame}{Experimento 3: conjuntos de dados MNIST}
%   \begin{itemize}
%     \item Conjuntos de dados binários com $784$ ($28 \times 28$) atributos (Tabela~\ref{tab:mnist});
%     \item Normalização de média zero e desvio padrão igual a um;
%     \item Avaliação feita através do método \textit{holdout} com divisão $80\%-20\%$;
%     \item Hiperparâmetros obtidos através de validação cruzada:
%       \begin{itemize}
%         \item $k$-\textit{fold}, com $k = 5$;
%         \item Variação de $C$ no intervalo $\{2^{-5}, \cdots, 2^{15}\}$.
%       \end{itemize}
%     \item 30 execuções pareadas;
%     \item Avaliação de equivalência realizada pelo teste de Friedman.
%   \end{itemize}
% \end{frame}

\begin{frame}{Experimento 3: conjuntos de dados MNIST\footnote[frame]{O método GATE não foi avaliado para o conjuntos de dados MNIST.}}
  % \tabela{mnist}{Divisão de dados da base MNIST.}{.8}
\end{frame}

\begin{frame}{Experimento 4 e 5: }
  \begin{itemize}
    \item Conjuntos de dados HAB e BCW;
    \item Experimento 4: variação da divisão treino/teste;
    \item Experimento 5: variação do hiperparâmetro $C$.
    % \item Normalização de média zero e desvio padrão igual a um;
    % \item Avaliação feita através do método \textit{holdout} com divisão $80\%-20\%$;
    % \item Hiperparâmetros obtidos através de validação cruzada:
    %   \begin{itemize}
    %     \item $k$-\textit{fold}, com $k = 5$;
    %     \item Variação de $C$ no intervalo $\{2^{-5}, \cdots, 2^{15}\}$.
    %   \end{itemize}
    % \item 30 execuções pareadas.
  \end{itemize}
\end{frame}

% \begin{frame}{Experimento 5: variação do hiperparâmetro $C$}
%   \begin{itemize}
%     \item Conjuntos de dados HAB e BCW;
%     \item Normalização de média zero e desvio padrão igual a um;
%     \item Avaliação feita através do método \textit{holdout} com divisão $80\%-20\%$;
%     \item 30 execuções pareadas;
%     \item Variação do hiperperâmetro $C$ no invervalo $\{2^{-5}, \cdots, 2^{9}\}$.
%   \end{itemize}
% \end{frame}

% \begin{frame}{Parâmetros específicos de cada método de treinamento}
% \tabela{parametros-1}{Parâmetros específicos de cada método de treinamento I.}{.8}
% \end{frame}

% \begin{frame}{Parâmetros específicos de cada método de treinamento}
% \tabela{parametros-2}{Parâmetros específicos de cada método de treinamento II.}{.8}
% \end{frame}

\subsection{Resultados e Discussões}

\begin{frame}{Experimento 1: prova de conceito}
\begin{figure}[H]
    \begin{center}  
      \begin{subfigure}[b]{.40\textwidth}
        % %\includegraphics[width=\textwidth]{figuras/graficos/artificial/pai/sv-qp.pdf}
        \caption{QP}
      \end{subfigure}
      ~
      \begin{subfigure}[b]{.40\textwidth}
        % %\includegraphics[width=\textwidth]{figuras/graficos/artificial/pai/sv-smo.pdf}
        \caption{SMO}
      \end{subfigure}
      
    \end{center}
    \caption{Superfícies de decisão e vetores suporte geradas pelos métodos de treinamento a partir de problema artificial I.}\label{fig:simulacoes-pa-i-resultados}
\end{figure}
\end{frame}

\begin{frame}[noframenumbering]{Experimento 1: prova de conceito}
\begin{figure}[H]
    \begin{center}  
      \begin{subfigure}[b]{.40\textwidth}
        % %\includegraphics[width=\textwidth]{figuras/graficos/artificial/pai/sv-ka.pdf}
        \caption{KA}
      \end{subfigure}
      ~
      \begin{subfigure}[b]{.40\textwidth}
        % %\includegraphics[width=\textwidth]{figuras/graficos/artificial/pai/sv-lpso.pdf}
        \caption{LPSO}
      \end{subfigure}
      
    \end{center}
    \caption{Superfícies de decisão e vetores suporte geradas pelos métodos de treinamento a partir de problema artificial I.}\label{fig:simulacoes-pa-i-resultados}
\end{figure}
\end{frame}





\begin{frame}[noframenumbering]{Experimento 1: prova de conceito}
\begin{figure}[H]
    \begin{center}  
      \begin{subfigure}[b]{.40\textwidth}
        % %\includegraphics[width=\textwidth]{figuras/graficos/artificial/pai/sv-gate.pdf}
        \caption{GATE}
      \end{subfigure}
      ~
      \begin{subfigure}[b]{.40\textwidth}
        % %\includegraphics[width=\textwidth]{figuras/graficos/artificial/pai/sv-sate.pdf}
        \caption{SATE}
      \end{subfigure}
      
    \end{center}
    \caption{Superfícies de decisão e vetores suporte geradas pelos métodos de treinamento a partir de problema artificial I.}\label{fig:simulacoes-pa-i-resultados}
\end{figure}
\end{frame}

\begin{frame}[noframenumbering]{Experimento 1: prova de conceito}
\begin{figure}[H]
    \begin{center}  
      \begin{subfigure}[b]{.40\textwidth}
        % %\includegraphics[width=\textwidth]{figuras/graficos/artificial/paii/sv-qp.pdf}
        \caption{QP}
      \end{subfigure}
      ~
      \begin{subfigure}[b]{.40\textwidth}
        % %\includegraphics[width=\textwidth]{figuras/graficos/artificial/paii/sv-smo.pdf}
        \caption{SMO}
      \end{subfigure}
      
    \end{center}
    \caption{Superfícies de decisão e vetores suporte geradas pelos métodos de treinamento a partir de problema artificial II.}\label{fig:simulacoes-pa-i-resultados}
\end{figure}
\end{frame}

\begin{frame}[noframenumbering]{Experimento 1: prova de conceito}
\begin{figure}[H]
    \begin{center}  
      \begin{subfigure}[b]{.40\textwidth}
        % %\includegraphics[width=\textwidth]{figuras/graficos/artificial/paii/sv-ka.pdf}
        \caption{KA}
      \end{subfigure}
      ~
      \begin{subfigure}[b]{.40\textwidth}
        % %\includegraphics[width=\textwidth]{figuras/graficos/artificial/paii/sv-lpso.pdf}
        \caption{LPSO}
      \end{subfigure}
      
    \end{center}
    \caption{Superfícies de decisão e vetores suporte geradas pelos métodos de treinamento a partir de problema artificial II.}\label{fig:simulacoes-pa-i-resultados}
\end{figure}
\end{frame}


\begin{frame}[noframenumbering]{Experimento 1: prova de conceito}
\begin{figure}[H]
    \begin{center}  
      \begin{subfigure}[b]{.40\textwidth}
        % %\includegraphics[width=\textwidth]{figuras/graficos/artificial/paii/sv-gate.pdf}
        \caption{GATE}
      \end{subfigure}
      ~
      \begin{subfigure}[b]{.40\textwidth}
        % %\includegraphics[width=\textwidth]{figuras/graficos/artificial/paii/sv-sate.pdf}
        \caption{SATE}
      \end{subfigure}
      
    \end{center}
    \caption{Superfícies de decisão e vetores suporte geradas pelos métodos de treinamento a partir de problema artificial II.}\label{fig:simulacoes-pa-i-resultados}
\end{figure}
\end{frame}


\begin{frame}[noframenumbering]{Experimento 1: prova de conceito}

  % \tabela{resultados-pa}{Síntese dos resultados para os conjuntos de dados PAI e PAII.}{0.8}
\end{frame}

\begin{frame}{Experimento 2: conjuntos de dados reais}
  % \tabela{resultados-1a}{Métricas de desempenho -- taxa de acerto (ACC) e número de vetores suporte (\#~VS) -- para os métodos SATE, GATE, LPSO, SMO, QP e KA; e resultados de testes estatísticos (T. E.) em relação aos métodos SATE e GATE. Os símbolos \equi~e~\infe~significam equivalência ou não equivalência, respectivamente.}{0.6}
\end{frame}

\begin{frame}[noframenumbering]{Experimento 2: conjuntos de dados reais}
  % \tabela{resultados-1b}{Métricas de desempenho -- taxa de acerto (ACC) e número de vetores suporte (\#~VS) -- para os métodos SATE, GATE, LPSO, SMO, QP e KA; e resultados de testes estatísticos (T. E.) em relação aos métodos SATE e GATE. Os símbolos \equi~e~\infe~significam equivalência ou não equivalência, respectivamente.}{0.6}
\end{frame}

\begin{frame}[noframenumbering]{Experimento 2: conjuntos de dados reais}
  \begin{figure}[!htbp]
    \centering
    % %\includegraphics[width=0.75\textwidth]{figuras/graficos/new/res_1.pdf}
    \caption{Taxa de acerto (\%) e \# Vetores Suporte (\%) do SATE, GATE, LPSO, SMO, QP e KA para os conjuntos de dados AUS, BCW, HAB e HEA.}\label{fig:accuracy_training_patterns}
  \end{figure}
\end{frame}

\begin{frame}[noframenumbering]{Experimento 2: conjuntos de dados reais}
  \begin{figure}[!htbp]
    \centering
    % %\includegraphics[width=0.75\textwidth]{figuras/graficos/new/res_2.pdf}
    \caption{Taxa de acerto (\%) e \# Vetores Suporte (\%) do SATE, GATE, LPSO, SMO, QP e KA para os conjuntos de dados PID, SON e VCP.}\label{fig:accuracy_training_patterns}
  \end{figure}
\end{frame}

\begin{frame}[noframenumbering]{Experimento 2: conjuntos de dados reais}
  % \tabela{resultados-1c}{Métricas de desempenho -- taxa de acerto (ACC) e número de vetores suporte (\#~VS) -- para os métodos SATE, GATE, LPSO, SMO, QP e KA; e resultados de testes estatísticos (T. E.) em relação aos métodos SATE e GATE. Os símbolos \equi~e~\infe~significam equivalência ou não equivalência, respectivamente.}{0.6}
\end{frame}

\begin{frame}[noframenumbering]{Experimento 2: conjuntos de dados reais}
  % \tabela{resultados-1d}{Métricas de desempenho -- taxa de acerto (ACC) e número de vetores suporte (\#~VS) -- para os métodos SATE, GATE, LPSO, SMO, QP e KA; e resultados de testes estatísticos (T. E.) em relação aos métodos SATE e GATE. Os símbolos \equi~e~\infe~significam equivalência ou não equivalência, respectivamente.}{0.6}
  \vspace{5em}
\end{frame}

\begin{frame}[noframenumbering]{Experimento 2: conjuntos de dados reais}
% \tabela{resultados-2}{Métricas de desempenho para os métodos SATE, LPSO, SMO, QP e KA; e resultados de testes estatísticos (T. E.) em relação ao método SATE. Os símbolos \equi~e~\infe~significam equivalência ou não equivalência, respectivamente.}{.6}
\end{frame}

\begin{frame}{Experimento 3: conjunto de dados MNIST}
  % \tabela{resultados-mnist-a}{Métricas de desempenho -- taxa de acerto (ACC) e número de vetores suporte (\#~VS) -- para os métodos SATE, GATE, LPSO, SMO, QP e KA; e resultados de testes estatísticos (T. E.) em relação aos métodos SATE e GATE. Os símbolos \equi~e~\infe~significam equivalência ou não equivalência, respectivamente.}{0.6}
\end{frame}

\begin{frame}[noframenumbering]{Experimento 3: conjunto de dados MNIST}
  % \tabela{resultados-mnist-b}{Métricas de desempenho -- taxa de acerto (ACC) e número de vetores suporte (\#~VS) -- para os métodos SATE, GATE, LPSO, SMO, QP e KA; e resultados de testes estatísticos (T. E.) em relação aos métodos SATE e GATE. Os símbolos \equi~e~\infe~significam equivalência ou não equivalência, respectivamente.}{0.6}
\end{frame}

\begin{frame}[noframenumbering]{Experimento 3: conjunto de dados MNIST}
  % \tabela{resultados-mnist-c}{Métricas de desempenho -- taxa de acerto (ACC) e número de vetores suporte (\#~VS) -- para os métodos SATE, GATE, LPSO, SMO, QP e KA; e resultados de testes estatísticos (T. E.) em relação aos métodos SATE e GATE. Os símbolos \equi~e~\infe~significam equivalência ou não equivalência, respectivamente.}{0.6}
\end{frame}





\begin{frame}{Experimento 4: variação da divisão treino/teste}
\begin{figure}[!htbp]
    \centering
    \begin{subfigure}[b]{0.48\textwidth}
        % %\includegraphics[width=\textwidth]{figuras/graficos/hab-acc-percentage.pdf}
        \caption{}
    \end{subfigure}
    \begin{subfigure}[b]{0.48\textwidth}
        % %\includegraphics[width=\textwidth]{figuras/graficos/hab-sv-percentage.pdf}
        \caption{}
    \end{subfigure}
    
    % \begin{subfigure}[b]{0.48\textwidth}
        % %\includegraphics[width=\textwidth]{figuras/graficos/vcp-acc-percentage.pdf}
    %     \caption{}
    % \end{subfigure}
    % \begin{subfigure}[b]{0.48\textwidth}
        % %\includegraphics[width=\textwidth]{figuras/graficos/vcp-sv-percentage.pdf}
    %     \caption{} 
    % \end{subfigure}
    
    \caption{Taxa de acerto (\# VS) do SATE, SMO, KA, QP e LPSO \textit{versus} porcentagem de treino, para o conjunto de dados HAB no lado esquerdo (direito).}\label{fig:accuracy_training_patterns}
\end{figure}
\end{frame}

\begin{frame}[noframenumbering]{Experimento 4: variação da divisão treino/teste}
\begin{figure}[!htbp]
    \centering
   
    \begin{subfigure}[b]{0.48\textwidth}
        %\includegraphics[width=\textwidth]{figuras/graficos/bcw-acc-percentage.pdf}
        \caption{}
    \end{subfigure}
    \begin{subfigure}[b]{0.48\textwidth}
        %\includegraphics[width=\textwidth]{figuras/graficos/bcw-sv-percentage.pdf}
        \caption{}
    \end{subfigure}
    
    
    % \begin{subfigure}[b]{0.48\textwidth}
        % %\includegraphics[width=\textwidth]{figuras/graficos/vcp-acc-percentage.pdf}
    %     \caption{}
    % \end{subfigure}
    % \begin{subfigure}[b]{0.48\textwidth}
        % %\includegraphics[width=\textwidth]{figuras/graficos/vcp-sv-percentage.pdf}
    %     \caption{} 
    % \end{subfigure}
    
    \caption{Taxa de acerto (\# VS) do SATE, SMO, KA, QP e LPSO \textit{versus} porcentagem de treino, para o conjunto de dados BCW no lado esquerdo (direito).}\label{fig:accuracy_training_patterns}
\end{figure}
\end{frame}





\begin{frame}{Experimento 5: variação do hiperparâmetro $C$}
\begin{figure}[!htbp]
    \centering
   
  \begin{subfigure}[b]{0.48\textwidth}
        %\includegraphics[width=\textwidth]{figuras/graficos/hab-acc-constraint.pdf}
        \caption{}
    \end{subfigure}
    \begin{subfigure}[b]{0.48\textwidth}
        %\includegraphics[width=\textwidth]{figuras/graficos/hab-sv-constraint.pdf}
        \caption{}
    \end{subfigure}
  
   
   % \vspace{1em}
   
    % \begin{subfigure}[b]{0.48\textwidth}
        % %\includegraphics[width=\textwidth]{figuras/graficos/vcp-acc-constraint.pdf}
    %     \caption{}
    % \end{subfigure}
    % \begin{subfigure}[b]{0.48\textwidth}
        % %\includegraphics[width=\textwidth]{figuras/graficos/vcp-sv-constraint.pdf}
    %     \caption{}
    % \end{subfigure}
    
    \caption{
    Taxa de acerto (\# VS) do SATE, SMO, KA e QP \textit{versus} o hiperparâmetros $C$ para o conjunto de dados HAB no lado esquerdo (direito).}\label{fig:accuracy_C1} 
\end{figure}
\end{frame}

\begin{frame}[noframenumbering]{Experimento 5: variação do hiperparâmetro $C$}
\begin{figure}[!htbp]
    \centering

    \begin{subfigure}[b]{0.48\textwidth}
        %\includegraphics[width=\textwidth]{figuras/graficos/bcw-acc-constraint.pdf}
        \caption{}
    \end{subfigure}
    \begin{subfigure}[b]{0.48\textwidth}
        %\includegraphics[width=\textwidth]{figuras/graficos/bcw-sv-constraint.pdf}
        \caption{}
    \end{subfigure}
   
   % \vspace{1em}
   
    % \begin{subfigure}[b]{0.48\textwidth}
        % %\includegraphics[width=\textwidth]{figuras/graficos/vcp-acc-constraint.pdf}
    %     \caption{}
    % \end{subfigure}
    % \begin{subfigure}[b]{0.48\textwidth}
        % %\includegraphics[width=\textwidth]{figuras/graficos/vcp-sv-constraint.pdf}
    %     \caption{}
    % \end{subfigure}
    
    \caption{Taxa de acerto (\# VS) do SATE, SMO, KA e QP \textit{versus} o hiperparâmetros $C$ para o conjunto de dados BCW no lado esquerdo (direito).}\label{fig:accuracy_C1} 
\end{figure}
\end{frame}


\section{Conclusões e Trabalhos futuros}

\begin{frame}{Conclusões e Trabalhos futuros}
  \begin{block}{Conclusões}
    \begin{itemize}
      \item  Desenvolvimento de duas abordagens de treinamento para o SVM;
      \item  Restrições do SVM foram incorporadas com êxito nos métodos;
      \item  Desenvolvimento de um algoritmo de ajuste que eleva a esparsidade;
      \item  GATE e SATE
      \begin{itemize}
        \item Equivalentes ao SMO em termos de \#VS e Taxa de acerto;
        \item Mais esparsos que o KA, o QP e o LPSO.
      \end{itemize}
    \end{itemize}
  \end{block}
  \begin{block}{Trabalhos futuros}
    \begin{itemize}
      \item  Geração de soluções iniciais mais próximas a solução ideal;
      \item  Paralelização de alguns componentes do método GATE;
      \item  Tornar os classificadores multiclasses;
      \item  Usar funções de \textit{kernel} não lineares.
    \end{itemize}
  \end{block}
\end{frame}

% \section{Produção científica}

% \begin{frame}{Produção Científica (aceitos)}
%   \begin{itemize}
%     \item \textbf{DIAS, M. L. D.}; ROCHA NETO, A. R. Evolutionary support vector machines: A dual approach. In: IEEE. \textit{2016 IEEE Congress on Evolutionary Computation (CEC)}. Vancouver, Canada, 2016. p. 2185--2192
%     \vspace{0.5em}
%     \item \textbf{DIAS, M. L. D.}; ROCHA NETO, A. R. A novel simulated annealing-based learning algorithm for training support vector machines. In: \textit{2016 International Conference on Intelligent Systems Design and Applications (ISDA)}. Porto, Portugal, 2016.
%     \vspace{0.5em}
%     \item DE SOUSA, L. S.; \textbf{DIAS, M. L. D.}; ROCHA NETO, A. R. Máquinas de vetores-suporte de mínimos quadrados esparsas via recozimento simulado. In: \textit{Simpósio Brasileiro de Automação Inteligente (SBAI)}. Rio Grande do Norte, Brasil, 2015
%   \end{itemize}
% \end{frame}

% \begin{frame}{Produção Científica (em revisão)}
%   \begin{itemize}
%     \item \textbf{DIAS, M. L. D.}; ROCHA NETO, A. R. Training soft margin support vector machines by simulated annealing: a dual approach. In: Elsevier; \textit{Expert Systems with Applications (ESWA)}.
%     \vspace{0.5em}
%     \item \textbf{DIAS, M. L. D.}; DE SOUSA, L. S.; ROCHA NETO, A. R.; FREIRE, A. L. Fixed-Size Extreme Learning Machines through Simulated Annealing. In: Springer; \textit{Neural Processing Letters (NEPL)}.
%     \vspace{0.5em}
%     \item ALVES, S. S. A.; \textbf{DIAS, M. L. D.}; ROCHA NETO, A. R. Genetic algorithms on solving the SVR quadratic optimization problem in its dual formulation. In: \textit{European Symposium on Artificial Neural Networks, Computational Intelligence and Machine Learning (ESANN)}. Bruges, Belgium, 2017.
%   \end{itemize}
% \end{frame}

\section{Referências}

\begin{frame}[allowframebreaks]{Referências}\scriptsize
    % \bibliographystyle{IEEEtran}
    \bibliography{references}
\end{frame}

% \section{Dúvidas}

% \begin{frame}{Dúvidas e Contato}

% \begin{center}
% {\huge Dúvidas?}
% \end{center}
% \vspace{3em}

% \begin{block}{E-mails}
% \begin{itemize}
%   \item Madson L. Dantas Dias (madson.dias@ppgcc.ifce.edu.br)
%   \item Ajalmar R. Rocha Neto (ajalmar@ifce.edu.br)
% \end{itemize}
% \end{block}
% \end{frame}

\end{document}