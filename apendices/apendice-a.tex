\chapter{Aspectos de Implementação}
% ---

% Um dos objetivos deste trabalho é proporcionar uma melhoria no dia-a-dia  dos
% usuários. Portanto, neste apêndice são descritos aspectos  de implementação da
% solução apresentada no decorrer do trabalho.

\section{\textit{Hardware} utilizado}\label{sec:hardware-utilizado}

O \stb[] (STB) escolhido possui especificações técnicas robustas, listadas na 
tabela \ref{tab:stb-specs}. Este STB foi resultado de um projeto de pesquisa
financiado pela Financiadora de Estudos e Projetos (FINEP) e Agência Nacional
de Energia Elétrica (ANEEL) e foi reutilizado neste trabalho por se tratar de 
uma pesquisa em que o autor participou.

\tabela{stb-specs}{Especificação STB}{0.9}

% Por se tratar de um sistema embarcado, seu poder de processamento, memória,
% alimentação externa e portas de entrada e saída 

\section{\textit{Software} utilizado}\label{sec:software-utilizado}

Por se tratar de um sistema embarcado robusto, faz-se necessário a instalação
de programas capazes de gerenciar o \hardware[] disponível. Para tanto, o
sistema operacional escolhido foi o GNU/Linux distribuição Linaro (derivado da
distribuição Ubuntu) \textit{kernel} 3.4 portável para a arquitetura 
ARM com \textit{drivers} específicos para a plataforma desenvolvida.

  \begin{figure}[H]
    \begin{center}  
    \caption{Camadas do sistema embarcado.}
      \begin{tikzpicture}[font=\sffamily]
      \begin{scope}[shift={(0cm,0cm)}, fill opacity=0.8]
        \node[draw,text width=4cm,text height=0.6cm] 
        at (0,0) {\textbf{Aplicações}};
        \node[draw,text width=4cm,text height=0.6cm, fill=mycolor4] 
        at (0,-1) {\textbf{Middleware}};
        \node[draw,text width=4cm,text height=0.6cm, fill=mycolor3] 
        at (0,-2) {\textbf{Sistema Operacional}};
        \node[draw,text width=4cm,text height=0.5cm, fill=mycolor2] 
        at (0,-3) {\textbf{Kernel/Drivers}};
        \node[draw,text width=4cm,text height=0.7cm, fill=mycolor1] 
        at (0,-4) {\textbf{Hardware}};
      \end{scope}
      \end{tikzpicture} 
      \label{fig:sistema-embarcado}
      \par
      Fonte: Próprio autor.
    \end{center}
  \end{figure}

Na camada de usuário utilizamos o \software[] \xbmc[] - um gerenciador de mídia
- como plataforma de exibição gráfica. É através dele que o usuário irá 
interagir com as aplicações desenvolvidas. Foram necessárias algumas 
modificações por conta dos objetivos específicos e do contexto do trabalho 
desenvolvido. Na figura \ref{fig:tela-principal} abaixo destaque para a tela 
inicial do \software[] \xbmc[] e para o menu ``Sua Saúde''. Os outros menus
representam funcionalidades padrão do \software[] em questão, tais como
acesso a mídias (imagens, vídeos e áudios) e acesso a TV Digital aberta.

Ao clicar no menu ``Sua Saúde'', acessamos a tela com os módulos desenvolvidos
para este trabalho. 

\figuradupla{tela-principal}{Tela principal do STB. Destaque para o menu
``Sua Saúde''}{tela-principal-sua-saude}{Tela principal do menu ``Sua Saúde''}

\section{\textit{Software} implementado}\label{sec:software-implementado}

Nossa solução foi projetada para atender aos requisitos que uma plataforma
embarcada com essas características demanda. Apesar das especificações do
\hardware[], o desenvolvimento foi encaminhado para obtermos um \software[]
simples em consumo de recursos.

As soluções foram implementadas utilizando, quando necessário, o paradigma
de programação Orientada a Objetos. O desenvolvimento procedural foi aplicado
com o intuito de realizar pequenas tarefas no sistema, tais como agendamento
de execução de tarefas, atualização das aplicações, desligamento automático etc.

Além disso, também fizemos uso de \webservices[] através da Internet, acessando
outros sistemas e complementando as funcionalidades desejadas. As linguagens
escolhidas para compor o sistema foram as linguagens \python[] e \shell.

\subsection{Módulo \textit{web}}

Foi necessária a implantação de um módulo \textit{web} para gerenciar as 
informações básicas do usuário. Neste sistema será possível cadastrar um 
paciente, seus alertas e contatos de emergência. O sistema foi desenvolvido
na linguagem \python[] e funciona na nuvem.

\subsection{Módulo \textit{service.notification}}\label{subsec:notification}

O módulo \textit{service.notification} foi desenvolvido utilizando a linguagem
\python[] e sua lógica faz com que constantemente execute um \textit{pooling} 
à aplicação em nuvem buscando por mensagens destinadas à esta instância de 
\software. Caso a pergunta tenha uma resposta negativa, o programa espera um 
tempo e refaz a requisição. Caso a resposta seja positiva, ou seja, exista 
alguma mensagem para aquele usuário, o programa faz o \download[] e agenda a
exibição. 

A resposta da requisição carrega informações pertinentes, um exemplo no formato
JSON é apresentado a seguir:

\begin{listing}[ht!]
\inputminted{json}{codigos/message.json}
\caption{Conteúdo da requisição de uma mensagem}
\label{lst:notification-json}
\end{listing}

O \software[] no STB verifica o conteúdo JSON, atentando-se para os campos 
\textit{startDateTime} e \textit{expirationDateTime} que representam quando 
iniciar a exibição da mensagem e quando não exibir mais a mensagem 
respectivamente, o tipo de notificação - se será do tipo \textit{pop-up} ou
caixa de diálogo -, a criticidade da mensagem - indicando se ela deve ser
exibida interrompendo a programação ou não e o conteúdo da mensagem. 

Após a correta leitura do conteúdo JSON, a mensagem é preparada para ser 
exibida. O código \ref{lst:notification-popup} faz uso da API do aplicativo
\xbmc[] para apresentar a mensagem no formato \textit{pop-up} representado pela
figura \ref{fig:notification-2}.

\begin{listing}[ht!]
\inputminted{python}{codigos/notification-popup.py}
\caption{Definição do método utilizado para apresentar as notificações no modo
\textit{pop-up}}
\label{lst:notification-popup}
\end{listing}

O método representado no código \ref{lst:notification-ok} faz uso de um outro 
método da API, obtendo o resultado visto na figura \ref{fig:notification-1}.

\begin{listing}[ht!]
\inputminted{python}{codigos/notification-ok.py}
\caption{Definição do método utilizado para apresentar as notificações no modo
ok}
\label{lst:notification-ok}
\end{listing}

A mensagem será exibida na tela de acordo com o tipo de notificação escolhida. O
tipo caixa de diálogo ocupa o centro da tela e tem um tamanho considerado, o 
intuito é tomar a atenção do usuário para a mensagem a ser exibida, já o tipo
\textit{pop-up} aparece no canto superior da tela - sendo esta a mais discreta 
das duas. Veja nas figuras \ref{fig:notification-2} e \ref{fig:notification-1} 
as duas possibilidades em ação.

\figuradupla{notification-2}{Notificação do tipo \textit{pop-up}}
{notification-1}{Notificação do tipo caixa de diálogo}

\subsection{Módulo \textit{service.dado}}\label{subsec:dado}

O módulo \textit{service.dado} foi desenvolvido utilizando a linguagem 
\python[]. O usuário pode iniciar a aplicação através do menu ``Sua Saúde''
(figura \ref{fig:tela-principal}) e em seguida escolher a opção ``Não está se 
sentindo bem?'' (figura \ref{fig:tela-principal-sua-saude}) no \stb[] através do 
controle remoto. O módulo apresenta uma tela inicial como a apresentada na 
figura \ref{fig:tela-principal-dado}.

O funcionamento deste módulo baseia-se no acionamento da aplicação através
do controle remoto do \stb[] e a interação através de suas telas no televisor
do usuário. 

\figurasimples{tela-principal-dado}
{Tela principal do módulo \textit{service.dado}}{14cm}

A partir da tela principal da aplicação, o usuário que se sente mal escolhe 
dentre um conjunto de sintomas, àqueles que representam o seu estado atual. Em
seguida, esses sintomas são enviados através de um \webservice[] para um
módulo de inferência. A conclusão é retornada para o STB, que a processa e 
apresenta para o usuário.

\subsection{Módulo \textit{service.panicbutton}}\label{subsec:panicbutton}

