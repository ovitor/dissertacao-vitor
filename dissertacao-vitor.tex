%% abtex2-modelo-trabalho-academico.tex, v<VERSION> laurocesar
%% Copyright 2012-<COPYRIGHT_YEAR> by abnTeX2 group at http://www.abntex.net.br/
%%
%% This work may be distributed and/or modified under the
%% conditions of the LaTeX Project Public License, either version 1.3
%% of this license or (at your option) any later version.
%% The latest version of this license is in
%%   http://www.latex-project.org/lppl.txt
%% and version 1.3 or later is part of all distributions of LaTeX
%% version 2005/12/01 or later.
%%
%% This work has the LPPL maintenance status `maintained'.
%%
%% The Current Maintainer of this work is the abnTeX2 team, led
%% by Lauro César Araujo. Further information are available on
%% http://www.abntex.net.br/
%%
%% This work consists of the files abntex2-modelo-trabalho-academico.tex,
%% abntex2-modelo-include-comandos and abntex2-modelo-references.bib
%%

% ------------------------------------------------------------------------
% ------------------------------------------------------------------------
% abnTeX2: Modelo de Trabalho Academico (tese de doutorado, dissertacao de
% mestrado e trabalhos monograficos em geral) em conformidade com
% ABNT NBR 14724:2011: Informacao e documentacao - Trabalhos academicos -
% Apresentacao
% ------------------------------------------------------------------------
% ------------------------------------------------------------------------

\documentclass[
	% -- opções da classe memoir --
	12pt,				% tamanho da fonte
	openright,			% capítulos começam em pág ímpar (insere página vazia caso preciso)
	twoside,			% para impressão em recto e verso. Oposto a oneside
	a4paper,			% tamanho do papel.
	% -- opções da classe abntex2 --
	%chapter=TITLE,		% títulos de capítulos convertidos em letras maiúsculas
	%section=TITLE,		% títulos de seções convertidos em letras maiúsculas
	%subsection=TITLE,	% títulos de subseções convertidos em letras maiúsculas
	%subsubsection=TITLE,% títulos de subsubseções convertidos em letras maiúsculas
	% -- opções do pacote babel --
	english,			% idioma adicional para hifenização
	french,				% idioma adicional para hifenização
	spanish,			% idioma adicional para hifenização
	brazil				% o último idioma é o principal do documento
	]{abntex2}

% ---
% Pacotes básicos
% ---
\usepackage{lmodern}			% Usa a fonte Latin Modern
\usepackage[utf8]{inputenc}		% Codificacao do documento (conversão automática dos acentos)
\usepackage[T1]{fontenc}		% Selecao de codigos de fonte.
\usepackage{lastpage}			% Usado pela Ficha catalográfica
\usepackage{indentfirst}		% Indenta o primeiro parágrafo de cada seção.
\usepackage{color}				% Controle das cores
\usepackage{graphicx}			% Inclusão de gráficos
\usepackage{float}				% Habilitar opção H nos elementos
\usepackage{microtype} 			% para melhorias de justificação
% ---
% Adicionado por Madson Dias
% ---

\usepackage{utils/ifce-modelo} 					% Pacote com informações da customização
\usepackage{utils/ifce-facilitadores} 					% Pacote com facilitadore 
\graphicspath{ {figuras/graficos/} {figuras/imagens/}}  % Inclusão dos paths para imagens
\usepackage{xstring} 									% Criar comandos com IF
\usepackage{caption}
\usepackage{subcaption}
%\usepackage[portuguese, ruled, linesnumbered]{utils/algorithm2e}
\usepackage{amsmath}
\usepackage{algorithm}
\usepackage[noend]{algpseudocode}
\makeatletter
\def\BState{\State\hskip-\ALG@thistlm}
\renewcommand{\ALG@name}{Algoritmo}
\renewcommand{\listalgorithmname}{Lista de algoritmos}
\makeatother

% Declaracoes em Português
\algrenewcommand\algorithmicend{\textbf{fim}}
\algrenewcommand\algorithmicdo{\textbf{faça}}
\algrenewcommand\algorithmicwhile{\textbf{enquanto}}
\algrenewcommand\algorithmicfor{\textbf{para}}
\algrenewcommand\algorithmicif{\textbf{se}}
\algrenewcommand\algorithmicthen{\textbf{então}}
\algrenewcommand\algorithmicelse{\textbf{senão}}
\algrenewcommand\algorithmicreturn{\textbf{devolve}}
\algrenewcommand\algorithmicfunction{\textbf{função}}



\algrenewcommand\algorithmicforall{\textbf{para todo}}
\algrenewcommand\algorithmicloop{\textbf{loop}}
\algrenewcommand\algorithmicrepeat{\textbf{repetir}}
\algrenewcommand\algorithmicuntil{\textbf{até}}
\algrenewcommand\algorithmicprocedure{\textbf{procedimento}}
\algrenewcommand\algorithmicrequire{\textbf{Exige:}}
\algrenewcommand\algorithmicensure{\textbf{Garante:}}
\newcommand{\Not}{\textbf{não }}





% Rearranja os finais de cada estrutura
%\algrenewtext{Procedure}{\textbf{Procedimento~} }
\algrenewtext{goto}{\textbf{vai para~} }
\algrenewtext{EndWhile}{\algorithmicend\ \algorithmicwhile}
\algrenewtext{EndFor}{\algorithmicend\ \algorithmicfor}
\algrenewtext{EndIf}{\algorithmicend\ \algorithmicif}
\algrenewtext{EndFunction}{\algorithmicend\ \algorithmicfunction}

% O comando For, a seguir, retorna 'para #1 -- #2 até #3 faça'
\algnewcommand\algorithmicto{\textbf{até}}
\algrenewtext{For}[3]%
{\algorithmicfor\ #1 $\gets$ #2 \algorithmicto\ #3 \algorithmicdo}
\setlist[itemize]{noitemsep, topsep=0pt, leftmargin=1.75cm}





% ---
% Pacotes adicionais, usados apenas no âmbito do Modelo Canônico do abnteX2
% ---
\usepackage{lipsum}				% para geração de dummy text
% ---

% ---
% Pacotes de citações
% ---
\usepackage[brazilian,hyperpageref]{backref}	 % Paginas com as citações na bibl
\usepackage[alf]{abntex2cite}	% Citações padrão ABNT

% ---
% CONFIGURAÇÕES DE PACOTES
% ---

% ---
% Configurações do pacote backref
% Usado sem a opção hyperpageref de backref
\renewcommand{\backrefpagesname}{Citado na(s) página(s):~}
% Texto padrão antes do número das páginas
\renewcommand{\backref}{}
% Define os textos da citação
\renewcommand*{\backrefalt}[4]{
	\ifcase #1 %
		Nenhuma citação no texto.%
	\or
		Citado na página #2.%
	\else
		Citado #1 vezes nas páginas #2.%
	\fi}%
% ---

% ---
% Informações de dados para CAPA e FOLHA DE ROSTO
% ---
% Informações do Autor e do trabalho
% ------------------------------------
\autor{Vitor de Carvalho Melo Lopes}
\titulo{<Título da Dissertação>}
\linha{Computação aplicada} % Inteligência Artificial, Computação aplicada ou Engenharia de Software

\orientador{<Nome do Orientador>}
\coorientador{<Nome do Coorientador>} % Se você tem um coorientador, descomente esta linha

% Professores convidados para a banca
% ------------------------------------
% 
%  - Caso tenha um terceiro professor convidado, remova os comentários

\nomeprofessorA{<Nome do Professor A>}
\instituicaoprofessorA{<Instituição do Professor A> (<Sigla A>)}

\nomeprofessorB{<Nome do Professor B>}
\instituicaoprofessorB{<Instituição do Professor B> (<Sigla B>)}

%\nomeprofessorC{<Nome do Professor C>}
%\instituicaoprofessorC{<Instituição do Professor C> (<Sigla C>)}

% ---


% ---
% Configurações de aparência do PDF final

% alterando o aspecto da cor azul
\definecolor{blue}{RGB}{0,0,0}

% informações do PDF
\makeatletter
\hypersetup{
     	%pagebackref=true,
		pdftitle={\@title},
		pdfauthor={\@author},
    	pdfsubject={\imprimirpreambulo},
	    pdfcreator={LaTeX with abnTeX2},
		pdfkeywords={abnt}{latex}{abntex}{abntex2}{trabalho acadêmico},
		colorlinks=true,       		% false: boxed links; true: colored links
    	linkcolor=blue,          	% color of internal links
    	citecolor=blue,        		% color of links to bibliography
    	filecolor=magenta,      		% color of file links
		urlcolor=blue,
		bookmarksdepth=4
}
\makeatother
% ---

% ---
% Espaçamentos entre linhas e parágrafos
% ---

% O tamanho do parágrafo é dado por:
\setlength{\parindent}{1.3cm}

% Controle do espaçamento entre um parágrafo e outro:
\setlength{\parskip}{0.2cm}  % tente também \onelineskip

% ---
% compila o indice
% ---
\makeindex
% ---

% ----
% Início do documento
% ----
\begin{document}

% Seleciona o idioma do documento (conforme pacotes do babel)
%\selectlanguage{english}
\selectlanguage{brazil}

% Retira espaço extra obsoleto entre as frases.
\frenchspacing

% ----------------------------------------------------------
% ELEMENTOS PRÉ-TEXTUAIS
% ----------------------------------------------------------
% \pretextual
\imprimircapa % Capa *
\imprimirfolhaderosto % Folha de rosto *

\begin{folhadeaprovacao}

    \begin{center}
    \includegraphics[width=2.5cm]{brasao_republica.jpg} \\
    \vspace{.5cm}
    {\ABNTEXchapterfont\large\imprimirinstituicao}
    \vspace{1cm}

    {\ABNTEXchapterfont\large\imprimirautor}

    % \vspace*{\fill}\vspace*{\fill}
    % \begin{center}
    %   \ABNTEXchapterfont\bfseries\Large\imprimirtitulo
    % \end{center}
    %\vspace*{\fill}
    
    %\hspace{.45\textwidth}
    % \begin{minipage}{.5\textwidth}
    %     \imprimirpreambulo
    %     {\large\imprimirdata}
    % \end{minipage}%
    %\vspace*{\fill}
   \end{center}
   Esta dissertação foi julgada adequada para a obtenção do título de Mestre em Ciência da
Computação, sendo aprovada pela Coordenação do Programa de Pós-Graduação em Ciência
da Computação do Instituto Federal de Educação, Ciência e Tecnologia do Ceará e pela banca
examinadora:

\vfill

  \begin{center}
  \begin{minipage}{10cm}
   \assinatura{\textbf{\imprimirorientadorRotulo~\imprimirorientador} \\ Instituto Federal de Educação, Ciência e Tecnologia do Ceará (IFCE)}

   \if \imprimircoorientador
    
   \else
    \assinatura{\textbf{\imprimircoorientadorRotulo~\imprimircoorientador}  \\ Instituto Federal de Educação, Ciência e Tecnologia do Ceará (IFCE)}
   \fi

   \if \nomeprofessorA \instituicaoprofessorA
    
   \else
    \assinatura{\textbf{\imprimirnomeprofessorA}  \\ \imprimirinstituicaoprofessorA}
   \fi

   \if \nomeprofessorB
    
   \else
    \assinatura{\textbf{\imprimirnomeprofessorB}  \\ \imprimirinstituicaoprofessorB}
   \fi
   
   \if \nomeprofessorC
    
   \else
    \assinatura{\textbf{\imprimirnomeprofessorC}  \\ \imprimirinstituicaoprofessorC}
   \fi
  \end{minipage}

  \end{center} 
  \vfill
   %\assinatura{\textbf{Professor}  \\ Instituto Federal do Ceará (IFCE)}
      
   \begin{center}
    \vspace*{\fill}
    {\large\imprimirlocal}
    \par
    {\large\imprimirdata}
    \vspace*{1cm}
  \end{center}
  
\end{folhadeaprovacao}


%\includepdf{pretextual/folhadeaprovacao.pdf} 	% Folha de aprovação (Depois da apresentação)
\begin{dedicatoria}
   \vspace*{\fill}
   \centering
   \noindent
   \textit{ Este trabalho é dedicado às crianças adultas que,\\
   quando pequenas, sonharam em se tornar cientistas.} \vspace*{\fill}
\end{dedicatoria} 				% Dedicatória
\begin{agradecimentos}

  Aos meus pais Roberval e Ângela e irmãos Júlia e Matheus pelas alegrias compartilhadas.

  À Sofia pelo companheirismo e presença sempre carinhosa e atenciosa.

  Ao meu orientador Antonio Mauro pelos ensinamentos ao longo deste período. 

  Aos colegas de mestrado Madson, David, Renan, Gustavo e Marcelo com os quais compartilhei
  conhecimentos diversos, alegrias e dificuldades vivenciadas no laboratório LAMP/PPGCC.

  Aos colegas do Laboratório de Redes (LAR) no Instituto Federal do Ceará - Campus Aracati
  que me ajudaram com debates e ideias.

  À todos os professores do mestrado pelos ensinamentos e pelos ricos debates. 

\end{agradecimentos}
 			% Agradecimento
\begin{epigrafe}
    \vspace*{\fill}
	\begin{flushright}
		\textit{%
		``Se a preocupação está em ter, ter, ter uma pessoa cada vez se preocupará
    menos em ser, ser, ser.''\\
		(José Saramago)}
	\end{flushright}
\end{epigrafe}
 				% Epígrafe
\setlength{\absparsep}{18pt} % ajusta o espaçamento dos parágrafos do resumo
\begin{resumo}

A expectativa de vida do brasileiro aumentou de 45,5 para 75,5 anos entre 1940
e 2015, segundo dados do IBGE. Idosos podem necessitar de maiores cuidados
hospitalares com maior tempo de recuperação, o que aumenta o gasto com saúde.
Nesse sentido, a Assistência Domiciliar à Saúde (ADS) - uma modalidade de
atenção ao paciente realizado em seu próprio domicílio, – apresenta-se como
opção mais adequada ao paciente. Além de reduzir gastos hospitalares a ADS
permite o contato mais intenso com familiares na recuperação do paciente.
Estudos realizados por profissionais de saude identificaram problemas no
cenario de ADS, tais como a dificuldade que possuem os diversos atores
envolvidos: o paciente, o cuidador, familiares, equipe de saude e gestores. O
cuidador mereceu especial atenção tendo em vista que ele nem sempre possui os
conhecimentos e as habilidades necessárias para atendendimento do doente. Este
trabalho descreve a visão computacional desses cenários de saúde e propõe a TV
Health das Coisas, uma solução inteligente para o ambiente de ADS baseada no
modelo brasileiro de TV digital e na tecnologia Internet das Coisas. Esta
solução tem como substrato um sistema embarcado associado à uma TV digital
(\textit{hub}) que coleta dados de diversos sensores existentes no ambiente de ADS, além
de servir como interface para o paciente a diversas aplicações interativas. Os
dados coletados alimentam modelos inteligentes de ontologia que permitem a
inferência local (\textit{hub}) de informações necessárias à tomada de decisão dos
atores do ADS: paciente, cuidador, equipe de saúde, familiares, etc. Outro
aspecto inteligente do TV Health das Coisas diz respeito a aprendizagem de
máquinas via o sistema \nextsaude, do qual a TV Health das Coisas é um
componente. Foi definida para este cenário de ADS uma arquitetura orientada a
contexto (\textit{context-aware concept}), enriquecida com conceitos da
plataforma \textit{OpenIoT} (Plataforma de Internet das Coisas).  Um protótipo
já se encontra operacional com suas arquiteturas funcional e de engenharia
formalmente especificadas.  Assim, paciente, cuidador, familiares, equipes de
saúde e gestores podem, de forma seletiva, receber alertas, informações etc, via
TV dispositivos móveis e outros mecanismos integrados ao sistema. A TV Health
das Coisas resultou em um projeto financiado pela Embrapii e está sendo
negociado para implantação em um plano de saúde.

 \textbf{Palavras-chave}: atenção domiciliar. internet das coisas. computação ubíqua. computação pervasiva.
 \end{resumo}
 					% Resumo em português
\begin{resumo}[Abstract]
 \begin{otherlanguage*}{english}

According to IBGE, the life expectation of Brazillian people increased from
45,5 to 75,5 years between 1940 and 2015. Because of inherent fragility
of age, the elderly demands more hospital care, with more recovery time
which increases the money that the Estate spend in each patient. The
Home Care - health care modality that treats the patient in his own
home - presents as more adequate option for the patient, besides the
reduction of hospital expenses. One of the problems identified in the
scenario of home care it's the need to help the patient and the caregiver,
once that the caregiver not always have necessary knowlegde to attend
the patient. His does not always have the caregiver figure daily or in
some situations, the caregiver doesn't exist. The Information Technology
can provide mechanisms and instruments for this type of service.
Considering that the television is an equipment present in practically all
Brazillian residences, this work proposes TV Health of Things, a smart
solution for the ADS environment based on the Brazilian model of digital TV
and technology Internet of Things. TV Health of Things is a ``Set-Top Box''
(digital converter) system that collects data from various sensors in the
home care environment and serves as an interface to a variety of interactive
applications. These collected data feed ontology models that allow the
inference of information necessary for the decision making of the ADS actors
(patient, caregiver, health team, family, etc). Therefore, the various actors
mentioned can receive alerts, information, etc. via TV, web applications or 
mobile devices integrated into the system. The proposal is a component of 
\nextsaude[], a health interoperability platform based on the OpenEHR
standard. It makes use of archetypes to integrate with the \nextsaude[] platform
and thus to share services available by the platform (regulation, pharmacy
etc). TV Health of Things is a context-aware (\textit{context-aware concept}) 
architecture, enriched with concepts from the \textit {OpenIoT} platform. 
A prototype is already operational and its IoT version has its formally 
specified functional and engineering architectures.


\textbf{Keywords}: home care. iot. ubiquitous and pervasive computing.
 \end{otherlanguage*}
\end{resumo}
 				% Resumo em inglês
% inserir lista de ilustrações
\pdfbookmark[0]{\listfigurename}{lof}
\listoffigures*
\cleardoublepage
% inserir lista de tabelas
\pdfbookmark[0]{\listtablename}{lot}
\listoftables*
\cleardoublepage
% ---
% inserir lista de algoritmos
\pdfbookmark[0]{\listalgorithmname}{loa}
\listofalgorithms
\cleardoublepage
% ---
\begin{siglas}
  \item[ABS] Anti-lock Breaking System
  \item[ADS] Assistência Domiciliar à Saúde 
  \item[SAMDU] Serviço de Assistência Médica Domiciliar e de Urgência 
  \item[SUS] Sistema Único de Saúde 
  \item[SUDS] Sistema Único e Descentralizado de Saúde 
  \item[STB] Set-Top Box
\end{siglas}  			% Lista de abreviaturas e siglas
\begin{simbolos}
\item[$ \Gamma $] Letra grega Gama
\item[$ \Lambda $] Lambda
\item[$ \zeta $] Letra grega minúscula zeta
\item[$ \in $] Pertence
\end{simbolos} 				% Lista de símbolos
% inserir o sumario
\pdfbookmark[0]{\contentsname}{toc}
\tableofcontents*
\cleardoublepage
% ---



% ----------------------------------------------------------
% ELEMENTOS TEXTUAIS
% ----------------------------------------------------------
\textual

\chapter{Introdução}\label{cap:introducao}

O Brasil vem passando por um processo de envelhecimento da população e
um aumento da expectativa de vida crescente desde a década de 1960. 
Com os atuais índices, a taxa do envelhecimento populacional atingirá, 
em 2025, cerca de 15\% da população brasileira com indivíduos acima 
de 60 anos. 

Os mais idosos, por conta da fragilidade inerente à idade, necessitam de 
cuidados especiais na hospitalização, além de demandarem mais tempo na 
recuperação. Muitas vezes há uma demanda de vários profissionais de uma equipe 
médica multidisciplinar para se recuperar totalmente.

Essa situação afeta políticas públicas dos governos municipais, estaduais e federal -
que, segundo a Constituição Federal, deve prover atendimento hospitalar
universal à população - tornando o gasto público com hospitalização de idosos
maiores a cada ano.

Portanto, esse processo que a população brasileira vem sofrendo, ocasiona mudanças 
nos paradigmas de atendimento à saúde. Nessa perspectiva, surge a atenção domiciliar
%Uma outra abordagem para casos semelhantes é a atenção domiciliar
ou \textit{home care}, modelo definido como o tratamento do paciente 
em seu próprio lar, com a presença ou não de um cuidador - figura responsável
por acompanhar o idoso em suas atividades diárias, ao assumir um papel 
de fundamental importância no acompanhamento do paciente em seu cotidiano. 

Pesquisas mostram que esse método traz benefícios pois o 
paciente encontra-se em um ambiente conhecido e está na presença contínua de
seus familiares.

Assim, pesquisas realizadas indicam uma mudança gradativa no modelo de
tratamento de idosos. A escolha da atenção domiciliar em detrimento da
hospitalização traz benefícios sociais, psicológicos e
econômicos para o paciente. %, mas também benefícios econômicos para o Estado.

No que concerne às políticas públicas desenvolvidas pelo Estado, 
a vantagem é de reduzir os custos com internação. Estudos revelam 
que é possível reduzir custos de internação hospitalar com uma 
abordagem em atenção domiciliar nos casos de menor gravidade, 
ou seja, casos em que o paciente não corre risco de morte. 
% o aumento de casos em que o paciente pode ser tratado v

\section{Motivação para a Dissertação}\label{sec:motivacao}

Observo na sociedade atual, através de notícias divulgadas, nos meios
de comunicação, de conversas informais, de filmes etc que está cada
vez mais crescente a preocupação com os idosos, haja vista o
aumento da expectativa de vida. Me pergunto como é possível essas pessoas
terem uma qualidade de vida na velhice se sabemos que, nessa idade,
os problemas de saúde são acentuados, como falta de autonomia,
dificuldades de locomoção, preocupação com remédios etc.

Em muitos casos, cresce o número de internações hospitalares e,
consequentemente, os gastos do Estado. Visando enfrentar essa 
realidade, a comunidade acadêmica, % a partir de meados dos anos tais,
começa a se pesquisar sobre os cuidados aos idosos no espaço domiciliar.

A linha de pesquisa de computação aplicada à saúde é vista como uma parte
importante para a melhoria na qualidade dos serviços prestados na área.
Com a pesquisa realizada, a comunidade acadêmica possibilitou o avanço
de atenção domiciliar, pesquisando e desenvolvendo sistemas inteligentes
que auxiliem no tratamento e ajudem os envolvidos.

\section{Descrição do problema}\label{sec:descricao-problema}



A partir do descrito, se pode identificar que a internação domiciliar
de pacientes idosos é possível. O auxílio ao idoso, ao cuidador
ou ainda, à equipe médica que o acompanha carece de soluções eficientes
e acessíveis financeiramente.

\section{Objetivos Geral e Específicos}\label{sec:objetivos}

Em seu sentido mais particular, os seguintes objetivos específicos são:

\begin{itemize}
	\item ...; 
	\item ...;
\end{itemize}

\section{Produção científica}\label{sec:producao}
Durante este projeto de mestrado, os seguintes trabalhos científicos foram aceitos e publicados, a saber:

\begin{itemize}
	\item \textbf{Einstein, A.}, 1905. \textbf{The photoelectric effect}. Ann. Phys, 17(132), p.4;
	\item \textbf{Einstein, A.}, 1904. \textbf{Zur allgemeinen molekularen Theorie der Wärme}. Annalen der Physik, 319(7), pp.354-362.
\end{itemize}

\section{Estrutura da Dissertação}\label{sec:estrutura}
\lipsum[1]
 				% Capítulo 0 - Introdução
\chapter{Fundamentação Teórica}\label{cap:fundamentacao-teorica}

Este capítulo apresenta a fundamentação teórica, separada em aspectos de saúde e
tecnológicos. As seções de saúde abordam termos como Assistência Domiciliar à
Saúde e suas particularidades, além de explanar, brevemente, sobre a história da
hospitalização no Brasil. A seção \ref{sec:aspectos-tecnologicos} trata das
tecnologias e conceitos tecnológicos utilizados neste trabalho.

\section{Aspectos de Saúde}\label{sec:aspectos-de-saude}

Segundo Inaiá Mello, no livro ``Humanização nos Hospitais do Brasil''
\cite{inaia2008}, uma das primeiras instituições voltadas para o cuidado com a
saúde, foi a fundação da Santa Casa de Misericórdia de Santos, em 1543, cuja
principal atividade era prestar assistência de cunho caritativo a pessoas pobres
e desabrigados. Tal estrutura permanece inalterada até o final do século XIX,
e início do século XX.

Com o Governo Getúlio Vargas, a partir de 1930, ocorre no Brasil o processo de
industrialização, que trouxe crescimento rápido e desordenado às cidades,
principalmente São Paulo e Rio de Janeiro. As transformações econômicas e
sociais resultantes desse processo, a falta de saneamento básico, a pobreza etc
foram motivos para que parte da população reivindicasse mais atenção do Governo
em relação aos cuidados de saúde \cite{carvalho1984}.

Carvalho indica, ainda, que apesar dessa pressão, não existiu uma política de
saúde clara por parte das autoridades. Muitas vezes, algumas ações se voltavam
para a criação de condições sanitárias mínimas, que se mostravam limitadas
frente às reais necessidades da população. Dessa forma, as décadas subsequentes
não foram significativas no tocante a uma ampliação dos serviços de saúde
oferecidos à população.

Conforme análise de Inaiá Mello, ainda nos anos 1970, surgem tentativas de
universalizar o acesso à assistência à saúde. Alguns Programas e Sistemas foram
iniciados, sendo válido citar, (i) Sistema Único e Descentralizado de Saúde, o
SUDS e (ii) o Sistema Único de Saúde, o SUS. Entretanto, em virtude da vigência
da ditadura militar, implantada em 1964, que levou o país a vivenciar um estado
de exceção, tais propostas não conseguiram se concretizar. Assim, somente no
período da redemocratização, ocorrida em meados dos anos 1980, é que o SUS foi
criado oficialmente pela Constituição Federal de 1988, Lei 8080/90
\footnote{Acessível em
\url{http://www.planalto.gov.br/ccivil_03/leis/L8080.htm}}, com o objetivo de
garantir à população Brasileira, o acesso universal às ações e serviços de
saúde.

Paralelo à criação do Sistema Único de Saúde (SUS), ocorreu o avanço tecnológico
que alcançou a prática médica, aperfeiçoando, com isso a infraestrutura
hospitalar. Dessa forma, os hospitais deixaram de ser espaços para abrigarem
pobres desamparados e passaram a proporcionar tratamentos mais elaborados. O
hospital passa a oferecer  procedimentos cirúrgicos, atendimentos de urgência,
internações, tornando a instituição dispendiosa. Os estudiosos começam a
identificar a possibilidade de tratamentos e cuidados com a  saúde que não
estejam, necessariamente, vinculados ao ambiente hospitalar.

Como consequência, surgiram diversas mudanças no atendimento, onde a Assistência
Domiciliar à Saúde (ADS) se tornou uma modalidade disponível.

\subsection{Assistência Domiciliar à Saúde}
\label{subsec:assistencia-domiciliar-a-saude}

A Assistência Domiciliar à Saúde (ADS) divide-se basicamente em grupos de
enfermagem e fisioterapia - nas modalidades mais básicas - até um atendimento
multiprofissional, possibilitando um apoio ao paciente como um todo. A ADS pode
ser provida tanto pelo setor privado quanto pelo setor público
\cite{amaral2001assistencia}.

Os primeiros registros da ADS no Brasil surgem em 1967, na cidade de São
Paulo, no Hospital do Servidor Público. O principal objetivo dessa abordagem
era a liberação de leitos no hospital, levando para o domicílio procedimentos
básicos, de baixa complexidade clínica.

Já na década de 90, segundo Tavolari, houve um aumento considerável na
quantidade de empresas privadas provendo o serviço de ADS, apenas 5 empresas
prestavam esse tipo de  serviço, já em 1999, esse número subiu para mais de 180
\cite{tavolari2000desenvolvimento}.

Amaral et al define a ADS como uma sequência de serviços residuais a serem
oferecidos, depois que o indivíduo já recebeu atendimento primário e prévios,
ou seja, aquele que já recebeu atendimento primário com consequente diagnóstico
e tratamento.

Amaral e Tavolari lembram, ainda, que o atendimento domiciliar pode acelerar a
recuperação do  paciente e promover a redução de custos hospitalares, além de
ser uma solução mais  humanista para os portadores de doenças crônicas ou de
longa duração, frente à  hospitalização. Dessa forma, a assistência domiciliar à
saúde, tem como objetivos principais: (1) humanização no atendimento; (2) maior
rapidez na recuperação do paciente, devido à proximidade com os seus familiares;
(3) diminuição do risco de infecção hospitalar; (4) Otimização de leitos
hospitalares para pacientes que deles necessitem; e (5) Redução do custo/dia da
internação.

\subsubsection{Envolvidos}\label{subsubsec:envolvidos}

Para o entendimento geral da modalidade ADS, faz-se necessário separar e
explicar a atuação de cada um dos envolvidos. O paciente, componente principal
é aquele que sofre algum problema físico ou mental. A família é responsável
por prover um ambiente propício a melhora do paciente, já que o círculo familiar
é a base e raiz da estrutura social.

Outra figura importante na atenção ao paciente é a equipe multiprofissional -
composta de médicos, enfermeiros, psicólogos, terapeutas, assistentes sociais,
farmacêuticos, cuidadores e outros - visando propiciar, através da integração
das diversas áreas de conhecimentos, a melhoria efetiva do paciente.

O cuidador, muitas vezes é, um familiar, alguém próximo à família ou alguém
contratado. Seu papel principal é cuidar do paciente, ajudando nas tarefas
diárias, como alimentação, lazer, socialização, limpeza do paciente, entre
outros \cite{amaral2001assistencia}.


\subsubsection{Nomenclatura}\label{subsubsec:nomenclatura}

Apesar de não haver uma definição formal, a Assistência Domiciliar à Saúde pode
ser separada em três modalidades, diferenciadas, principalmente, pelo grau de 
atenção dispensada ao paciente. 

É defendido por Tavolari, Fernandes e Medina, que o termo Assistência Domiciliar
à Saúde é genérico e referente a todo e qualquer procedimento de saúde realizado
em domicílio, não importando o grau de complexidade. Já o termo Internação
Domiciliar é aplicado quando, dos procedimentos realizados, o cuidado intensivo
e multiprofissional é perceptível, caracteriza-se ainda, pelo transporte de
parte da estrutura hospitalar para o domicílio do paciente. O paciente, nesse
caso, é categorizado com complexidade alta ou moderada.

Já no Atendimento Domiciliar, caracteriza-se o paciente num estado de menor 
complexidade médica, o cuidado é realizado por uma equipe multiprofissional ou
não \cite{tavolari2000desenvolvimento}.  

Giacomozzi contrapõe, e cita como termo geral, Atenção Domiciliar à Saúde, 
que se refere na realidade, a uma categoria da Atenção Domiciliar, englobando
o atendimento, a visita e as internações domiciliares \cite{giacomozzi2006pratica}.

A Atenção Domiciliar à Saúde é, ainda, considerada por Giacomozzi, ``um
componente do \textit{continuum} dos cuidados à saúde, pois os serviços de
saúde são oferecidos ao indivíduos e sua família [...] minimizando os efeitos
das incapacidades ou doenças, incluindo aquelas sem perspectiva de cura.''
Já o termo Assistência Domiciliar à Saúde, é formado por atividades de cunho 
ambulatorial, adicionado a essa categorização, a modalidade Visita Domiciliar
prioriza verificar a realidade do paciente, além de realizar ações educativas.  

A partir dessa separação, podemos propor uma solução mais específica para os 
casos de Internação Domiciliar.

% \subsection{Urgência e Emergência}\label{subsec:urgencia-emergencia}
% \lipsum[1]

% \subsection{Atenção domiciliar}\label{subsec:atencao-domiciliar}
% \lipsum[1]

\section{Aspectos Tecnológicos}\label{sec:aspectos-tecnologicos}
 				% Capitulo 1 - Fundamentação Teórica

% ----------------------------------------------------------
% Finaliza a parte no bookmark do PDF
% para que se inicie o bookmark na raiz
% e adiciona espaço de parte no Sumário
% ----------------------------------------------------------
\phantompart

% ---
% Conclusão
% ---
\chapter{Conclusões e trabalhos futuros} \label{cap:conclusao}
% ---

Apesar da proposta descrita nesta dissertação ter como base um projeto de
pesquisa e desenvolvimento no setor elétrico fomentado pela Agência Nacional de
Energia Elétrica (ANEEL) em paceria com a ENEL (antiga Companhia Energética do
Ceará - COELCE) chamado METAL (Mecanismo de Comunicação entre Concessionárias e
Clientes baseada na TV Digital) teve como ponto de partida o projeto de pesquisa
\nextsaude.

O projet METAL (número do projeto: PD-0039-0062-2012), foi executado pela
empresa cearense CRAFF. Destinava-se a pesquisa e desenvolvimento de uma
plataforma em nuvem para gerenciamento e um \stb[] com aplicações interativas
para atender ao público alvo da fornecedora de energia elétrica no Estado do
Ceará. Como resultado, a empresa disponibilizou o sistema na nuvem e aplicações
interativas através do STB que permitiam o cliente ter acesso aos serviços da
companhia através da sua televisão.

Tomando como base a assertiva que um dos grandes meios de entretenimento e
informação da população brasileira é a televisão e o advento do Sistema
Brasileiro de Televisão Digital com a interatividade inerente ao sistema, foi
pensado que a televisão pudesse ser utilizada também como um meio de interação
com o usuário. Para fornecer mais funcionalidades foi necessário a aplicação do
STB com poder de processamento e conexão à Internet disponbilizados na casa do
usuário.   

O projeto de pesquisa \nextsaude[] (número do projeto: 6424611/2014) composto
por três módulos: (1) \textit{hardware}, (2) gerenciamento de ontologias e (3)
aplicação,  se propôs a desenvolver soluções especializadas e gerar inovações
tecnológicas de interoperabilidade sintática, integração semântica de dados,
tomada de decisão automatizada e o consequente aconselhamento aos atores
envolvidos (do usuário ao gestor) em sistemas público de saúde. Como prova de
conceito, o projeto fez uso do STB e sensores em conjunto com a televisão digital
para focar na  assistência domiciliar à saúde para doentes, idosos e cuidadores.

O autor desse trabalho foi responsável pela execução do módulo de
\textit{hardware} que era constituído de um sistema embarcado e aplicação de
captura de dados do paciente em sua residência (assistência domiciliar à
saúde), seja ela através de sensores ou por meio de aplicativos com entrada de
dados. 

Ao término do projeto \nextsaude[] o módulo de \textit{hardware} foi renomeado para
TV-Health e era composto de um \stb[], aplicações interativas e sistema de alertas 
e avisos. Os outros dois módulos receberam o nome de plataforma \nextsaude.

Além disso, foram realizadas reuniões com o laboratório Telemídia da PUC - Rio
no sentido de distribuir as aplicações interativas de saúde desenvolvidas para o TV
Health nos STBs entregues aos beneficiários do programa Bolsa Família do Governo Federal.
Essa ação colocaria as aplicações em uso para milhões de brasileiros
melhorando assim o acesso a informação e saúde.

%Além disso, a coordenação do projeto disponibilizou a plataforma
%\nextsaude[] e todos os seus subsistemas para que secretarias de saúde dos
%municípios pudessem implantar o sistema em suas gestões.

%Durante a execução do projeto, 

Por fim, percebeu-se a necessidade de um ambiente de atenção domiciliar à saúde
de qualidade, a importância do cuidador no auxílio ao idoso/doente além da
qualidade de vida que é necessária para terceira idade ou doentes crônicos.
Através das pesquisas, nota-se a importância da tecnologia da
informação para prover uma solução mais impactante para o usuário. 

A percepção de instrumentalização dos atores (cuidador, equipe de saúde, idoso
ou doente) através da tecnologia também surgiu nesse período. Aplicações
móveis, interface \web[] e acompanhamento do doente através da Internet foram
tópicos importantes para o projeto.

À época da entrega da primeira versão do TV Health, percebeu-se a partir da
pesquisa do estado da arte de tecnologias emergentes que a Internet das Coisas
irá revolucionar a nossa sociedade. Diversos estudos apontam para uma transformação
profunda nas áreas da indústria e de serviços.

Não será diferente com o ambiente de atenção domiciliar como nós o conhecemos. Através
de ``coisas'' conectadas dotadas de sensores, poderemos melhorar a experiência
de usuário. Será possível, então, entregar um serviço bem mais transparente
para o usuário final e estaria alinhado com as novas soluções propostas ao
redor do mundo.

A arquitetura proposta para atenção domiciliar à saúde faz uso de tecnologias
já conhecidas, tais como sistema sensível a contexto e sistemas embarcados, e
ontologias. Também incluiu Internet das Coisas. Auxiliar doente, cuidador,
parentes e equipe de saúde através do STB e dispositivos móveis foi um dos
objetivos deste trabalho. 

A arquitetura considera a utilização do STB como um \textit{hub} de
comunicação. Todas as ``coisas'', ou seja, sensores coletando sinais vitais do
paciente, sensores coletando informações do ambiente e dispositivos móveis
contribuindo para um sistema repleto de informações sobre o ator (idoso
ou doente). Além disso a arquitetura contempla a tomada de decisão, tanto
local, quanto na nuvem, através da utilização de ontologias e aprendizagem de
máquinas.

Com essas características o objeto de estudo foi aprovado como um
projeto a ser executado em parceria com o Polo Emprapii de Inovação/IFCE e
a empresa cearense EXATA. Além disso, uma alteração do objeto de estudo
está em fase de negociação com um plano de saúde privado para implantação
da TV Health das Coisas em unidades hospitalares.

\section{Produção científica}\label{sec:producao}  

Durante este projeto de mestrado, os seguintes trabalhos científicos foram
aceitos e publicados, a saber:

\begin{itemize}
  \item \textbf{LOPES, V. C. M.}; QUEIROZ, E.; FREITAS, N.; OLIVEIRA, M.; MONTEIRO, O. TV-Health:
  A Context-Aware health Care Application for Brazilian Digital TV. In ACM.
  \textit{Proceedings of the 22nd Brazilian Symposium on Multimedia and the Web (Webmedia)}. 
  Teresina, Brasil. 2016, pp. 103-106.

  \item \textbf{LOPES, V. C. M.}; ROCHA, E.; QUEIROZ, E.; FREITAS, N.; VIANA, D.; OLIVEIRA, M. VITESSE 
  - more intelligence with emerging technologies for health systems. In: IEEE. 
  \textit{2016 7th International Conference on the Network of the Future (NOF)}. Buzios, Brasil. 2016, pp. 1-3.
\end{itemize}

Além do seguinte artigo aceito para publicação:

\begin{itemize}
  \item \textbf{LOPES, V. C. M.}; MOTA, H.; OLIVEIRA, M.; CARVALHO, G. Towards an Emergency/Urgency
    approach based on the Brazilian Digital TV. \textit{Multi Conference on Computer Science 
    and Information Systems (MCCSIS)}. Ilha da Madeira, Portugal, 2016.
\end{itemize}

\section{Trabalhos futuros} \label{sec:trabalhos-futuros}

Uma vez definida a arquitetura e as características gerais que um sistema de
atenção domiciliar à saúde atual deve ter, podemos propor a inclusão de
novas funcionalidades a arquitetura apresentada, tais como novos métodos
de aquisição de dados, com a ajuda de processamento de linguagem natural.

A necessidade de gerenciar corretamente todos os dispositivos que a Internet
das Coisas proporciona cria uma linha de pesquisa em que é necessário aprimorar
a comunicação e gerência de IoT e Computação em Nuvem.

Para finalizar, é possível realizar estudos relacionados ao tratamento em domicílio de
doenças específicas - tal como o Parkinson, portadores de pressão alta,
diabetes etc - englobando o uso de aprendizagem de máquinas com o intuito de detectar
antecipadamente o surgimento de uma doença específica.
 				% Capítulo de introdução

% ----------------------------------------------------------
% ELEMENTOS PÓS-TEXTUAIS
% ----------------------------------------------------------
\postextual
% ----------------------------------------------------------

% ----------------------------------------------------------
% Referências bibliográficas
% ----------------------------------------------------------
\bibliography{referencias-vitor}

% ----------------------------------------------------------
% Glossário
% ----------------------------------------------------------
%
% Consulte o manual da classe abntex2 para orientações sobre o glossário.
%
%\glossary

% ----------------------------------------------------------
% Apêndices
% ----------------------------------------------------------

% ---
% Inicia os apêndices
% ---
\begin{apendicesenv}

% Imprime uma página indicando o início dos apêndices
\partapendices
% Insere os apêndices
\chapter{TV-Health} \label{apendice:tv-health}
% ---

\section{Software utilizado} \label{apendice:software-utilizado}

\figurasimples{tela-principal}{Tela principal do STB. Destaque para o menu
  ``Sua Saúde''.}{14cm}

\figurasimples{tela-principal-sua-saude}{Tela principal do menu ``Sua Saúde''.}{14cm}

\figurasimples{notification-1}{Notificação do tipo caixa de diálogo.}{16cm}

\figurasimples{notification-2}{Notificação do tipo \textit{pop-up}.}{16cm}

\figurasimples{tela-principal-dado}{Tela principal do módulo 
\textit{service.dado}.}{16cm}

\figurasimples{notificacao-botao-de-panico}{Aviso para tranquilizar o usuário em 
situação de emergência.}{16cm}

\section{Diagramas UML} \label{apendice:esp-formal}

\subsection{Módulo \textit{service.notification}} 
\label{apendice:notification}

\figurasimples{web-db}{Diagrama Entidade/Relação para 
módulo \textit{web}.}{12cm}

\subsection{Módulo \textit{service.notification}} 
\label{apendice:notification}

\figurasimples{notification-uc}{Diagrama de Caso de Uso para
módulo \textit{service.notification}.}{16cm}

\figurasimples{notification-ad}{Diagrama de Atividades para
módulo \textit{service.notification}.}{14cm}

\figurasimples{notification-sd-dialogbox}{Diagrama de Sequência para
módulo \textit{service.notification}. Alerta do tipo caixa de
diálogo.}{16cm}

\figurasimples{notification-sd-popup}{Diagrama de Sequência para
módulo \textit{service.notification}. Alerta do tipo 
\textit{pop-up}.}{16cm}

\subsection{Módulo \textit{service.dado}}
\label{apendice:dado}

\figurasimples{dado-uc}{Diagrama de Caso de Uso para
módulo \textit{service.dado}.}{16cm}

\figurasimples{dado-ad}{Diagrama de Atividades para
módulo \textit{service.dado}.}{12cm}

\figurasimples{dado-sd}{Diagrama de Sequência para
módulo \textit{service.dado}.}{16cm}

\subsection{Módulo \textit{service.panicbutton}}
\label{apendice:panicbutton}

\figurasimples{panicbutton-uc}{Diagrama de Caso de Uso para
módulo \textit{service.panicbutton}.}{16cm}

\figurasimples{panicbutton-ad}{Diagrama de Atividades para
módulo \textit{service.panicbutton}.}{12cm}

\figurasimples{panicbutton-sd}{Diagrama de Sequência para
módulo \textit{service.panicbutton}.}{10cm}



%\include{apendices/apendice-b}
%\include{apendices/apendice-c}
%\include{apendices/apendice-d}




\end{apendicesenv}
% ---


% ----------------------------------------------------------
% Anexos
% ----------------------------------------------------------

% ---
% Inicia os anexos
% ---
\begin{anexosenv}

% Imprime uma página indicando o início dos anexos
\partanexos
% Insere os anexos
\chapter{Morbi ultrices rutrum lorem.}
% ---
\lipsum[30]

%\chapter{Cras non urna sed feugiat cum sociis natoque penatibus et magnis dis
parturient montes nascetur ridiculus mus}
% ---

\lipsum[31]
%\include{anexos/anexo-c}
%\include{anexos/anexo-d}

\end{anexosenv}

%---------------------------------------------------------------------
% INDICE REMISSIVO
%---------------------------------------------------------------------
\phantompart
\printindex
%---------------------------------------------------------------------

\end{document}
